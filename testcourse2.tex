\documentclass[12pt]{article}

\usepackage{edXpsl}	% edX
\usepackage{amsmath, amsthm, amsfonts, amssymb, color, mathrsfs, comment}
\usepackage{cancel}

%------------------------------------------
\parindent=0pt
\parskip=1ex

\begin{document}
%-----------------------------------------------%

%\documentclass[12pt]{article}
%
%\usepackage{edXpsl}	% edX
%\usepackage{amsmath, amsthm, amsfonts, amssymb, color, mathrsfs, comment}
%\usepackage{cancel}
%
%%------------------------------------------
%\parindent=0pt
%\parskip=1ex
%
%\begin{document}
%%-----------------------------------------------%

\newcounter{edXtext}
\newcounter{edXproblem}
\newcounter{edXvideo}
\newcounter{showhide}
\newcounter{psetproblem}
%\newcounter{edXvertical}


%If seems \ifplastex needs to go after the \begin{document}
% \newif\ifplastex 
% \plastexfalse
% \ifplastex
  %% Many of these could be better handled in python. When I get the chance to understand plastex better I'll do that. JMO
   % \def\USING{USING PLASTEX FIX MACROS}
   \def\cancel#1{#1}
%   \def\footnote#1{} %\par {\small{\color{red} FOOTNOTE:} #1}}
   \def\mylabel#1{\label{#1}\tag{\value{equation}}}

   \def\displaynametag{display_name}

   %\def\includesvg{\includegraphics}
   %\def\includegraphics{\edXincludegraphics}

   \newcounter{mynumbereditem}
   \newcounter{temp}
   \def\mysecnum{\arabic{section}}
   \def\mynumbereditemnum{\arabic{mynumbereditem}}
   \def\mynumbereditemlabel{\mysecnum.\mynumbereditemnum}

   \def\mysection#1{\section{#1}}

   \newenvironment{theorem}{\addtocounter{mynumbereditem}{1}%
\textbf{Theorem \mynumbereditemlabel.} }{\par\vspace{.5in}}
   \newenvironment{proposition}{\addtocounter{mynumbereditem}{1}%
\textbf{Proposition \mynumbereditemlabel.} }{\par\vspace{.5in}}
   \newenvironment{lemma}{\addtocounter{mynumbereditem}{1}%
\textbf{Lemma \mynumbereditemlabel.} }{\par\vspace{.5in}}
   \newenvironment{definition}{\addtocounter{mynumbereditem}{1}%
\textbf{Definition \mynumbereditemlabel.} }{\par\vspace{.5in}}
   \newenvironment{remarks}{\addtocounter{mynumbereditem}{1}%
\textbf{Remarks \mynumbereditemlabel.} }{\par\vspace{.5in}}
   \newenvironment{corollary}{\addtocounter{mynumbereditem}{1}%
\textbf{Corollary \mynumbereditemlabel.} }{\par\vspace{.5in}}
   \newenvironment{remark}{\addtocounter{mynumbereditem}{1}%
\textbf{Remark \mynumbereditemlabel.} }{\par\vspace{.5in}}
   \newenvironment{examples}{\addtocounter{mynumbereditem}{1}%
\textbf{Examples \mynumbereditemlabel.} }{\par\vspace{.5in}}
   \newenvironment{example}{\addtocounter{mynumbereditem}{1}%
\textbf{Example \mynumbereditemlabel.} }{\par\vspace{.5in}}
   \newenvironment{exercise}{\addtocounter{mynumbereditem}{1}%
\textbf{Exercise \mynumbereditemlabel.} }{\par\vspace{.5in}}

   \newenvironment{stheorem}{\textbf{Theorem. }}{\par\vspace{.5in}}
   \newenvironment{sproposition}{\textbf{Proposition. }}{\par\vspace{.5in}}
   \newenvironment{slemma}{\textbf{Lemma. }}{\par\vspace{.5in}}
   \newenvironment{sdefinition}{\textbf{Definition. }}{\par\vspace{.5in}}
   \newenvironment{sremarks}{\textbf{Remarks. }}{\par\vspace{.5in}}
   \newenvironment{scorollary}{\textbf{Corollary. }}{\par\vspace{.5in}}
   \newenvironment{sremark}{\textbf{Remark. }}{\par\vspace{.5in}}
   \newenvironment{sexamples}{\textbf{Examples. }}{\par\vspace{.5in}}
   \newenvironment{sexample}{\textbf{Example. }}{\par\vspace{.5in}}
   \newenvironment{sexercise}{\textbf{Exercise. }}{\par\vspace{.5in}}


% \def\dpa{weight="1" showanswer="attempted" attempts="3"}

\def\beginedxsequential#1#2{\begin{edXsection}{#1}[#2 url_name="\edxbaseoutputname-sequential"]}

\def\endedxsequential{\end{edXsection} \setcounter{psetproblem}{0}}


\def\beginedxtext#1{\refstepcounter{edXtext}\begin{edXtext}{#1}[url_name="\edxbaseoutputname-tab\theedXvertical-text\theedXtext"]}
\def\endedxtext{\end{edXtext}}
\def\beginedxproblem#1#2{\refstepcounter{edXproblem}\begin{edXproblem}{#1}{url_name="\edxbaseoutputname-tab\theedXvertical-problem\theedXproblem" #2}}
\def\endedxproblem{\end{edXproblem}}

%generate pset problem names
\def\beginedxpset#1#2{\refstepcounter{edXproblem}\refstepcounter{psetproblem}\begin{edXproblem}{#1 (\thepsetproblem)}{url_name="\edxbaseoutputname-tab\theedXvertical-problem\theedXproblem" #2}}
\def\endedxpset{\end{edXproblem}}

\def\beginedxvertical#1{\begin{edXvertical}{#1}[url_name="\edxbaseoutputname-vertical\theedXvertical"]}
\def\endedxvertical{\end{edXvertical} \setcounter{edXtext}{0} \setcounter{edXproblem}{0} \setcounter{edXvideo}{0}}


%New-allow for source command
\providecommand{\doedxvideo}[3][]{\refstepcounter{edXvideo}\edXvideo{#2}{#3}[url_name="\edxbaseoutputname-tab\theedXvertical-video\theedXvideo" #1]}

%New wrapper.
\newenvironment{shh}{}{}
\def\beginedxshowhide#1{\begin{shh}\begin{edXshowhide}{#1}}  
\def\endedxshowhide{\end{edXshowhide}\end{shh}}


% EVH added
% \def\edXmathlet#1{\edXxml{<iframe src="https://s3.amazonaws.com/1801-static-assets/build/#1.html" width="820 px" height="630 px" style="border:0px"/>}}

%JEF added
% \providecommand{\includesvg}[2][400]{\edXxml{<img src="images/#2.svg" width="#1 px" style="margin: 10px 25px 25px 25px; border:0px"/>}}
\providecommand{\includesvg}[2][400]{\edXxml{<img src="/static/images/#2.svg" width="#1 px" style="margin: 10px 25px 25px 25px; border:0px"/>}}


%-------------------------------------------------%

\def\myheader#1{\noindent \textbf{#1}\par}

\def\ds{\displaystyle}
\def\Re{\mathrm{Re\,}}
\def\Im{\mathrm{Im\,}}

\newcommand{\R}{\mathbb R}
\newcommand{\Z}{\mathbb Z}
\newcommand{\C}{\mathbb C}
\newcommand{\Q}{\mathbb Q}
\newcommand{\inv}{^{-1}}
\newcommand{\eps}{\epsilon}
\newcommand{\veca}{\mathbf{a}}
\newcommand{\vecb}{\mathbf{b}}
\newcommand{\vecc}{\mathbf{c}}
\newcommand{\vecw}{\mathbf{w}}
\newcommand{\vecv}{\mathbf{v}}
\newcommand{\vecu}{\mathbf{u}}
\newcommand{\vecd}{\mathbf{d}}
\newcommand{\vece}{\mathbf{e}}
\newcommand{\vecx}{\mathbf{x}}
\newcommand{\veco}{\mathbf{0}}
\newcommand{\vecy}{\mathbf{y}}
\newcommand{\F}{{\bf {F}}}

%commands for Mattuck/Jerison Problems
% \def\qw{\qquad}
% \def\q{\quad}
% \def\f{\frac}
% \def\disp{\displaystyle}
% \def\To{\implies}
% \def\e{\epsilon}
% \def\t{\theta}
% \def\D{\Delta}

%These two are needed for content from David Jerison's notes
% \def\Implies{\ \implies \ }
% \def\Iff{\ \iff \ }

\def\imgdir{images}
%_________________________________



%\def\defaultproblemattributes{attempts="5" showanswer="attempted" rerandomization="per_student"}


% \def\qw{\qquad}
% \def\q{\quad}
% \def\fig{\includegraphics}

% \def\inv{^{-1}}

% \def\dpaA{showanswer="finished" attempts="1" rerandomize="per_student"}
% \def\dpaB{showanswer="finished" attempts="3" rerandomize="per_student"}
% \def\dpaC{showanswer="finished" attempts="5" rerandomize="per_student"}
% \def\dpaD{showanswer="finished" attempts="7" rerandomize="per_student"}
% \def\dpa#1{showanswer="finished" attempts="#1" rerandomize="per_student"}
\def\dpa#1{showanswer="finished" attempts="#1" rerandomize="per_student"}
% \def\dpwa[1]{showanswer="finished" weight="#1" attempts="1" rerandomize="per_student"}
% \def\dpwb[1]{showanswer="finished" weight="#1" attempts="2" rerandomize="per_student"}
% \def\dpwc[1]{showanswer="finished" weight="#1" attempts="3" rerandomize="per_student"}
% \def\dpwd[1]{showanswer="finished" weight="#1" attempts="4"  rerandomize="per_student"}
% \def\dpwe[1]{showanswer="finished" weight="#1" attempts="5" rerandomize="per_student"}
\def\dpaZ{attempts="100" rerandomize="onreset" showanswer="attempted"}
\def\dpadnd{showanswer="attempted" attempts="3" rerandomize="onreset"}
\def\resetdpa#1{showanswer="attempted" attempts="#1" rerandomize="onreset"}


%color for 18.01x
%added by Jen
\definecolor{blue}{cmyk}{1,1,0,0}
\definecolor{orange}{cmyk}{0,0.5,1,0}

\definecolor{bordeaux}{cmyk}{0,.84,.71,.40}
\newcommand{\keya}{\color{bordeaux}}

\definecolor{royalblue}{cmyk}{.72,.54, 0, .45}
% \newcommand{\keyb}{\color{royalblue}}
\newcommand{\keyb}{\color{bordeaux}}


\protected\def\blue#1{%
  \ifmmode
  	{\color{blue}{#1}}
  \else
  	\textbf{{\color{blue}{#1}}}
   \fi
}
\protected\def\red#1{%
  \ifmmode
  	{\color{orange}{#1}}
  \else
  	\textbf{{\color{orange}{#1}}}
   \fi
}



 %custom macros

\def\defaultproblemattributes{attempts="1" showanswer="attempted" rerandomize="per_student"}

% edXcourse: {course_number}{course display_name}[optional arguments like semester]
\begin{edXcourse}{Test}{Test LinAlg}[semester="2018_Summer" info_sidebar_name="Other Documents" start="2018-01-11T12:00" end="2018-12-18T18:00" course_image="rice-logo.jpg" display_coursenumber="TestLinAlg" course_organization="TestRice" graceperiod="1800 seconds" invitation_only="true" allow_anonymous="false" mobile_available="true"  org="Rice"]
 
% \begin{edXchapter}{Getting started}[url_name="cyca" start="2015-08-23T18:00"]
%  

% \def\edxbaseoutputname{edxtutorial}

% \input{overview/edxTutorial.tex}
 

% \endedxsequential
 
% \end{edXchapter} 





\begin{edXchapter}{Vectors and Matrices}[url_name="block1" start="2018-01-11T16:00"]

\def\edxbaseoutputname{b1lineareqns}



\beginedxsequential{Systems of Linear Equations}{due="2018-12-18T14:15" graded="true" format="Exercises"}







\beginedxvertical{Page One}

\beginedxtext{Preliminaries}


At the end of this sequence, and after some practice, you should be able to:

\begin{itemize}
\item Understand the definition of linear transformation.  
\item Be able to determine whether a function between vector spaces is a linear transformations. 
\item Recognize basic properties of linear transformations.
\end{itemize}

For time budgeting purposes, this sequence has 4 videos totaling 14 minutes, 
plus some questions.  

% Remember, when you're doing the online learning sequences, you may seek help if you 
% do not understand a video, but you should think about all of the questions 
% entirely individually.  You have pledged to do so under the Honor Code!  


\endedxtext

\endedxvertical

\beginedxvertical{Introduction}



\doedxvideo{Functions}{nMrAAozL4F0}


\beginedxproblem{Matrix Multiplication Domain and Codomain}{\dpa1}

Define the function $T$ by $T(x) = Ax$, where $A = \left[ \begin{array}{ccc}
6 & 5 & 3 \\
2 & 4 &  23 
\end{array} \right].$

What is the domain of $T$?

\edXabox{type="multichoice" expect="$\R^3$" options="$\R$","$\R^2$","$\R^3$"}

What is the codomain of $T$?

\edXabox{type="multichoice" expect="$\R^2$" options="$\R$","$\R^2$","$\R^3$"}

\edXsolution{ 
$A$ has 3 columns. Therefore, for the product $Ax$ to be defined, $x$ must have 3 rows, and will be a vector in $\R^3$. \\
Since $A$ has two rows, the product $Ax$ will be a vector in $\R^2$.
}

\endedxproblem



\endedxvertical





\beginedxvertical{Defining Linear Transformations}

\doedxvideo{Linear Transformations}{ASQadpJlL94}
%example: rotations, non-example: something quadratic


\endedxvertical









\beginedxvertical{Linear Transformation Definition}

\beginedxtext{Definition of Linear Transformation}

{\keya{\bf{Definition.}}} 
Let $V,W$ be vector spaces over a field $F$.    A function $T: V\rightarrow W$ is a {\keyb{\bf linear transformation}}
if both of the following properties hold:

\begin{itemize}
\item For any vectors $v, w$ in the domain, $T(v + w) = T(v) + T(w).$ 
\item For any vector $v$ in the domain, and any scalar $a\in F$, $T(av) = aT(v)$.    
\end{itemize}

Informally, if $T$ satisfies the first condition we say that ``$T$ respects vector addition", and 
if $T$ satisfies the second condition we say that ``$T$ respects scalar multiplication."  

\endedxtext

\endedxvertical


\beginedxvertical{Some Examples and Non-examples}



\beginedxproblem{Linear Transformation? 1}{\dpa1}

Define $T: \R^3 \rightarrow \R^3$ by $T(v) = 2v$.  

Does $T$ respect vector addition?  That is, is $T(v + w) = T(v) + T(w)$ for all 
vectors $v,w$?
Try a couple of specific numerical examples if you're not sure.  

\edXabox{expect="Yes" options="Yes","No"}

Does $T$ respect scalar multiplication?  
That is, is $T(av) = aT(v)$ for all vectors $v$ and scalars $a$?
Try a couple of specific numerical examples if you're not sure.  


\edXabox{expect="Yes" options="Yes","No"}

Is $T$ a linear transformation?

\edXabox{expect="Yes" options="Yes","No"}

\edXsolution{ 

It appears that the transformation is linear. We can prove it with algebra: \\

We have $T(v + w) = 2(v+w) = 2v + 2w = T(v) + T(w)$.  Therefore $T$ respects vector addition.  

Similarly $ T(av) = 2(av) = a(2v) =  aT(v),$ so $T$ respects scalar multiplication.  Thus, 
$T$ is a linear transformation.

}

\endedxproblem


\beginedxproblem{Linear Transformation? 2}{\dpa1}

Define $T: \R^2 \rightarrow \R^2$ by $T(v) = \veco$.  

Does $T$ respect vector addition?  That is, is $T(v + w) = T(v) + T(w)$ for all 
vectors $v, w$?
Try a couple of specific numerical examples if you're not sure.  

\edXabox{expect="Yes" options="Yes","No"}

Does $T$ respect scalar multiplication?  
That is, is $T(av) = aT(v)$ for all vectors $v$ and scalars $a$?
Try a couple of specific numerical examples if you're not sure.  


\edXabox{expect="Yes" options="Yes","No"}

Is $T$ a linear transformation?

\edXabox{expect="Yes" options="Yes","No"}

\edXsolution{ 

$T(v+w) = \veco = \veco + \veco = T(v) + T(w)$, so $T$ respects addition.  
Also, 
$T(av) =  \veco = a\veco = aT(v)$, so $T$ respects scalar multiplication.  

Hence $T$ is a linear transformation.  

}

\endedxproblem

\beginedxproblem{Linear Transformation? 3}{\dpa1}

Define $T: \R^2 \rightarrow \R^2$ by $T(v) = \left[\begin{array}{c}
4 \\
0 
\end{array} \right]$.  

Does $T$ respect vector addition?  That is, is $T(v + w) = T(v) + T(w)$ for all 
vectors $v,w$?
Try a couple of specific numerical examples if you're not sure.  

\edXabox{expect="No" options="Yes","No"}

Does $T$ respect scalar multiplication?  
That is, is $T(av) = aT(v)$ for all vectors $v$ and scalars $a$?
Try a couple of specific numerical examples if you're not sure.  


\edXabox{expect="No" options="Yes","No"}

Is $T$ a linear transformation?

\edXabox{expect="No" options="Yes","No"}

\edXsolution{ 
Here, trying specific vectors provides a counterexample, which is sufficient to show that $T$ is not a linear transformation.\\
Let $v_1 = \left[ \begin{array}{c}
1\\2\\5 
\end{array} \right]$, 
$v_2 = \left[ \begin{array}{c}
-4\\0\\9 
\end{array} \right]$\\
$T(v_1) + T(v_2) = \left[\begin{array}{c} 4 \\ 0 \end{array} \right] 
+  \left[\begin{array}{c} 4 \\ 0 \end{array} \right]  
=  \left[\begin{array}{c} 8 \\ 0 \end{array} \right]$\\
Meanwhile, $T(v_1 + v_2) = \left[\begin{array}{c} 4 \\ 0 \end{array} \right]$.
$T$ does not respect vector addition and is not linear.\\

We also see that 
$8T(v_1) = 8 \left[\begin{array}{c} 4 \\ 0 \end{array} \right] 
= \left[\begin{array}{c} 32 \\ 0 \end{array} \right]$\\
Meanwhile, $T(8v_1) = \left[\begin{array}{c} 4 \\ 0 \end{array} \right]$.
$T$ does not respect scalar multiplication; this on its own would also be enough to know that $T$ is not linear.
}

\endedxproblem


\endedxvertical


\beginedxvertical{More Linear Transformations}

\doedxvideo{A Linear Transformation outside of R^n}{Pyv9GXNYJ44}



\beginedxproblem{Linear Transformation? 4}{\dpa1}

Define $T: \mathbb{P} \rightarrow \R$ by $T(f) = f(1)$.  

To get a sense of how $T$ works, let's do a quick example.  Suppose $f$ is the element of 
$\mathbb{P}$ given by $f(t) = 2t^2 -5$.  
What is $T(f)$?  

\edXabox{type="numerical" expect="-3"}


Does $T$ respect vector addition?  That is, is $T(f+g) = T(f) + T(g)$ for all 
vectors $f,g \in \mathbb{P}$?
Try a couple of specific numerical examples if you're not sure.  

\edXabox{expect="Yes" options="Yes","No"}

Does $T$ respect scalar multiplication?  
That is, is $T(af) = aT(f)$ for all vectors $f \in \mathbb{P}$ and scalars $a$?
Try a couple of specific numerical examples if you're not sure.  

\edXabox{expect="Yes" options="Yes","No"}

Is $T$ a linear transformation?

\edXabox{expect="Yes" options="Yes","No"}

\edXsolution{ 
$T$ respects vector addition because:\\
$f(1) + g(1) = (f+g)(1)$\\
i.e. we can evaluate polynomials at 1 separately and add the results, or we can add the polynomials and evaluate the result at 1, and we'll get the same result.

% This must be true in general for all polynomials $f$ and $g$. We can test this with some specific examples, but we give a (somewhat sketchy) proof here.\\ Suppose $f$ and $g$ contain no like terms between them. Then the right side of the equation, the evaluation of $f+g$ at 1, would be calculated by evaluating the terms of $f$ and $g$ separately and then adding the results, which is precisely the left hand side of the equation. Otherwise, $f$ and $g$ contain like terms, and for any two terms $ax^n$ and $bx^n$ belonging to $f$ and $g$ respectively, $a(1^n) + b(1^n) = a + b = (a+b)(1^n)$.\\
$T$ respects scalar multiplication because:\\
$cT(f) = cf(1) = (cf)(1) = T(cf)$\\
i.e. we can evaluate a polynomial at 1 and multiply the result by a scalar, or multiply the polynomial by the scalar first and then evaluate the result at 1, and we'll get the same result. 

% Again, it is simple to try some examples to observe this, but we can also prove it.\\
% Suppose $f(x) = a_nx^n + a_{n-1}x_{n-1} + ... a_1x + a_0$. Then $cf(1) = c(a_n + ... + a_0) = ca_n + ca_{n-1} + ... + ca_0 = (cf)(1)$.

}

\endedxproblem



\endedxvertical


\beginedxvertical{Linear Transformations Properties}

\doedxvideo{Linear Transformation Properties}{oe4fgdPAsgA}




\beginedxproblem{Linear Transformation Practice}{\dpa1}

Suppose $T: \mathbb{P} \rightarrow \R^2$ is a linear transformation.  Let $f,g\in \mathbb{P}$ be
the polynomials given by $f(t) = t^2$ and $g(t) = t$.  

Suppose that $T(f) = \left[\begin{array}{c} 1 \\ 2  \end{array} \right]$
and $T(g) = \left[\begin{array}{c} 2 \\ -1  \end{array} \right]$



If $h \in \mathbb{P}$ is the polynomial given by $h(t) = -2t^2$, what must $T(h)$ be?

To enter a vector such as $\left[\begin{array}{c} 1 \\ 2  \\ 3 \end{array} \right]$, you can either:
\begin{itemize}
\item
enter it as 
you would a $3\times 1$ matrix: [[1],[2],[3]]  
\item
or, you can enter it as <1,2,3>
\end{itemize}

Use decimals only.  

\begin{edXscript}
def VectorEntry(expect, ans):
	import ast
	import numpy as np 
  	atol = 0.01
	list_expect = ast.literal_eval(expect)
	vec_expect = np.matrix(list_expect)
  	ret = {"ok":False}
	try:
  		# input format [[1],[2],[3]]
		list_ans = ast.literal_eval(ans)
		vec_ans = np.matrix(list_ans)
  		if vec_ans.shape != vec_expect.shape:
  			ret['msg'] = 'Wrong shape of vector!'
  		elif np.allclose(vec_ans, vec_expect,atol,1e-08):
  			ret['ok'] = True
  		else:
  		# More error message. Will improve this part
  			ret['msg'] = 'something is wrong'   			
	#except SyntaxError:
		#ret['msg'] = 'Wrong input format'
	except SyntaxError:
  		# input format &lt;1,2,3&gt;
		list_ans = ans.replace('&lt;', '[').replace('&gt;', ']')
		list_ans = ast.literal_eval(list_ans)
		vec_ans = np.transpose(np.matrix(list_ans))
  		if vec_ans.shape != vec_expect.shape:
  			ret['msg'] = 'Wrong shape of vector!'
  		elif np.allclose(vec_ans, vec_expect,0.01,1e-08):
  			ret['ok'] = True
  		else:
    		# More error message. Will improve this part
  			ret['msg'] = 'something is wrong' 
  	except:
  		ret['msg'] = 'Wrong input format'
  	return ret 
\end{edXscript}



\edXabox{type="custom" cfn="VectorEntry" expect="[[-2],[-4]]"}


If $j \in \mathbb{P}$ is the polynomial given by $h(t) = 3t^2-t$, what must $T(j)$ be?



\edXabox{type="custom" cfn="VectorEntry" expect="[[1],[7]]"}


\edXsolution{ 
We are assured that $T$ is a linear transformation, so we have: 
\[T(h) = T(-2t^2)  = -2T(t^2) = -2T(f) = -2\left[\begin{array}{c} 1 \\ 2  \end{array} \right] = \left[\begin{array}{c} -2 \\ -4  \end{array} \right]. \]

Also, 
\[T(j) = T(3t^2-t) = T(3t^2) + T(-t) = 3T(t^2) + -T(t) = 3T(f) - T(g) = 3\left[\begin{array}{c} 1 \\ 2  \end{array} \right] - \left[\begin{array}{c} 2 \\ -1  \end{array} \right] = \left[\begin{array}{c} 1 \\ 7  \end{array} \right]. \]
}


\endedxproblem



\endedxvertical


% \beginedxvertical{Linear Transformations Properties}

% \doedxvideo{Matrix Multiplication as a Linear Transformation}{jMRJ-efZOJs}



% \endedxvertical







% 

\beginedxvertical{Wait for it}

\beginedxtext{Coming Soon}

Coming soon!

\endedxtext

\endedxvertical




\endedxsequential



\end{edXchapter}




\end{edXcourse}
\end{document}
