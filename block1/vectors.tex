

\beginedxvertical{Page One}

\beginedxtext{Preliminaries}





At the end of this sequence, and after some practice, you should be able to:

\begin{itemize}
\item Compute elementary vector operations in $\R^n$.  
\item Describe the geometry of vector operations.
\end{itemize}


For time budgeting purposes, this sequence has X videos totaling X minutes, 
plus some questions.  




\endedxtext

\endedxvertical


\beginedxvertical{The Template Vector Space}

\doedxvideo{R^n}{3QM3Lg15Hi8}




\beginedxproblem{Vector entry}{\dpa5}


Let $v = \left[\begin{array}{c} 2 \\ 3  \\ 4 \end{array} \right]$ and let 
$w = \left[\begin{array}{c} 1 \\ -1  \\ 0 \end{array} \right].$





What is $v+w$?  To enter a vector such as $\left[\begin{array}{c} 1 \\ 2  \\ 3 \end{array} \right]$, you can either:
\begin{itemize}
\item
enter it as 
you would a $3\times 1$ matrix: [[1],[2],[3]]  
\item
or, you can enter it as <1,2,3>
\end{itemize}

Use decimals only.  

\begin{edXscript}
def VectorEntry(expect, ans):
	import ast
	import numpy as np 
  	atol = 0.01
	list_expect = ast.literal_eval(expect)
	vec_expect = np.matrix(list_expect)
  	ret = {"ok":False}
	try:
  		# input format [[1],[2],[3]]
		list_ans = ast.literal_eval(ans)
		vec_ans = np.matrix(list_ans)
  		if vec_ans.shape != vec_expect.shape:
  			ret['msg'] = 'Wrong shape of vector!'
  		elif np.allclose(vec_ans, vec_expect,atol,1e-08):
  			ret['ok'] = True
  		else:
  		# More error message. Will improve this part
  			ret['msg'] = 'something is wrong'   			
	#except SyntaxError:
		#ret['msg'] = 'Wrong input format'
	except SyntaxError:
  		# input format &lt;1,2,3&gt;
		list_ans = ans.replace('&lt;', '[').replace('&gt;', ']')
		list_ans = ast.literal_eval(list_ans)
		vec_ans = np.transpose(np.matrix(list_ans))
  		if vec_ans.shape != vec_expect.shape:
  			ret['msg'] = 'Wrong shape of vector!'
  		elif np.allclose(vec_ans, vec_expect,0.01,1e-08):
  			ret['ok'] = True
  		else:
    		# More error message. Will improve this part
  			ret['msg'] = 'something is wrong' 
  	except:
  		ret['msg'] = 'Wrong input format'
  	return ret 
\end{edXscript}



\edXabox{type="custom" cfn="VectorEntry" expect="[[3],[2],[4]]"}

What is $-3v$?  

\edXabox{type="custom" cfn="VectorEntry" expect="[[-6],[-9],[-12]]"}


What is $2w-3v$?  



\edXabox{type="custom" cfn="VectorEntry" expect="[[-4],[-11],[-12]]"}


\edXsolution{ 

$v+w = \left[\begin{array}{c} 2+1 \\ 3+(-1)  \\ 4+0 \end{array} \right]=\left[\begin{array}{c} 3 \\ 2  \\ 4 \end{array} \right]$


$-3v = \left[\begin{array}{c} (-3)2 \\ (-3)3  \\ (-3)4 \end{array} \right]=\left[\begin{array}{c} -6 \\ -9  \\ -12 \end{array} \right]$


$2w-3v=\left[\begin{array}{c} 2\cdot 1 \\ 2\cdot (-1)  \\ 2\cdot 0 \end{array} \right]+\left[\begin{array}{c} (-3)2 \\ (-3)3  \\ (-3)4 \end{array} \right]=\left[\begin{array}{c} 2 \\ -2  \\ 0 \end{array} \right]+
\left[\begin{array}{c} -6 \\ -9  \\ -12 \end{array} \right]=\left[\begin{array}{c} -4 \\ -11  \\ -12 \end{array} \right]$

}


\endedxproblem



\beginedxproblem{Where does it live?}{\dpa2}

The vector $w = \left[\begin{array}{c} 1 \\ -1  \\ 0 \\ 0 \end{array} \right]$  is 
an element of which of the following vector spaces?  Click all that apply.  


\edXabox{type="oldmultichoice" expect="$\R^4$" options="$\R^2$","$\R^3$","$\R^4$","$\R^5$"}

\edXsolution{The vector $w$ has 4 real entries.  Thus, it is an element of $\R^4$.

It is not an element of (for instance) $\R^2$.  The vector $\left[\begin{array}{c} 1 \\ -1   \end{array} \right]$
is an element of $\R^2$, but this is not the same as $w$.  }

\endedxproblem






\endedxvertical


\beginedxvertical{R^n and the Zero Vector}

\beginedxtext{R^n}

{\keya{\bf{Definition.}}} 
For a non-negative integer $n$, $\R^n$ is defined to be the set 
\[
\left\{\left[\begin{array}{c} a_1 \\ a_2 \\ \vdots \\ a_n
\end{array} \right] \ : \ a_1, a_2, \ldots a_n \in \R \right\}. \]  
With addition and scalar multiplication as defined previously, 
$\R^n$ is an example of a vector
space over the field $\R$.  


\endedxtext



\beginedxtext{Notes on Notation}

The vector $\left[\begin{array}{c} 0 \\ 0 \\ \vdots \\ 0
\end{array} \right] \in \R^n$ is called the {\keyb{\bf zero vector}}.  (In fact,
every vector space, not just $\R^n$, has a zero vector.)  

To distinguish it from the scalar 0, in print we will denote the zero vector
by $\veco$.  When handwritten, we will usually write the zero vector as a 
zero with an arrow over it, like $\vec{0}$.  Other letters that represent vectors will
not have the arrow notation, however; we will typically use letters like $u,v,w,x$ for 
vectors, while letters like $a,b,c$ will represent scalars.  

  
Be careful!  The zero vector in the space $\R^3$ is 
$\veco  = \left[\begin{array}{c} 0 \\ 0 \\  0
\end{array} \right],$ whereas the zero vector in the space 
$\R^4$ is 
$\veco  = \left[\begin{array}{c} 0 \\ 0 \\ 0 \\  0
\end{array} \right].$  Even though we use the same symbol, these are different objects!  
One has to judge from context which space any given $\veco$ lies in.  
 
\endedxtext


\endedxvertical


\beginedxvertical{Points and Arrows}

\doedxvideo{Geometry of R^n}{4P4qoU2s1Rs}



\beginedxproblem{Picture Problem}{\dpa2}

Vectors $v$ and $w$ are given.  

\begin{center}
\includesvg[450]{c1s3vectoraddsubtract}
\end{center}

Which vector is equal to $v+w$?    

\edXabox{expect="g" options="a","b","c","d","e","f","g","h"}

Which vector is equal to $v-w$?    

\edXabox{expect="d" options="a","b","c","d","e","f","g","h"}
Which vector is equal to $w-v$?    

\edXabox{expect="a" options="a","b","c","d","e","f","g","h"}

Which vector is equal to $-v$?    

\edXabox{expect="c" options="a","b","c","d","e","f","g","h"}

\edXsolution{If we align the tip of $w$ and the tail of $v$, we obtain $g$.  Now, $-w$ swaps the tip and tail of $w$, so to find $v-w$, we fix $v$, align the tip of $w$ with the tip of $v$ and draw a vector from the tail of $v$ to the tail of $w$.  To find $w-v$, we repeat the above with the roles of $w$ and $v$ swapped.  Or, we can note that $w-v=-(v-w)$ and $-d=a$.  Finally, $-v$ is the vector with the same magnitude as $v$, but opposite direction to $v$.}

\endedxproblem


\endedxvertical


\beginedxvertical{Beyond R^n}

\doedxvideo{Other Vector Spaces}{P0vitfigkm4}

\endedxvertical


\beginedxvertical{Vector Spaces}

\beginedxtext{Vector Spaces}

A {\keyb{\bf vector space}} over a field $F$ is, essentially, a set $V$ in which one can 
do vector addition and scalar multiplication.  For a more formal definition, see below. 

\begin{edXshowhide}{Formal Definition of a Vector Space}


A vector space $V$ over a field $F$ is a set, together with an addition operation on $V$
and a scalar multiplication of elements of $F$ on elements of $V$, that satisfies
the following conditions:

\begin{itemize}
\item Addition in $V$ is associative and commutative; that is, for all $u,v,w \in V$, we have
$u+(v+w) = (u+v)+w$ and $u+v  = v+u$.  
\item There is an element $\veco \in V$ such that $v+\veco = v$ for all $v \in V$.
\item For every $v\in V$, there is an element $-v \in V$ such that $-v + v = \veco$.  
\item $1v = v$ for all $v \in V$.  
\item Scalar multiplication distributes over addition; that is, for all $a \in F$ and $u,v \in V$, , $a(u+v) = au + av$.  
\item Scalar addition distributes as well; that is, for all $a,b \in F$ and $v \in V$, , $(a+b)v = av + bv$.  
\item For all $a,b \in F$, and $v\in V$, we have $a(bv) = (ab)v$.  
\end{itemize}

\end{edXshowhide}




\endedxtext


\beginedxproblem{Vector Spaces}{\dpa2}

True or False: The only vector spaces over the scalar field $\R$ are the spaces
$\R^n$ for $n> 0$.  

\edXabox{expect="False" options="True","False"}

\edXsolution{False.  We have seen other examples such as the set of polynomials. }

\endedxproblem


\endedxvertical

