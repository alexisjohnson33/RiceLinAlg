

\beginedxvertical{Page One}

\beginedxtext{Preliminaries}





At the end of this sequence, and after some practice, you should be able to:

\begin{itemize}
\item Understand the definition of linear independence.  
\item Be able to determine whether a list of vectors is linearly independent. 
\item Know conditions on whether a list of vectors in $\R^n$ can be linearly independent.
\end{itemize}


For time budgeting purposes, this sequence has X videos totaling X minutes, 
plus some questions.  




\endedxtext

\endedxvertical


\beginedxvertical{Linear Independence}


\beginedxproblem{Existence vs. Uniqueness}{\dpa2}

Given a matrix $A$, consider the equation $A\vecx = \veco$.   

Regarding such equations, we have discussed two main questions: the existence question
(does there exist at least one solution?) and the uniqueness question (is there at most one
solution?).  For this equation, one of these questions always has the same answer, no matter
what $A$ is.  Which question does not always have the same answer?  

\edXabox{expect="Uniqueness" options="Existence","Uniqueness"}

\edXsolution{  }

\endedxproblem

\doedxvideo{Linear Independence}{???}


\beginedxproblem{Definition}{\dpa1}


Which of these is the definition of linearly dependent?  

A list of vectors $\{v_1; v_2; \ldots v_n\}$ is linearly dependent if...

\edXabox{type="multichoice" expect="...there exist scalars
$a_1, a_2, \ldots a_n$, not all zero, such that $a_1v_1 + a_2v_2 + \ldots a_nv_n = \veco$." options="...there exist scalars
$a_1, a_2, \ldots a_n$, not all zero, such that $a_1v_1 + a_2v_2 + \ldots a_nv_n = \veco$.","...there exist scalars
$a_1, a_2, \ldots a_n$, all not zero, such that $a_1v_1 + a_2v_2 + \ldots a_nv_n = \veco$.","Both of these are the definition."}

\edXsolution{ 
}


\endedxproblem


\endedxvertical


\beginedxvertical{Linear Independence Problems}


\beginedxtext{Definition of Linearly Independent}

We say that a list of vectors $\{v_1; v_2; \ldots v_n\}$ in a vector space $V$ are 
{\keyb{\bf linearly independent}} if the only way to pick scalars $a_1, a_2, \ldots a_n$ such that
$a_1v_1 + a_2v_2 + \ldots + a_nv_n = \veco$ is if $a_1 = a_2 = \ldots = a_n = 0$.  



\endedxtext

\beginedxproblem{List 1}{\dpa1}


Consider the list of vectors $\{v_1; v_2; v_3\}$, where

\[v_1 = \left(\begin{array}{c} 1 \\ -1  \\ 0 \\ 0 \end{array} \right), 
v_2 = \left(\begin{array}{c} 2 \\ -1  \\ 3 \\ 5 \end{array} \right),  
v_3 = \left(\begin{array}{c} -3 \\ 3  \\ 0 \\ 0 \end{array} \right). \]

Can you find scalars $a_1, a_2, a_3$, not all zero, such that 
$a_1v_1 + a_2 v_2 + a_3 v_3 = \veco$?  If so, enter an example of such scalars below.
If not, enter `No'  

\edXabox{expect="Placeholder" options="Placeholder"}



Is the list $\{v_1; v_2; v_3\}$ linearly dependent or linearly independent?  

\edXabox{expect="Linearly dependent" options="Linearly dependent","Linearly independent"}


\edXsolution{ 
}


\endedxproblem



\beginedxproblem{List 2}{\dpa1}


Consider the list of vectors $\{v_1; v_2; v_3\}$, where

\[v_1 = \left(\begin{array}{c} 1 \\ -1  \\ 3 \\ 4 \\5 \end{array} \right), 
v_2 = \left(\begin{array}{c} 0 \\ 0  \\ 0 \\ 0 \\ 0 \end{array} \right),  
v_3 = \left(\begin{array}{c} -3 \\ 3  \\ 3 \\ 5 \\ 1 \end{array} \right). \]

Can you find scalars $a_1, a_2, a_3$, not all zero, such that 
$a_1v_1 + a_2 v_2 + a_3 v_3 = \veco$?  If so, enter an example of such scalars below.
If not, enter `No'.  

\edXabox{expect="Placeholder" options="Placeholder"}


Is the list $\{v_1; v_2; v_3\}$ linearly dependent or linearly independent?  

\edXabox{expect="Linearly dependent" options="Linearly dependent","Linearly independent"}

\edXsolution{ 
}


\endedxproblem


\beginedxproblem{List 3}{\dpa1}


Consider the list of vectors $\{v_1; v_2\}$, where $v_1$ and $v_2$ are
vectors in $\R^2$ as given in this picture:

[picture]

Can you find scalars $a_1, a_2$, not both zero, such that 
$a_1v_1 + a_2 v_2 = \veco$?  If so, enter an example of such scalars below.
If not, enter `No'.  

\edXabox{expect="Placeholder" options="Placeholder"}


Is the list $\{v_1; v_2\}$ linearly dependent or linearly independent?  

\edXabox{expect="Linearly independent" options="Linearly dependent","Linearly independent"}

\edXsolution{ 
}


\endedxproblem


\endedxvertical


\beginedxvertical{Determining Linear Independence}



\doedxvideo{Determining Linear Independence}{???}



\beginedxproblem{Independent? 1}{\dpa2}

Is the following list of vectors linearly independent?

\[v_1 = \left(\begin{array}{c} 1 \\ -1  \\ 3 \\ 4 \\5 \end{array} \right), 
v_2 = \left(\begin{array}{c} 2 \\ 1  \\ 3 \\ 1 \\ 3 \end{array} \right),  
v_3 = \left(\begin{array}{c} 3 \\ 3  \\ 3 \\ -2 \\ 1 \end{array} \right). \]


\edXabox{expect="Linearly dependent" options="Linearly dependent","Linearly independent"}

\endedxproblem


\beginedxproblem{Fill in the blank}{\dpa2}

What goes in the blank to make a correct statement?

If the matrix $A$ when row-reduced yields $\underline{\;\;\;\;\;\;\;\;\;\;}$, then
the columns of $A$ cannot be linearly independent.  

\edXabox{type="multichoice" expect="a free variable" options="a row of all zeros","a free variable","Either of the above"}

\endedxproblem


\beginedxproblem{Fill in the blank 2}{\dpa2}

What goes in the blank to make a correct statement?

If the matrix $A$ has $\underline{\;\;\;\;\;\;\;\;\;\;}$, then
there must be a free variable.  

\edXabox{type="multichoice" expect="more columns than rows" options="more rows than columns","more columns than rows"}

\endedxproblem


\beginedxproblem{Fill in the blank 3}{\dpa2}

What goes in the blank to make a correct statement?

A list of  $\underline{\;\;\;\;\;\;\;\;\;\;}$ cannot be linearly independent.

\edXabox{type="multichoice" expect="more columns than rows" options="3 vectors in $\R^5$","5 vectors in $\R^3$"}

\endedxproblem


\endedxvertical

\beginedxvertical{The Chart}



\doedxvideo{Completing the Chart}{XXX}



\beginedxproblem{Quick Picks}{\dpa2}

Without doing any row-reduction, you should be able to pick out three of the 
following lists of vectors that are linearly dependent.  Which three?  

\begin{enumerate}


\item
$\left( \begin{array}{c} 1 \\ 1 \\ 1 \end{array} \right) , 
\left( \begin{array}{c} 1 \\ 1 \\ 0 \end{array} \right) , 
\left( \begin{array}{c} 1 \\ 0 \\ 1 \end{array} \right) 
$


\item
$\left( \begin{array}{c} 1 \\ 2 \\ 3 \end{array} \right) , 
\left( \begin{array}{c} 0 \\ 0 \\ 0 \end{array} \right) , 
\left( \begin{array}{c} 2 \\ 5 \\ 7 \end{array} \right) $



\item
$\left( \begin{array}{c} 1 \\ 2 \\ 1 \\ 4\end{array} \right) , 
\left( \begin{array}{c} 2 \\ 13 \\ 3 \\ 7\end{array} \right) , 
\left( \begin{array}{c} 0 \\ 1 \\ 5 \\ -3\end{array} \right) , 
\left( \begin{array}{c} 15 \\ -2 \\ 6 \\ 2\end{array} \right) , 
\left( \begin{array}{c} 5 \\ 3 \\ 8 \\ 11\end{array} \right) $



\item
$\left( \begin{array}{c} 2 \\ 3 \\ 5 \\ 7 \\ 11 \end{array} \right) ,
\left( \begin{array}{c} 1 \\ 2 \\ 3 \\ 4 \\5 \end{array} \right) , 
\left( \begin{array}{c} 2 \\ 4 \\ 6 \\ 8 \\ 10 \end{array} \right) , 
\left( \begin{array}{c} 1 \\ 1 \\ 2 \\ 3 \\ 5 \end{array} \right) $

\item
$\left( \begin{array}{c} 0 \\ 1 \\ 0\\ 2 \\ 3 \\ 0 \end{array} \right) , 
\left( \begin{array}{c} 0 \\ 2 \\ 0\\ 7 \\ 1 \\ 0 \end{array} \right) , 
\left( \begin{array}{c} 0 \\ 12 \\ 0\\ 23 \\ 35 \\ 0 \end{array} \right) $


\item
$\left( \begin{array}{c} 1 \\ 1 \\ 1 \\ 1 \end{array} \right) , 
\left( \begin{array}{c} 1 \\ 1 \\ 2 \\ 1 \end{array} \right) , 
\left( \begin{array}{c} 1 \\ 1 \\ 1 \\ 2 \end{array} \right) 
$
\end{enumerate}




\edXabox{type="oldmultichoice" expect="A" options="A","B","C","D","E","F"}

\edXsolution{  }

\endedxproblem



\endedxvertical


\beginedxvertical{One Last Proposition}

\doedxvideo{Geometry and One Last Proposition}{XXX}


\endedxvertical

