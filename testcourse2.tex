\documentclass[12pt]{article}

\usepackage{edXpsl}	% edX
\usepackage{amsmath, amsthm, amsfonts, amssymb, color, mathrsfs, comment}
\usepackage{cancel}

%------------------------------------------
\parindent=0pt
\parskip=1ex

\begin{document}
%-----------------------------------------------%

%\documentclass[12pt]{article}
%
%\usepackage{edXpsl}	% edX
%\usepackage{amsmath, amsthm, amsfonts, amssymb, color, mathrsfs, comment}
%\usepackage{cancel}
%
%%------------------------------------------
%\parindent=0pt
%\parskip=1ex
%
%\begin{document}
%%-----------------------------------------------%

\newcounter{edXtext}
\newcounter{edXproblem}
\newcounter{edXvideo}
\newcounter{showhide}
\newcounter{psetproblem}
%\newcounter{edXvertical}


%If seems \ifplastex needs to go after the \begin{document}
% \newif\ifplastex 
% \plastexfalse
% \ifplastex
  %% Many of these could be better handled in python. When I get the chance to understand plastex better I'll do that. JMO
   % \def\USING{USING PLASTEX FIX MACROS}
   \def\cancel#1{#1}
%   \def\footnote#1{} %\par {\small{\color{red} FOOTNOTE:} #1}}
   \def\mylabel#1{\label{#1}\tag{\value{equation}}}

   \def\displaynametag{display_name}

   %\def\includesvg{\includegraphics}
   %\def\includegraphics{\edXincludegraphics}

   \newcounter{mynumbereditem}
   \newcounter{temp}
   \def\mysecnum{\arabic{section}}
   \def\mynumbereditemnum{\arabic{mynumbereditem}}
   \def\mynumbereditemlabel{\mysecnum.\mynumbereditemnum}

   \def\mysection#1{\section{#1}}

   \newenvironment{theorem}{\addtocounter{mynumbereditem}{1}%
\textbf{Theorem \mynumbereditemlabel.} }{\par\vspace{.5in}}
   \newenvironment{proposition}{\addtocounter{mynumbereditem}{1}%
\textbf{Proposition \mynumbereditemlabel.} }{\par\vspace{.5in}}
   \newenvironment{lemma}{\addtocounter{mynumbereditem}{1}%
\textbf{Lemma \mynumbereditemlabel.} }{\par\vspace{.5in}}
   \newenvironment{definition}{\addtocounter{mynumbereditem}{1}%
\textbf{Definition \mynumbereditemlabel.} }{\par\vspace{.5in}}
   \newenvironment{remarks}{\addtocounter{mynumbereditem}{1}%
\textbf{Remarks \mynumbereditemlabel.} }{\par\vspace{.5in}}
   \newenvironment{corollary}{\addtocounter{mynumbereditem}{1}%
\textbf{Corollary \mynumbereditemlabel.} }{\par\vspace{.5in}}
   \newenvironment{remark}{\addtocounter{mynumbereditem}{1}%
\textbf{Remark \mynumbereditemlabel.} }{\par\vspace{.5in}}
   \newenvironment{examples}{\addtocounter{mynumbereditem}{1}%
\textbf{Examples \mynumbereditemlabel.} }{\par\vspace{.5in}}
   \newenvironment{example}{\addtocounter{mynumbereditem}{1}%
\textbf{Example \mynumbereditemlabel.} }{\par\vspace{.5in}}
   \newenvironment{exercise}{\addtocounter{mynumbereditem}{1}%
\textbf{Exercise \mynumbereditemlabel.} }{\par\vspace{.5in}}

   \newenvironment{stheorem}{\textbf{Theorem. }}{\par\vspace{.5in}}
   \newenvironment{sproposition}{\textbf{Proposition. }}{\par\vspace{.5in}}
   \newenvironment{slemma}{\textbf{Lemma. }}{\par\vspace{.5in}}
   \newenvironment{sdefinition}{\textbf{Definition. }}{\par\vspace{.5in}}
   \newenvironment{sremarks}{\textbf{Remarks. }}{\par\vspace{.5in}}
   \newenvironment{scorollary}{\textbf{Corollary. }}{\par\vspace{.5in}}
   \newenvironment{sremark}{\textbf{Remark. }}{\par\vspace{.5in}}
   \newenvironment{sexamples}{\textbf{Examples. }}{\par\vspace{.5in}}
   \newenvironment{sexample}{\textbf{Example. }}{\par\vspace{.5in}}
   \newenvironment{sexercise}{\textbf{Exercise. }}{\par\vspace{.5in}}


% \def\dpa{weight="1" showanswer="attempted" attempts="3"}

\def\beginedxsequential#1#2{\begin{edXsection}{#1}[#2 url_name="\edxbaseoutputname-sequential"]}

\def\endedxsequential{\end{edXsection} \setcounter{psetproblem}{0}}


\def\beginedxtext#1{\refstepcounter{edXtext}\begin{edXtext}{#1}[url_name="\edxbaseoutputname-tab\theedXvertical-text\theedXtext"]}
\def\endedxtext{\end{edXtext}}
\def\beginedxproblem#1#2{\refstepcounter{edXproblem}\begin{edXproblem}{#1}{url_name="\edxbaseoutputname-tab\theedXvertical-problem\theedXproblem" #2}}
\def\endedxproblem{\end{edXproblem}}

%generate pset problem names
\def\beginedxpset#1#2{\refstepcounter{edXproblem}\refstepcounter{psetproblem}\begin{edXproblem}{#1 (\thepsetproblem)}{url_name="\edxbaseoutputname-tab\theedXvertical-problem\theedXproblem" #2}}
\def\endedxpset{\end{edXproblem}}

\def\beginedxvertical#1{\begin{edXvertical}{#1}[url_name="\edxbaseoutputname-vertical\theedXvertical"]}
\def\endedxvertical{\end{edXvertical} \setcounter{edXtext}{0} \setcounter{edXproblem}{0} \setcounter{edXvideo}{0}}


%New-allow for source command
\providecommand{\doedxvideo}[3][]{\refstepcounter{edXvideo}\edXvideo{#2}{#3}[url_name="\edxbaseoutputname-tab\theedXvertical-video\theedXvideo" #1]}

%New wrapper.
\newenvironment{shh}{}{}
\def\beginedxshowhide#1{\begin{shh}\begin{edXshowhide}{#1}}  
\def\endedxshowhide{\end{edXshowhide}\end{shh}}


% EVH added
% \def\edXmathlet#1{\edXxml{<iframe src="https://s3.amazonaws.com/1801-static-assets/build/#1.html" width="820 px" height="630 px" style="border:0px"/>}}

%JEF added
% \providecommand{\includesvg}[2][400]{\edXxml{<img src="images/#2.svg" width="#1 px" style="margin: 10px 25px 25px 25px; border:0px"/>}}
\providecommand{\includesvg}[2][400]{\edXxml{<img src="/static/images/#2.svg" width="#1 px" style="margin: 10px 25px 25px 25px; border:0px"/>}}


%-------------------------------------------------%

\def\myheader#1{\noindent \textbf{#1}\par}

\def\ds{\displaystyle}
\def\Re{\mathrm{Re\,}}
\def\Im{\mathrm{Im\,}}

\newcommand{\R}{\mathbb R}
\newcommand{\Z}{\mathbb Z}
\newcommand{\C}{\mathbb C}
\newcommand{\Q}{\mathbb Q}
\newcommand{\inv}{^{-1}}
\newcommand{\eps}{\epsilon}
\newcommand{\veca}{\mathbf{a}}
\newcommand{\vecb}{\mathbf{b}}
\newcommand{\vecc}{\mathbf{c}}
\newcommand{\vecw}{\mathbf{w}}
\newcommand{\vecv}{\mathbf{v}}
\newcommand{\vecu}{\mathbf{u}}
\newcommand{\vecd}{\mathbf{d}}
\newcommand{\vece}{\mathbf{e}}
\newcommand{\vecx}{\mathbf{x}}
\newcommand{\veco}{\mathbf{0}}
\newcommand{\vecy}{\mathbf{y}}
\newcommand{\F}{{\bf {F}}}

%commands for Mattuck/Jerison Problems
% \def\qw{\qquad}
% \def\q{\quad}
% \def\f{\frac}
% \def\disp{\displaystyle}
% \def\To{\implies}
% \def\e{\epsilon}
% \def\t{\theta}
% \def\D{\Delta}

%These two are needed for content from David Jerison's notes
% \def\Implies{\ \implies \ }
% \def\Iff{\ \iff \ }

\def\imgdir{images}
%_________________________________



%\def\defaultproblemattributes{attempts="5" showanswer="attempted" rerandomization="per_student"}


% \def\qw{\qquad}
% \def\q{\quad}
% \def\fig{\includegraphics}

% \def\inv{^{-1}}

% \def\dpaA{showanswer="finished" attempts="1" rerandomize="per_student"}
% \def\dpaB{showanswer="finished" attempts="3" rerandomize="per_student"}
% \def\dpaC{showanswer="finished" attempts="5" rerandomize="per_student"}
% \def\dpaD{showanswer="finished" attempts="7" rerandomize="per_student"}
% \def\dpa#1{showanswer="finished" attempts="#1" rerandomize="per_student"}
\def\dpa#1{showanswer="finished" attempts="#1" rerandomize="per_student"}
% \def\dpwa[1]{showanswer="finished" weight="#1" attempts="1" rerandomize="per_student"}
% \def\dpwb[1]{showanswer="finished" weight="#1" attempts="2" rerandomize="per_student"}
% \def\dpwc[1]{showanswer="finished" weight="#1" attempts="3" rerandomize="per_student"}
% \def\dpwd[1]{showanswer="finished" weight="#1" attempts="4"  rerandomize="per_student"}
% \def\dpwe[1]{showanswer="finished" weight="#1" attempts="5" rerandomize="per_student"}
\def\dpaZ{attempts="100" rerandomize="onreset" showanswer="attempted"}
\def\dpadnd{showanswer="attempted" attempts="3" rerandomize="onreset"}
\def\resetdpa#1{showanswer="attempted" attempts="#1" rerandomize="onreset"}


%color for 18.01x
%added by Jen
\definecolor{blue}{cmyk}{1,1,0,0}
\definecolor{orange}{cmyk}{0,0.5,1,0}

\definecolor{bordeaux}{cmyk}{0,.84,.71,.40}
\newcommand{\keya}{\color{bordeaux}}

\definecolor{royalblue}{cmyk}{.72,.54, 0, .45}
% \newcommand{\keyb}{\color{royalblue}}
\newcommand{\keyb}{\color{bordeaux}}


\protected\def\blue#1{%
  \ifmmode
  	{\color{blue}{#1}}
  \else
  	\textbf{{\color{blue}{#1}}}
   \fi
}
\protected\def\red#1{%
  \ifmmode
  	{\color{orange}{#1}}
  \else
  	\textbf{{\color{orange}{#1}}}
   \fi
}



 %custom macros

\def\defaultproblemattributes{attempts="1" showanswer="attempted" rerandomize="per_student"}

% edXcourse: {course_number}{course display_name}[optional arguments like semester]
\begin{edXcourse}{Test}{Test LinAlg}[semester="2018_Summer" info_sidebar_name="Other Documents" start="2018-01-11T12:00" end="2019-12-18T18:00" course_image="rice-logo.jpg" display_coursenumber="TestLinAlg" course_organization="TestRice" graceperiod="1800 seconds" invitation_only="true" allow_anonymous="false" mobile_available="true"  org="Rice"]
 
% \begin{edXchapter}{Getting started}[url_name="cyca" start="2015-08-23T18:00"]
%  

% \def\edxbaseoutputname{edxtutorial}

% \input{overview/edxTutorial.tex}
 

% \endedxsequential
 
% \end{edXchapter} 





\begin{edXchapter}{Vectors and Matrices}[url_name="block1" start="2018-01-11T16:00"]



\def\edxbaseoutputname{b2matrixops}



\beginedxsequential{LinInd}{due="2019-12-13T14:15" graded="true" format="Exercises"}






\usepackage{enumerate}

\beginedxvertical{Page One}

\beginedxtext{Preliminaries}





At the end of this sequence, and after some practice, you should be able to:

\begin{itemize}
\item Understand the definition of linear independence.  
\item Be able to determine whether a list of vectors is linearly independent. 
\item Know conditions on whether a list of vectors in $\R^n$ can be linearly independent.
\end{itemize}


For time budgeting purposes, this sequence has X videos totaling X minutes, 
plus some questions.  




\endedxtext

\endedxvertical


\beginedxvertical{Linear Independence}


\beginedxvertical{Linear Independence Problems}


\beginedxtext{Definitions of Linearly Independent and Dependent}

{\keya{\bf{Definition.}}} We say that a list of vectors $\{v_1; v_2; \ldots v_n\}$ in a vector space $V$ is 
{\keyb{\bf linearly independent}} if the only way to pick scalars $a_1, a_2, \ldots a_n$ such that
$a_1v_1 + a_2v_2 + \ldots + a_nv_n = \veco$ is if $a_1 = a_2 = \ldots = a_n = 0$.  

{\keya{\bf{Definition.}}} A list of vectors $\{v_1; v_2; \ldots v_n\}$ in a vector space $V$ is 
{\keyb{\bf linearly dependent}} if there are scalars $a_1, a_2, \ldots a_n$, not all zero, such that
$a_1v_1 + a_2v_2 + \ldots + a_nv_n = \veco$.  


\endedxtext

\beginedxproblem{List 1}{\dpa5}

\begin{edXscript}

def test_ld_1(expect,ans):
    import ast
    import numpy as np 
    if (ans=='No') or (ans=='NO') or (ans=='no'):
        return {'ok': False, 'msg': 'The answer is not no'}
    list_expect = ast.literal_eval('[[1,-1,0,0],[2,-1,3,5],[-3,3,0,0]]')
    list_vec = []
    for vec in list_expect:
        list_vec.append(np.array(vec))
    list_scalars = [float(s) for s in ans.split(',')]
    ret = {'ok':False}
    # check scalars are not all zero
    max_sca = max(list_scalars)
    min_sca = min(list_scalars)
    if (max_sca == 0) and (min_sca ==0):
        ret['msg'] = 'Your answer should contain at least one non-zero scalar.'
        return ret
    # check there are enough scalars
    if (len(list_scalars) != len(list_vec)):
        ret['msg'] = 'Your answer contains too many/too few scalars.'
        return ret    
    # check if a0*v0 + a1*v1 + ... + a_n*vn = 0
    n = len(list_scalars)
    sum = 0
    for i in range(0,n):
        sum += list_scalars[i] * list_vec[i]
    print(sum)
    if np.allclose(sum, np.zeros(4)):
        ret['ok'] = True
    return ret  
\end{edXscript}


Consider the list of vectors $\{v_1; v_2; v_3\}$, where

\[v_1 = \left[\begin{array}{c} 1 \\ -1  \\ 0 \\ 0 \end{array} \right], 
v_2 = \left[\begin{array}{c} 2 \\ -1  \\ 3 \\ 5 \end{array} \right],  
v_3 = \left[\begin{array}{c} -3 \\ 3  \\ 0 \\ 0 \end{array} \right]. \]

Can you find scalars $a_1, a_2, a_3$, not all zero, such that 
$a_1v_1 + a_2 v_2 + a_3 v_3 = \veco$?  If so, enter an example of such scalars below.  (Use decimals only, separated by commas.  For instance, enter 3,2,5 if you want $a_1 = 3, a_2=2, a_3=5$.)
If not, enter `No'.



\edXabox{type="custom" cfn="test_ld_1" expect="3,0,1"}


Is the list $\{v_1; v_2; v_3\}$ linearly dependent or linearly independent?  

\edXabox{expect="Linearly dependent" options="Linearly dependent","Linearly independent"}


\edXsolution{ 
We notice that $-3v_1 = v_3$, so that $3  v_1 + 0 v_2 + 1  v_3 = \veco$. Therefore, $a_1 = 3$, $a_2 = 0$, and $a_3 = 1$ is a possible set of scalars that satisfies the equation. (Any scalar multiple of that triple is also a set that satisfies the equation.)

The list $\{v_1; v_2; v_3\}$ is linearly dependent because, as we've just seen, there exist sets of coefficients, not all zero, that yield $a_1v_1 + a_2 v_2 + a_3 v_3 = \veco$.
}


\endedxproblem



\beginedxproblem{List 2}{\dpa5}


\begin{edXscript}

def test_ld_12(expect,ans):
    import ast
    import numpy as np 
    if (ans=='No') or (ans=='NO') or (ans=='no'):
        return {'ok': False, 'msg': 'The answer is not no'}
    list_expect = ast.literal_eval('[[1,-1,3,4,5],[0,0,0,0,0],[-3,3,3,5,1]]')
    list_vec = []
    for vec in list_expect:
        list_vec.append(np.array(vec))
    list_scalars = [float(s) for s in ans.split(',')]
    ret = {'ok':False}
    # check scalars are not all zero
    max_sca = max(list_scalars)
    min_sca = min(list_scalars)
    if (max_sca == 0) and (min_sca ==0):
        ret['msg'] = 'Your answer should contain at least one non-zero scalar.'
        return ret
    # check there are enough scalars
    if (len(list_scalars) != len(list_vec)):
        ret['msg'] = 'Your answer contains too many/too few scalars.'
        return ret    
    # check if a0*v0 + a1*v1 + ... + a_n*vn = 0
    n = len(list_scalars)
    sum = 0
    for i in range(0,n):
        sum += list_scalars[i] * list_vec[i]
    print(sum)
    if np.allclose(sum, np.zeros(5)):
        ret['ok'] = True
    return ret  
\end{edXscript}

Consider the list of vectors $\{v_1; v_2; v_3\}$, where

\[v_1 = \left[\begin{array}{c} 1 \\ -1  \\ 3 \\ 4 \\5 \end{array} \right], 
v_2 = \left[\begin{array}{c} 0 \\ 0  \\ 0 \\ 0 \\ 0 \end{array} \right],  
v_3 = \left[\begin{array}{c} -3 \\ 3  \\ 3 \\ 5 \\ 1 \end{array} \right]. \]

Can you find scalars $a_1, a_2, a_3$, not all zero, such that 
$a_1v_1 + a_2 v_2 + a_3 v_3 = \veco$?  If so, enter an example of such scalars below.  (Use decimals only, separated by commas.  For instance, enter 3,2,5 if you want $a_1 = 3, a_2=2, a_3=5$.)
If not, enter `No'.  

\edXabox{type="custom" cfn="test_ld_12" expect="0,1,0"}


Is the list $\{v_1; v_2; v_3\}$ linearly dependent or linearly independent?  

\edXabox{expect="Linearly dependent" options="Linearly dependent","Linearly independent"}

\edXsolution{ The first answer is immediately "yes" because we have the zero vector in our set; we can choose any non-zero value for $a_2$, and with $a_1 = 0$ and $a_3 = 0$, the above equation will still be true. Because there exist sets of coefficients, not all zero, that make the above equation true, the list $\{v_1; v_2; v_3\}$ is linearly dependent.
}

\endedxproblem






\endedxvertical





% 

\beginedxvertical{Wait for it}

\beginedxtext{Coming Soon}

Coming soon!

\endedxtext

\endedxvertical




\endedxsequential


\end{edXchapter}




\end{edXcourse}
\end{document}
