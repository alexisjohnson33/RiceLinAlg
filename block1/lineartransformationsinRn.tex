

\beginedxvertical{Page One}

\beginedxtext{Preliminaries}


At the end of this sequence, and after some practice, you should be able to:

\begin{itemize}
\item Find the standard matrix for a linear transformation from $\R^n$ to $\R^m$.   
\item Know what geometric properties are preserved by linear transformations from 
$\R^n$ to $\R^m$.  
\end{itemize}

For time budgeting purposes, this sequence has 4 videos totaling 14 minutes, 
plus some questions.  

% Remember, when you're doing the online learning sequences, you may seek help if you 
% do not understand a video, but you should think about all of the questions 
% entirely individually.  You have pledged to do so under the Honor Code!  


\endedxtext

\endedxvertical

\beginedxvertical{Introduction}



\doedxvideo{Matrix Multiplication as Linear Transformation}{jMRJ-efZOJs}


\beginedxproblem{Where does it go?}{\dpa1}

Consider the following image.  

\begin{center}
\includesvg[400]{c1s8imageunderlintrans} 
\end{center}

When we apply the linear transformation $T: \R^2 \rightarrow \R^2$ given by $T(x) = \left[ \begin{array}{cc}
-1 & 0 \\ 0 & 1 \end{array} \right]$ to the image, which picture is the result?

\begin{center}
\includesvg[400]{c1s8imageunderlintransAB} 
\\
\includesvg[400]{c1s8imageunderlintransCD} 

\end{center}

\edXabox{expect="D" options="A","B","C","D"}


% When we apply the linear transformation $T: \R^2 \rightarrow \R^2$ given by $T(x) = \left[ \begin{array}{cc}
% -1 & 0 \\ 0 & -1 \end{array} \right]$ to the image, which picture is the result?

% \edXabox{expect="A" options="A","B","C","D"}


\edXsolution{ 

}

\endedxproblem


\beginedxproblem{Where does it go? 2}{\dpa1}

Consider the following image.  

\begin{center}
\includesvg[400]{c1s8imageunderlintrans} 
\end{center}

When we apply the linear transformation $T: \R^2 \rightarrow \R^2$ given by $T(x) = \left[ \begin{array}{cc}
0.5 & 0 \\ 0 & 2 \end{array} \right]$ to the image, which picture is the result?

\begin{center}
\includesvg[400]{c1s8imageunderlintransAB2} 

\end{center}

\edXabox{expect="B" options="A","B"}


% When we apply the linear transformation $T: \R^2 \rightarrow \R^2$ given by $T(x) = \left[ \begin{array}{cc}
% -1 & 0 \\ 0 & -1 \end{array} \right]$ to the image, which picture is the result?

% \edXabox{expect="A" options="A","B","C","D"}


\edXsolution{ 

}

\endedxproblem




\endedxvertical





\beginedxvertical{General Matrix Multiplication}

\doedxvideo{Multiplication by a General Matrix}{R1uBwxvKIXE}
%example: rotations, non-example: something quadratic




\beginedxproblem{The E Vectors}{\dpa1}

Suppose $T: \R^4\rightarrow \R^3$ is given by 
$T(x) = Ax$ where  
\[ A = \left[ \begin{array}{cccc} 1 & 3 & 5 & 7 \\ 2 & 0 & -1 & -1 \\ 1 & 1 & 3  & -2 \end{array} \right].\]

What is $T(e_3)$?  

To enter a vector such as $\left[\begin{array}{c} 1 \\ 2  \\ 3 \end{array} \right]$, you can either:
\begin{itemize}
\item
enter it as 
you would a $3\times 1$ matrix: [[1],[2],[3]]  
\item
or, you can enter it as <1,2,3>
\end{itemize}

Use decimals only.  

\begin{edXscript}
def VectorEntry(expect, ans):
	import ast
	import numpy as np 
  	atol = 0.01
	list_expect = ast.literal_eval(expect)
	vec_expect = np.matrix(list_expect)
  	ret = {"ok":False}
	try:
  		# input format [[1],[2],[3]]
		list_ans = ast.literal_eval(ans)
		vec_ans = np.matrix(list_ans)
  		if vec_ans.shape != vec_expect.shape:
  			ret['msg'] = 'Wrong shape of vector!'
  		elif np.allclose(vec_ans, vec_expect,atol,1e-08):
  			ret['ok'] = True
  		else:
  		# More error message. Will improve this part
  			ret['msg'] = 'something is wrong'   			
	#except SyntaxError:
		#ret['msg'] = 'Wrong input format'
	except SyntaxError:
  		# input format &lt;1,2,3&gt;
		list_ans = ans.replace('&lt;', '[').replace('&gt;', ']')
		list_ans = ast.literal_eval(list_ans)
		vec_ans = np.transpose(np.matrix(list_ans))
  		if vec_ans.shape != vec_expect.shape:
  			ret['msg'] = 'Wrong shape of vector!'
  		elif np.allclose(vec_ans, vec_expect,0.01,1e-08):
  			ret['ok'] = True
  		else:
    		# More error message. Will improve this part
  			ret['msg'] = 'something is wrong' 
  	except:
  		ret['msg'] = 'Wrong input format'
  	return ret 
\end{edXscript}


\edXabox{type="custom" cfn="VectorEntry" expect="[[5],[-1],[3]]"}


\edXsolution{ 

}


\endedxproblem

\beginedxproblem{Find a Matrix}{\dpa1}

Suppose the function $T: \R^3\rightarrow \R^2$ is given by 
\[T\left( \left[ \begin{array}{c} a_1 \\ a_2\\ a_3 \end{array} \right] \right) =\left[ \begin{array}{c} 2a_1-a_3 \\ -a_1 + a_2 \end{array} \right] \] 
for all vectors $\left[ \begin{array}{c} a_1 \\ a_2\\ a_3 \end{array} \right]  \in \R^3$.  

For what matrix $A$ is $T(x) = Ax$ for all $x \in \R^3$?  

To enter the matrix $\left[ \begin{array}{cc:c}
0&1&2 \\
1&2&3 \end{array} \right],$ type [[0,1,2],[1,2,3]]   

Use decimals only.  

\begin{edXscript}
def MatrixEntry(expect, ans):
  	import ast
	import numpy as np 
	ret= {'ok':False}
  	atol = 0.01
  	try:
		list_ans = ast.literal_eval(ans)
		list_expect = ast.literal_eval(expect)
  		matrix_ans = np.matrix(list_ans)
  		matrix_expect = np.matrix(list_expect) 
  		if matrix_ans.shape != matrix_expect.shape:
  			ret['msg'] = 'Wrong shape of matrix'
  		elif np.allclose(matrix_ans, matrix_expect,0.01,1e-08):
  			ret['ok'] = True
  		else:
  			ret['msg'] = 'Something is wrong'
	except SyntaxError:
		ret['msg'] = 'Wrong input format'
  	return ret
\end{edXscript}


\edXabox{type="custom" cfn="MatrixEntry" expect="[[2,0,-1],[-1,1,0]]"}

Is $T$ a linear transformation?  


\edXabox{expect="Yes" options="Yes","No"}

\edXsolution{ 

}


\endedxproblem


\endedxvertical





\beginedxvertical{Matrix for Rotation}


\beginedxproblem{Rotation}{\dpa2}

For the next two problems, we will be considering the linear transformation $T: \R^2 \rightarrow \R^2$
given by rotation counterclockwise by $45^\circ$.  


\begin{center}
\includesvg[400]{c1s8rotation} 

\end{center}

From the $45^\circ-45^\circ-9045^\circ$ triangles in the diagram, we can find that  
$T(e_1) = \left[ \begin{array}{c} \frac{\sqrt{2}}{2} \\ \frac{\sqrt{2}}{2}  \end{array} \right] \approx
\left[ \begin{array}{c} 0.707 \\ 0.707 \end{array} \right]$ and
$T(e_2) = \left[ \begin{array}{c} -\frac{\sqrt{2}}{2} \\ \frac{\sqrt{2}}{2}  \end{array} \right] \approx
\left[ \begin{array}{c} -0.707 \\ 0.707  \end{array} \right].$   

What is $T\left( \left[ \begin{array}{c} 4 \\ 0 \end{array} \right] \right)$?  (Remember to use decimals only, to within two decimal places.)  


\begin{edXscript}
def VectorEntry(expect, ans):
	import ast
	import numpy as np 
  	atol = 0.01
	list_expect = ast.literal_eval(expect)
	vec_expect = np.matrix(list_expect)
  	ret = {"ok":False}
	try:
  		# input format [[1],[2],[3]]
		list_ans = ast.literal_eval(ans)
		vec_ans = np.matrix(list_ans)
  		if vec_ans.shape != vec_expect.shape:
  			ret['msg'] = 'Wrong shape of vector!'
  		elif np.allclose(vec_ans, vec_expect,atol,1e-08):
  			ret['ok'] = True
  		else:
  		# More error message. Will improve this part
  			ret['msg'] = 'something is wrong'   			
	#except SyntaxError:
		#ret['msg'] = 'Wrong input format'
	except SyntaxError:
  		# input format &lt;1,2,3&gt;
		list_ans = ans.replace('&lt;', '[').replace('&gt;', ']')
		list_ans = ast.literal_eval(list_ans)
		vec_ans = np.transpose(np.matrix(list_ans))
  		if vec_ans.shape != vec_expect.shape:
  			ret['msg'] = 'Wrong shape of vector!'
  		elif np.allclose(vec_ans, vec_expect,0.01,1e-08):
  			ret['ok'] = True
  		else:
    		# More error message. Will improve this part
  			ret['msg'] = 'something is wrong' 
  	except:
  		ret['msg'] = 'Wrong input format'
  	return ret 
\end{edXscript}


\edXabox{type="custom" cfn="VectorEntry" expect="[[2.83],[2.83]]"}


\edXsolution{ 

}


\endedxproblem

\beginedxproblem{Rotation 2}{\dpa2}


What is $T\left( \left[ \begin{array}{c} 4 \\ 1 \end{array} \right] \right)$?  


\begin{edXscript}
def VectorEntry(expect, ans):
	import ast
	import numpy as np 
  	atol = 0.01
	list_expect = ast.literal_eval(expect)
	vec_expect = np.matrix(list_expect)
  	ret = {"ok":False}
	try:
  		# input format [[1],[2],[3]]
		list_ans = ast.literal_eval(ans)
		vec_ans = np.matrix(list_ans)
  		if vec_ans.shape != vec_expect.shape:
  			ret['msg'] = 'Wrong shape of vector!'
  		elif np.allclose(vec_ans, vec_expect,atol,1e-08):
  			ret['ok'] = True
  		else:
  		# More error message. Will improve this part
  			ret['msg'] = 'something is wrong'   			
	#except SyntaxError:
		#ret['msg'] = 'Wrong input format'
	except SyntaxError:
  		# input format &lt;1,2,3&gt;
		list_ans = ans.replace('&lt;', '[').replace('&gt;', ']')
		list_ans = ast.literal_eval(list_ans)
		vec_ans = np.transpose(np.matrix(list_ans))
  		if vec_ans.shape != vec_expect.shape:
  			ret['msg'] = 'Wrong shape of vector!'
  		elif np.allclose(vec_ans, vec_expect,0.01,1e-08):
  			ret['ok'] = True
  		else:
    		# More error message. Will improve this part
  			ret['msg'] = 'something is wrong' 
  	except:
  		ret['msg'] = 'Wrong input format'
  	return ret 
\end{edXscript}


\edXabox{type="custom" cfn="VectorEntry" expect="[[2.12],[3.54]]"}


\edXsolution{ 

}


\endedxproblem


\doedxvideo{Matrix for Rotation by 45 Degrees}{TN4eIiSNzcA}

\beginedxproblem{Rotation 3}{\dpa2}


If $T$ is still rotation counterclockwise by $45^\circ$, 
what is $T\left( \left[ \begin{array}{c} 3\sqrt{2} \\ -\sqrt{2} \end{array} \right] \right)$?  


\begin{edXscript}
def VectorEntry(expect, ans):
	import ast
	import numpy as np 
  	atol = 0.01
	list_expect = ast.literal_eval(expect)
	vec_expect = np.matrix(list_expect)
  	ret = {"ok":False}
	try:
  		# input format [[1],[2],[3]]
		list_ans = ast.literal_eval(ans)
		vec_ans = np.matrix(list_ans)
  		if vec_ans.shape != vec_expect.shape:
  			ret['msg'] = 'Wrong shape of vector!'
  		elif np.allclose(vec_ans, vec_expect,atol,1e-08):
  			ret['ok'] = True
  		else:
  		# More error message. Will improve this part
  			ret['msg'] = 'something is wrong'   			
	#except SyntaxError:
		#ret['msg'] = 'Wrong input format'
	except SyntaxError:
  		# input format &lt;1,2,3&gt;
		list_ans = ans.replace('&lt;', '[').replace('&gt;', ']')
		list_ans = ast.literal_eval(list_ans)
		vec_ans = np.transpose(np.matrix(list_ans))
  		if vec_ans.shape != vec_expect.shape:
  			ret['msg'] = 'Wrong shape of vector!'
  		elif np.allclose(vec_ans, vec_expect,0.01,1e-08):
  			ret['ok'] = True
  		else:
    		# More error message. Will improve this part
  			ret['msg'] = 'something is wrong' 
  	except:
  		ret['msg'] = 'Wrong input format'
  	return ret 
\end{edXscript}


\edXabox{type="custom" cfn="VectorEntry" expect="[[4],[2]]"}


\edXsolution{ 

}


\endedxproblem


\endedxvertical





\beginedxvertical{Standard Matrix}



\doedxvideo{The Standard Matrix for a Linear Transformation}{aPUQJWsg9Vw}

\beginedxtext{Standard Matrix}

If $T: \R^n \rightarrow \R^m$ is a linear transformation, then there is a unique $m \times n$ matrix
$A$ such that $T(x) = Ax$ for all $x\in \R^n$.  This matrix $A$
is called the {\keyb{\bf standard
matrix}} for $T$, and must be given by the formula
\[A =  \left[ \begin{array}{cccc} | & | & & | \\ 
T(e_1) & T(e_2) & \cdots & T(e_n) \\
 | & | & & | \end{array} \right], \]
 where $e_i \in \R^n$ is the vector whose $i$th entry equals 1, and all of whose other entries are 0.

\endedxtext

\endedxvertical





\beginedxvertical{Standard Matrix Questions}

\beginedxproblem{True or False}{\dpa1}


True or false: Every linear transformation from a vector space $V$ to a vector space $W$
has a standard matrix.  
 
\edXabox{expect="False" options="True","False"}


\edXsolution{ 
False.  Only linear transformations from $\R^n$ to $\R^m$ (or, more generally, from $F^n$ to $F^m$ where
$F$ is a field) have standard matrices.  
}


\endedxproblem


\beginedxproblem{Standard Matrix 1}{\dpa1}

For all $v\in \R^2$, let $T(v)$ be the reflection of $v$ across the horizontal axis of $\R^2$.  
$T$ is a linear transformation from $\R^2$ to $\R^2$.  What is its standard matrix?
 
\begin{edXscript}
def MatrixEntry(expect, ans):
  	import ast
	import numpy as np 
	ret= {'ok':False}
  	atol = 0.01
  	try:
		list_ans = ast.literal_eval(ans)
		list_expect = ast.literal_eval(expect)
  		matrix_ans = np.matrix(list_ans)
  		matrix_expect = np.matrix(list_expect) 
  		if matrix_ans.shape != matrix_expect.shape:
  			ret['msg'] = 'Wrong shape of matrix'
  		elif np.allclose(matrix_ans, matrix_expect,0.01,1e-08):
  			ret['ok'] = True
  		else:
  			ret['msg'] = 'Something is wrong'
	except SyntaxError:
		ret['msg'] = 'Wrong input format'
  	return ret
\end{edXscript}


\edXabox{type="custom" cfn="MatrixEntry" expect="[[1,0],[0,-1]]"}


\edXsolution{ 

}


\endedxproblem


\beginedxproblem{Standard Matrix 2}{\dpa1}

For all $v\in \R^2$, let $S(v)$ be the vector obtained by reflecting $v$ across the diagonal line $x=y$ 
and 
then rotating the result by $90^\circ$ counterclockwise.  
$S$ is a linear transformation from $\R^2$ to $\R^2$.  What is its standard matrix?
 
\begin{edXscript}
def MatrixEntry(expect, ans):
  	import ast
	import numpy as np 
	ret= {'ok':False}
  	atol = 0.01
  	try:
		list_ans = ast.literal_eval(ans)
		list_expect = ast.literal_eval(expect)
  		matrix_ans = np.matrix(list_ans)
  		matrix_expect = np.matrix(list_expect) 
  		if matrix_ans.shape != matrix_expect.shape:
  			ret['msg'] = 'Wrong shape of matrix'
  		elif np.allclose(matrix_ans, matrix_expect,0.01,1e-08):
  			ret['ok'] = True
  		else:
  			ret['msg'] = 'Something is wrong'
	except SyntaxError:
		ret['msg'] = 'Wrong input format'
  	return ret
\end{edXscript}


\edXabox{type="custom" cfn="MatrixEntry" expect="[[-1,0],[0,1]]"}


\edXsolution{ 

}


\endedxproblem



\beginedxproblem{How many linear transformations?}{\dpa1}

How many linear transformations $T:\R^4 \rightarrow \R^2$ satisfy 
the conditions
$T(e_1) = T(e_3) =  \left[\begin{array}{c} 2 \\ 3   \end{array} \right]$
and $T(e_4) = \veco$?  

\edXabox{type="multichoice" expect="infinitely many" options="zero","one","more than one but finitely many","infinitely many"}

\edXsolution{ 

}


\endedxproblem



\endedxvertical





\beginedxvertical{Back to Pictures}


\doedxvideo{More Pictures}{8zRN403o8tg}



\beginedxproblem{What is preserved?}{\dpa1}

Take a look at the video again.  Which properties seem to be preserved by linear transformations
from $\R^2$ to $\R^2$?  


For instance, does every transformation preserve the area of regions? 

\edXabox{expect="No" options="Yes","No"}


Do lines that start parallel always go in the same direction after transformation?  


\edXabox{expect="Yes" options="Yes","No"}

Do lines that start perpendicular remain perpendicular after transformation? 

\edXabox{expect="No" options="Yes","No"}

\edXsolution{  }

\endedxproblem

\endedxvertical










