

\beginedxvertical{Page One}

\beginedxtext{Preliminaries}


At the end of this sequence, and after some practice, you should be able to:

\begin{itemize}
\item Understand the definition of linear transformation.  
\item Be able to determine whether a function between vector spaces is a linear transformations. 
\item Recognize basic properties of linear transformations.
\end{itemize}

For time budgeting purposes, this sequence has 4 videos totaling 14 minutes, 
plus some questions.  

% Remember, when you're doing the online learning sequences, you may seek help if you 
% do not understand a video, but you should think about all of the questions 
% entirely individually.  You have pledged to do so under the Honor Code!  


\endedxtext

\endedxvertical

\beginedxvertical{Introduction}



\doedxvideo{Functions}{nMrAAozL4F0}


\beginedxproblem{System Consistency?}{\dpa1}

Define the function $T$ by $T(\vecx) = A\vecx$, where $A = \left[ \begin{array}{ccc}
6 & 5 & 3 \\
2 & 4 &  23 
\end{array} \right].$

What is the domain of $T$?

\edXabox{type="multichoice" expect="$\R^3$" options="$\R$","$\R^2$","$\R^3$"}

What is the codomain of $T$?

\edXabox{type="multichoice" expect="$\R^2$" options="$\R$","$\R^2$","$\R^3$"}

\edXsolution{ 
$A$ has 3 columns. Therefore, for the product $A\vecx$ to be defined, $\vecx$ must have 3 rows, and will be a vector in $\R^3$. \\
Since $A$ is 2x3 and $\vecx$ is 3x1, the product $A\vecx$ will be 2x1, a vector in $\R^2$.
}

\endedxproblem



\endedxvertical





\beginedxvertical{Defining Linear Transformations}

\doedxvideo{Linear Transformations}{ASQadpJlL94}
%example: rotations, non-example: something quadratic


\endedxvertical









\beginedxvertical{Linear Transformation Definition}

\beginedxtext{Definition of Linear Transformation}

Let $\F$ be a field.  A function $T: \F^n \rightarrow \F^m$ is a {\keyb{\bf linear transformation}}
if both of the following properties hold:

\begin{itemize}
\item For any vectors $\vecv, \vecw$ in the domain, $T(\vecv + \vecw) = T(\vecv) + T(\vecw).$ 
\item For any vector $\vecv$ in the domain, and any scalar $a\in \F$, $T(a\vecv) = aT(\vecv)$.    
\end{itemize}

Informally, if $T$ satisfies the first condition we say that ``$T$ respects vector addition", and 
if $T$ satisfies the second condition we say that ``$T$ respects scalar multiplication."  

\endedxtext

\endedxvertical


\beginedxvertical{Some Examples and Non-examples}



\beginedxproblem{Linear Transformation? 1}{\dpa1}

Define $T: \R^3 \rightarrow \R^3$ by $T(\vecv) = 2\vecv$.  

Does $T$ respect vector addition?  That is, is $T(\vecv + \vecw) = T(\vecv) + T(\vecw)$ for all 
vectors $\vecv,\vecw$?
Try a couple of specific numerical examples if you're not sure.  

\edXabox{expect="Yes" options="Yes","No"}

Does $T$ respect scalar multiplication?  
That is, is $T(a\vecv) = aT(\vecv)$ for all vectors $\vecv$ and scalars $a$?
Try a couple of specific numerical examples if you're not sure.  


\edXabox{expect="Yes" options="Yes","No"}

Is $T$ a linear transformation?

\edXabox{expect="Yes" options="Yes","No"}

\edXsolution{ 
We can demonstrate with a specific example: \\
Let $v_1 = \left[ \begin{array}{c}
1\\2\\5 
\end{array} \right]$, 
$v_2 = \left[ \begin{array}{c}
-4\\0\\9 
\end{array} \right]$\\
It checks out that $T(v_1) + T(v_2) = T(v_1 + v_2)$:\\
$\left[ \begin{array}{c} 2\\4\\10 \end{array} \right] 
+ \left[ \begin{array}{c} -8\\0\\18 \end{array} \right]
= \left[ \begin{array}{c} -6\\4\\28 \end{array} \right]
= 2 \left( \left[ \begin{array}{c} 1\\2\\5 \end{array} \right]
+ \left[ \begin{array}{c} -4\\0\\9 \end{array} \right] \right) $ \\
\\
It also checks out that $cT(v_1) = T(cv_1)$: \\
$8\left[ \begin{array}{c} 2\\4\\10 \end{array} \right] 
= 2\left[ \begin{array}{c} 8\\16\\40 \end{array} \right] 
= \left[ \begin{array}{c} 16\\32\\80 \end{array} \right] $ \\
\\
It appears that the transformation is linear. We can also prove it with algebra: \\

$T(v_1) + T(v_2) = 2v_1 + 2v_2
= 2(v_1 + v_2) = T(v_1 + v_2)$

$cT(v_1) = c \cdot 2(v_1) =  2 \cdot c(v_1) = T(cv_1)$

}

\endedxproblem


\beginedxproblem{Linear Transformation? 2}{\dpa1}

Define $T: \R^2 \rightarrow \R^2$ by $T(\vecv) = \veco$.  

Does $T$ respect vector addition?  That is, is $T(\vecv + \vecw) = T(\vecv) + T(\vecw)$ for all 
vectors $\vecv,\vecw$?
Try a couple of specific numerical examples if you're not sure.  

\edXabox{expect="Yes" options="Yes","No"}

Does $T$ respect scalar multiplication?  
That is, is $T(a\vecv) = aT(\vecv)$ for all vectors $\vecv$ and scalars $a$?
Try a couple of specific numerical examples if you're not sure.  


\edXabox{expect="Yes" options="Yes","No"}

Is $T$ a linear transformation?

\edXabox{expect="Yes" options="Yes","No"}

\edXsolution{ 

Proof that $T$ is a linear transformation:

$T(v_1) + T(v_2) = 0 + 0 = 0 = T(v_1 + v_2)$

$cT(v_1) = c \cdot 0 = 0 = T(cv_1)$

}

\endedxproblem

\beginedxproblem{Linear Transformation? 3}{\dpa1}

Define $T: \R^2 \rightarrow \R^2$ by $T(\vecv) = \left[\begin{array}{c}
4 \\
0 
\end{array} \right]$.  

Does $T$ respect vector addition?  That is, is $T(\vecv + \vecw) = T(\vecv) + T(\vecw)$ for all 
vectors $\vecv,\vecw$?
Try a couple of specific numerical examples if you're not sure.  

\edXabox{expect="No" options="Yes","No"}

Does $T$ respect scalar multiplication?  
That is, is $T(a\vecv) = aT(\vecv)$ for all vectors $\vecv$ and scalars $a$?
Try a couple of specific numerical examples if you're not sure.  


\edXabox{expect="No" options="Yes","No"}

Is $T$ a linear transformation?

\edXabox{expect="No" options="Yes","No"}

\edXsolution{ 
Here, trying specific vectors provides a counterexample, which is sufficient to show that $T$ is not a linear transformation.\\
Let $v_1 = \left[ \begin{array}{c}
1\\2\\5 
\end{array} \right]$, 
$v_2 = \left[ \begin{array}{c}
-4\\0\\9 
\end{array} \right]$\\
$T(v_1) + T(v_2) = \left[\begin{array}{c} 4 \\ 0 \end{array} \right] 
+  \left[\begin{array}{c} 4 \\ 0 \end{array} \right]  
=  \left[\begin{array}{c} 8 \\ 0 \end{array} \right]$\\
Meanwhile, $T(v_1 + v_2) = \left[\begin{array}{c} 4 \\ 0 \end{array} \right]$.
$T$ does not respect vector addition and is not linear.\\
$8\cdotT(v_1) = 8 \left[\begin{array}{c} 4 \\ 0 \end{array} \right] 
= \left[\begin{array}{c} 32 \\ 0 \end{array} \right]$\\
Meanwhile, $T(8v_1) = \left[\begin{array}{c} 4 \\ 0 \end{array} \right]$.
$T$ does not respect scalar multiplication; this on its own would also be enough to know that $T$ is not linear.
\endedxproblem


\endedxvertical


\beginedxvertical{More Linear Transformations}

\doedxvideo{A Linear Transformation outside of R^n}{Pyv9GXNYJ44}



\beginedxproblem{Linear Transformation? 4}{\dpa1}

Define $T: \mathbb{P} \rightarrow \R$ by $T(f) = f(1)$.  

To get a sense of how $T$ works, let's do a quick example.  Suppose $f$ is the element of 
$\mathbb{P}$ given by $f(t) = 2t^2 -5$.  
What is $T(f)$?  

\edXabox{type="numerical" expect="-3"}


Does $T$ respect vector addition?  That is, is $T(f+g) = T(f) + T(g)$ for all 
vectors $f,g \in \mathbb{P}$?
Try a couple of specific numerical examples if you're not sure.  

\edXabox{expect="Yes" options="Yes","No"}

Does $T$ respect scalar multiplication?  
That is, is $T(af) = aT(f)$ for all vectors $f \in \mathbb{P}$ and scalars $a$?
Try a couple of specific numerical examples if you're not sure.  


\edXabox{expect=Yes" options="Yes","No"}

Is $T$ a linear transformation?

\edXabox{expect="Yes" options="Yes","No"}

\edXsolution{ 
$T$ respects vector addition because:\\
$f(1) + g(1) = (f+g)(1)$\\
i.e. we can evaluate polynomials at 1 separately and add the results, or we can add the polynomials and evaluate the result at 1, and we'll get the same result.
This must be true in general for all polynomials $f$ and $g$. We can test this with some specific examples, but we give a (somewhat sketchy) proof here.\\ Suppose $f$ and $g$ contain no like terms between them. Then the right side of the equation, the evaluation of $f+g$ at 1, would be calculated by evaluating the terms of $f$ and $g$ separately and then adding the results, which is precisely the left hand side of the equation. Otherwise, $f$ and $g$ contain like terms, and for any two terms $ax^n$ and $bx^n$ belonging to $f$ and $g$ respectively, $a(1^n) + b(1^n) = a + b = (a+b)(1^n)$.\\
$T$ respects scalar multiplication because:\\
$cT(f) = cf(1) = (cf)(1) = T(cf)$\\
i.e. we can evaluate a polynomial at 1 and multiply the result by a scalar, or multiply the polynomial by the scalar first and then evaluate the result at 1, and we'll get the same result. Again, it is simple to try some examples to observe this, but we can also prove it.\\
Suppose $f(x) = a_nx^n + a_{n-1}x_{n-1} + ... a_1x + a_0$. Then $cf(1) = c(a_n + ... + a_0) = ca_n + ca_{n-1} + ... + ca_0 = (cf)(1)$.

}

\endedxproblem



\endedxvertical


\beginedxvertical{Linear Transformations Properties}

\doedxvideo{Linear Transformation Properties}{oe4fgdPAsgA}




\beginedxproblem{Linear Transformation Practice}{\dpa1}

Suppose $T: \mathbb{P} \rightarrow \R^2$ is a linear transformation.  Let $f,g\mathbb{P}$ be
the polynomials given by $f(t) = t^2$ and $g(t) = t$.  

Suppose that $T(f) = \left[\begin{array}{c} 1 \\ 2  \end{array} \right]$
and $T(g) = \left[\begin{array}{c} 2 \\ -1  \end{array} \right]$



If $h \in \mathbb{P}$ is the polynomial given by $h(t) = -2t^2$, what must $T(h)$ be?

To enter a vector such as $\left[\begin{array}{c} 1 \\ 2  \\ 3 \end{array} \right]$, you can either:
\begin{itemize}
\item
enter it as 
you would a $3\times 1$ matrix: [[1],[2],[3]]  
\item
or, you can enter it as <1,2,3>
\end{itemize}

Use decimals only.  

\begin{edXscript}
def VectorEntry(expect, ans):
	import ast
	import numpy as np 
  	atol = 0.01
	list_expect = ast.literal_eval(expect)
	vec_expect = np.matrix(list_expect)
  	ret = {"ok":False}
	try:
  		# input format [[1],[2],[3]]
		list_ans = ast.literal_eval(ans)
		vec_ans = np.matrix(list_ans)
  		if vec_ans.shape != vec_expect.shape:
  			ret['msg'] = 'Wrong shape of vector!'
  		elif np.allclose(vec_ans, vec_expect,atol,1e-08):
  			ret['ok'] = True
  		else:
  		# More error message. Will improve this part
  			ret['msg'] = 'something is wrong'   			
	#except SyntaxError:
		#ret['msg'] = 'Wrong input format'
	except SyntaxError:
  		# input format &lt;1,2,3&gt;
		list_ans = ans.replace('&lt;', '[').replace('&gt;', ']')
		list_ans = ast.literal_eval(list_ans)
		vec_ans = np.transpose(np.matrix(list_ans))
  		if vec_ans.shape != vec_expect.shape:
  			ret['msg'] = 'Wrong shape of vector!'
  		elif np.allclose(vec_ans, vec_expect,0.01,1e-08):
  			ret['ok'] = True
  		else:
    		# More error message. Will improve this part
  			ret['msg'] = 'something is wrong' 
  	except:
  		ret['msg'] = 'Wrong input format'
  	return ret 
\end{edXscript}



\edXabox{type="custom" cfn="VectorEntry" expect="[[-2],[-4]]"}


If $j \in \mathbb{P}$ is the polynomial given by $h(t) = 3t^2-t$, what must $T(j)$ be?



\edXabox{type="custom" cfn="VectorEntry" expect="[[1],[7]]"}


\edXsolution{ 
We are assured that $T$ is a linear transformation, so we have: \\
$T(h) = T(-2t^2)  = -2T(t^2) = -2T(f) = -2\left[\begin{array}{c} 1 \\ 2  \end{array} \right] = \left[\begin{array}{c} -2 \\ -4  \end{array} \right]$\\ \\
$T(j) = T(3t^2-t) = T(3t^2) + T(-t) = 3T(t^2) + -T(t) = 3T(f) - T(g)\\ = 3\left[\begin{array}{c} 1 \\ 2  \end{array} \right] - \left[\begin{array}{c} 2 \\ -1  \end{array} \right] = \left[\begin{array}{c} 1 \\ 7  \end{array} \right]$
}


\endedxproblem



\endedxvertical


% \beginedxvertical{Linear Transformations Properties}

% \doedxvideo{Matrix Multiplication as a Linear Transformation}{jMRJ-efZOJs}



% \endedxvertical





