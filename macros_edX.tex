%\documentclass[12pt]{article}
%
%\usepackage{edXpsl}	% edX
%\usepackage{amsmath, amsthm, amsfonts, amssymb, color, mathrsfs, comment}
%\usepackage{cancel}
%
%%------------------------------------------
%\parindent=0pt
%\parskip=1ex
%
%\begin{document}
%%-----------------------------------------------%

\newcounter{edXtext}
\newcounter{edXproblem}
\newcounter{edXvideo}
\newcounter{showhide}
\newcounter{psetproblem}
%\newcounter{edXvertical}


%If seems \ifplastex needs to go after the \begin{document}
% \newif\ifplastex 
% \plastexfalse
% \ifplastex
  %% Many of these could be better handled in python. When I get the chance to understand plastex better I'll do that. JMO
   % \def\USING{USING PLASTEX FIX MACROS}
   \def\cancel#1{#1}
%   \def\footnote#1{} %\par {\small{\color{red} FOOTNOTE:} #1}}
   \def\mylabel#1{\label{#1}\tag{\value{equation}}}

   \def\displaynametag{display_name}

   %\def\includesvg{\includegraphics}
   %\def\includegraphics{\edXincludegraphics}

   \newcounter{mynumbereditem}
   \newcounter{temp}
   \def\mysecnum{\arabic{section}}
   \def\mynumbereditemnum{\arabic{mynumbereditem}}
   \def\mynumbereditemlabel{\mysecnum.\mynumbereditemnum}

   \def\mysection#1{\section{#1}}

   \newenvironment{theorem}{\addtocounter{mynumbereditem}{1}%
\textbf{Theorem \mynumbereditemlabel.} }{\par\vspace{.5in}}
   \newenvironment{proposition}{\addtocounter{mynumbereditem}{1}%
\textbf{Proposition \mynumbereditemlabel.} }{\par\vspace{.5in}}
   \newenvironment{lemma}{\addtocounter{mynumbereditem}{1}%
\textbf{Lemma \mynumbereditemlabel.} }{\par\vspace{.5in}}
   \newenvironment{definition}{\addtocounter{mynumbereditem}{1}%
\textbf{Definition \mynumbereditemlabel.} }{\par\vspace{.5in}}
   \newenvironment{remarks}{\addtocounter{mynumbereditem}{1}%
\textbf{Remarks \mynumbereditemlabel.} }{\par\vspace{.5in}}
   \newenvironment{corollary}{\addtocounter{mynumbereditem}{1}%
\textbf{Corollary \mynumbereditemlabel.} }{\par\vspace{.5in}}
   \newenvironment{remark}{\addtocounter{mynumbereditem}{1}%
\textbf{Remark \mynumbereditemlabel.} }{\par\vspace{.5in}}
   \newenvironment{examples}{\addtocounter{mynumbereditem}{1}%
\textbf{Examples \mynumbereditemlabel.} }{\par\vspace{.5in}}
   \newenvironment{example}{\addtocounter{mynumbereditem}{1}%
\textbf{Example \mynumbereditemlabel.} }{\par\vspace{.5in}}
   \newenvironment{exercise}{\addtocounter{mynumbereditem}{1}%
\textbf{Exercise \mynumbereditemlabel.} }{\par\vspace{.5in}}

   \newenvironment{stheorem}{\textbf{Theorem. }}{\par\vspace{.5in}}
   \newenvironment{sproposition}{\textbf{Proposition. }}{\par\vspace{.5in}}
   \newenvironment{slemma}{\textbf{Lemma. }}{\par\vspace{.5in}}
   \newenvironment{sdefinition}{\textbf{Definition. }}{\par\vspace{.5in}}
   \newenvironment{sremarks}{\textbf{Remarks. }}{\par\vspace{.5in}}
   \newenvironment{scorollary}{\textbf{Corollary. }}{\par\vspace{.5in}}
   \newenvironment{sremark}{\textbf{Remark. }}{\par\vspace{.5in}}
   \newenvironment{sexamples}{\textbf{Examples. }}{\par\vspace{.5in}}
   \newenvironment{sexample}{\textbf{Example. }}{\par\vspace{.5in}}
   \newenvironment{sexercise}{\textbf{Exercise. }}{\par\vspace{.5in}}


% \def\dpa{weight="1" showanswer="attempted" attempts="3"}

\def\beginedxsequential#1#2{\begin{edXsection}{#1}[#2 url_name="\edxbaseoutputname-sequential"]}

\def\endedxsequential{\end{edXsection} \setcounter{psetproblem}{0}}


\def\beginedxtext#1{\refstepcounter{edXtext}\begin{edXtext}{#1}[url_name="\edxbaseoutputname-tab\theedXvertical-text\theedXtext"]}
\def\endedxtext{\end{edXtext}}
\def\beginedxproblem#1#2{\refstepcounter{edXproblem}\begin{edXproblem}{#1}{url_name="\edxbaseoutputname-tab\theedXvertical-problem\theedXproblem" #2}}
\def\endedxproblem{\end{edXproblem}}

%generate pset problem names
\def\beginedxpset#1#2{\refstepcounter{edXproblem}\refstepcounter{psetproblem}\begin{edXproblem}{#1 (\thepsetproblem)}{url_name="\edxbaseoutputname-tab\theedXvertical-problem\theedXproblem" #2}}
\def\endedxpset{\end{edXproblem}}

\def\beginedxvertical#1{\begin{edXvertical}{#1}[url_name="\edxbaseoutputname-vertical\theedXvertical"]}
\def\endedxvertical{\end{edXvertical} \setcounter{edXtext}{0} \setcounter{edXproblem}{0} \setcounter{edXvideo}{0}}


%New-allow for source command
\providecommand{\doedxvideo}[3][]{\refstepcounter{edXvideo}\edXvideo{#2}{#3}[url_name="\edxbaseoutputname-tab\theedXvertical-video\theedXvideo" #1]}

%New wrapper.
\newenvironment{shh}{}{}
\def\beginedxshowhide#1{\begin{shh}\begin{edXshowhide}{#1}}  
\def\endedxshowhide{\end{edXshowhide}\end{shh}}


% EVH added
% \def\edXmathlet#1{\edXxml{<iframe src="https://s3.amazonaws.com/1801-static-assets/build/#1.html" width="820 px" height="630 px" style="border:0px"/>}}

%JEF added
% \providecommand{\includesvg}[2][400]{\edXxml{<img src="images/#2.svg" width="#1 px" style="margin: 10px 25px 25px 25px; border:0px"/>}}
\providecommand{\includesvg}[2][400]{\edXxml{<img src="/static/images/#2.svg" width="#1 px" style="margin: 10px 25px 25px 25px; border:0px"/>}}


%-------------------------------------------------%

\def\myheader#1{\noindent \textbf{#1}\par}

\def\ds{\displaystyle}
\def\Re{\mathrm{Re\,}}
\def\Im{\mathrm{Im\,}}

\newcommand{\R}{\mathbb R}
\newcommand{\Z}{\mathbb Z}
\newcommand{\C}{\mathbb C}
\newcommand{\Q}{\mathbb Q}
\newcommand{\inv}{^{-1}}
\newcommand{\eps}{\epsilon}
\newcommand{\veco}{\mathbf{0}}

\newcommand{\F}{{\bf {F}}}

%commands for Mattuck/Jerison Problems
% \def\qw{\qquad}
% \def\q{\quad}
% \def\f{\frac}
% \def\disp{\displaystyle}
% \def\To{\implies}
% \def\e{\epsilon}
% \def\t{\theta}
% \def\D{\Delta}

%These two are needed for content from David Jerison's notes
% \def\Implies{\ \implies \ }
% \def\Iff{\ \iff \ }

\def\imgdir{images}
%_________________________________



%\def\defaultproblemattributes{attempts="5" showanswer="attempted" rerandomization="per_student"}


% \def\qw{\qquad}
% \def\q{\quad}
% \def\fig{\includegraphics}

% \def\inv{^{-1}}

% \def\dpaA{showanswer="finished" attempts="1" rerandomize="per_student"}
% \def\dpaB{showanswer="finished" attempts="3" rerandomize="per_student"}
% \def\dpaC{showanswer="finished" attempts="5" rerandomize="per_student"}
% \def\dpaD{showanswer="finished" attempts="7" rerandomize="per_student"}
% \def\dpa#1{showanswer="finished" attempts="#1" rerandomize="per_student"}
\def\dpa#1{showanswer="finished" attempts="#1" rerandomize="per_student"}
% \def\dpwa[1]{showanswer="finished" weight="#1" attempts="1" rerandomize="per_student"}
% \def\dpwb[1]{showanswer="finished" weight="#1" attempts="2" rerandomize="per_student"}
% \def\dpwc[1]{showanswer="finished" weight="#1" attempts="3" rerandomize="per_student"}
% \def\dpwd[1]{showanswer="finished" weight="#1" attempts="4"  rerandomize="per_student"}
% \def\dpwe[1]{showanswer="finished" weight="#1" attempts="5" rerandomize="per_student"}
\def\dpaZ{attempts="100" rerandomize="onreset" showanswer="attempted"}
\def\dpadnd{showanswer="attempted" attempts="3" rerandomize="onreset"}
\def\resetdpa#1{showanswer="attempted" attempts="#1" rerandomize="onreset"}


%color for 18.01x
%added by Jen
\definecolor{blue}{cmyk}{1,1,0,0}
\definecolor{orange}{cmyk}{0,0.5,1,0}

\definecolor{bordeaux}{cmyk}{0,.84,.71,.40}
\newcommand{\keya}{\color{bordeaux}}

\definecolor{royalblue}{cmyk}{.72,.54, 0, .45}
% \newcommand{\keyb}{\color{royalblue}}
\newcommand{\keyb}{\color{bordeaux}}


\protected\def\blue#1{%
  \ifmmode
  	{\color{blue}{#1}}
  \else
  	\textbf{{\color{blue}{#1}}}
   \fi
}
\protected\def\red#1{%
  \ifmmode
  	{\color{orange}{#1}}
  \else
  	\textbf{{\color{orange}{#1}}}
   \fi
}



