

\beginedxvertical{Page One}

\beginedxtext{Preliminaries}





At the end of this sequence, and after some practice, you should be able to:

\begin{itemize}
\item Describe the span of some lists of vectors.
\item Determine whether a list of vectors in $\R^m$ spans $\R^m$.  
\end{itemize}


For time budgeting purposes, this sequence has X videos totaling X minutes, 
plus some questions.  




\endedxtext

\endedxvertical


\beginedxvertical{Introduction to Linear Combinations}

\doedxvideo{Linear Combinations}{uupJBrWff-c}

\beginedxtext{Definition of Linear Combination}

Let $V$ be a vector space over a field $F$.  A {\keyb{\bf linear combination}} of a list of vectors $\{v_1; v_2; 
\ldots v_n\}$
in $V$ is a vector of the form $a_1v_1 + a_2v_2 + \ldots + a_n v_n$, where the $a_i$ are scalars in $F$.  


\endedxtext


\endedxvertical


\beginedxvertical{Linear Combinations}


\beginedxproblem{Linear Combo 1}{\dpa1}


Let $v_1 = \left[\begin{array}{c} 2 \\ 3  \\ 4 \end{array} \right]$ and  
$v_2 = \left[\begin{array}{c} 1 \\ -1  \\ 0 \end{array} \right]$ be vectors in $\R^3$.

Can you write the vector $x  = \left[\begin{array}{c} -4 \\ -6  \\ -8 \end{array} \right]$
as a linear combination $x = a_1 v_1 + a_2 v_2$?  If so, enter possible values for $a_1$ and
$a_2$.  If not, enter No in both boxes.  

\edXinline{$a_1= $ }\edXabox{type="formula" expect="-2" samples="No@1:5#5" feqin="1" tolerance=".01" inline="1"}\\
\edXinline{$a_2= $ }\edXabox{type="formula" expect="0" samples="No@1:5#5" feqin="1" tolerance=".01" inline="1"}\\


\edXsolution{ We may notice that $x$ is a scalar multiple of $v_1$: 
\\ \\$-2 \cdot \left[\begin{array}{c} 2 \\ 3  \\ 4 \end{array} \right] =  \left[\begin{array}{c} -4 \\ -6  \\ -8 \end{array} \right]$
\\ $-2 \cdot \,\,\,\,\,\, v_1 \,\,\,\,\,= \,\,\,\,\,\,\,\,\,\,x$
\\ \\It follows immediately that $a_1=-2$,$a_2= 0$ is a solution. 

}


\endedxproblem

\beginedxproblem{Linear Combo 2}{\dpa1}


Let $v_1 = \left[\begin{array}{c} 2 \\ 3  \\ 4 \end{array} \right]$ and  
$v_2 = \left[\begin{array}{c} 1 \\ -1  \\ 0 \end{array} \right] $  be vectors in $\R^3$.

Can you write the zero vector in $\R^3$
as a linear combination $\veco = a_1 v_1 + a_2 v_2$?  If so, enter possible values for $a_1$ and
$a_2$.  If not, enter No in both boxes.  

\edXinline{$a_1= $ }\edXabox{type="formula" expect="0" samples="No@1:5#5" feqin="1" tolerance=".01" inline="1"}\\
\edXinline{$a_2= $ }\edXabox{type="formula" expect="0" samples="No@1:5#5" feqin="1" tolerance=".01" inline="1"}\\



\edXsolution{ For any set of vectors in $\R^n$, we can always combine them to obtain the zero vector in $\R^n$ by simply setting all of the coefficients to zero. 
%Note that for certain sets of vectors in $\R^n$, it is also possible to write the zero vector as a linear combination with non-zero coefficients: e.g., \\ $1 \cdot \left[\begin{array}{c} 3  \\ 4 \end{array} \right] + -1 \cdot \left[\begin{array}{c} -3  \\ \\  -4 \end{array} \right] = \veco$. 
}


\endedxproblem


\beginedxproblem{Linear Combo 3}{\dpa1}


Let $v_1 = \left[\begin{array}{c} 2 \\ 3  \\ 4 \end{array} \right]$ and  
$v_2 = \left[\begin{array}{c} 1 \\ -1  \\ 0 \end{array} \right]$  be vectors in $\R^3$.

Can you write $v_2$
as a linear combination of $v_1$?   

\edXabox{expect="No" options="Yes","No"}


\edXsolution{ A "yes" answer here would imply that some scalar, multiplied by $v_1$, equals $v_2$. If $c \cdot v_1 = v_2$, then each entry of $v_1$, times $c$, equals its respective entry in $v_2$: \\ \\ $c \cdot 2 = 1$, \,\, $c \cdot 3 = -1$, \,\, and $c \cdot 4 = 0$. \\ \\ We can see that this is inconsistent; for example, the third equation implies $c = 0$, and yet that would not satisfy the first or second equation.  
}


\endedxproblem


\beginedxproblem{Linear Combo 4}{\dpa1}

\begin{center}
\includesvg[300]{c1s4lincombo}   
\end{center}

Given the picture above in $\R^2$, 
can you write $w$ as a linear combination $w = a_1 v_1 + a_2 v_2$?  If so, enter possible values for $a_1$ and
$a_2$.  If not, enter No in both boxes.  


\edXinline{$a_1= $ }\edXabox{type="formula" expect="-1" samples="No@1:5#5" feqin="1" tolerance=".01" inline="1"}\\
\edXinline{$a_2= $ }\edXabox{type="formula" expect="2" samples="No@1:5#5" feqin="1" tolerance=".01" inline="1"}\\



\edXsolution{ Notice how the grid of parallelograms shows us where we can expect to end up by applying $v_1$ and $v_2$ by various amounts; each lattice point is a linear combination, with integer coefficients, of $v_1$ and $v_2$. The question is, what combination of $v_1$ and $v_2$ gives us the same net effect as $w$? Since the head of $w$ lies on a lattice point of the grid, we can simply count our way to the answer: a movement of -1 spaces in the $v_1$ direction, and a movement of 2 spaces in the $v_2$ direction, takes us from the tail to the head of $w$. 
%\\ Notice also that we can make these moves in either order to accomplish the same thing; after all, addition is commutative, and by making a series of moves on this grid, we are simply adding vectors.
}

\endedxproblem

\beginedxproblem{Linear Combo 5}{\dpa1}


Let $f$ denote the polynomial $t^3  + 1$, and let $g$ denote the

polynomial $t^2 - t$.  Recall that the set of polynomials is a vector space over $\R$, so $f$ and $g$
are vectors in that vector space.  

Let $h$ denote $17t^3 + 2t^2 - 2t + 17$.
Can you write $h$  as a linear combination $h = a_1 f + a_2 g$?  If so, enter possible values for $a_1$ and
$a_2$.  If not, enter No in both boxes.  

\edXinline{$a_1= $ }\edXabox{type="formula" expect="17" samples="No,t@1,1:5,5#10" feqin="1" tolerance=".01" inline="1"}\\
\edXinline{$a_2= $ }\edXabox{type="formula" expect="2" samples="No,t@1,1:5,5#10" feqin="1" tolerance=".01" inline="1"}\\


Let $j$ denote $3t^4 + 3t$.
Can you write $j$  as a linear combination $j = a_1 f + a_2 g$?  If so, enter possible values for $a_1$ and
$a_2$.  If not, enter No in both boxes.  

\edXinline{$a_1= $ }\edXabox{type="formula" expect="No" samples="No,t@1,1:4,5#10" feqin="1" tolerance=".01" inline="1"}\\
\edXinline{$a_2= $ }\edXabox{type="formula" expect="No" samples="No,t@1,1:5,5#10" feqin="1" tolerance=".01" inline="1"}\\


\edXsolution{ 
For the first question, a little bit of trial and error gives us that $h = 17f + 2g$.  
% The coefficients of $h$ can be thought of as a vector: \\ \\ $v_h = \left[\begin{array}{c} 17 \\ 2  \\ -2 \\ 17 \end{array} \right]$. \\ \\ It may help to think the same way about the coefficients of $f$ and $g$, as vectors in $\R^4$: \\ \\ $v_f = \left[\begin{array}{c} 1 \\ 0  \\ 0 \\ 1 \end{array} \right]$ and $v_g = \left[\begin{array}{c} 0 \\ 1  \\ -1 \\ 0 \end{array} \right]$. \\ \\ 
% The question, then, is equivalent to asking if $v_h$ is a linear combination of $v_f$ and $v_g$. The zeros in $v_f$ and $v_g$ lead us to the solution; for example, because the first entry of $v_g$ is zero, and we need a combination of 17 for the first entry of $v_h$, we know that $a_1$ must be 17. (Analogously, because $g$ has no cubic term, and we need a combination of $17t^3$ for the first term of $h$, we know that 17 times the cubic term $t^3$ in $f$ is the only way to do the job.)
\\ \\ 
In the second problem, we are asked to combine $f$ and $g$ to make a polynomial with a quartic term (one that involves $t^4$). There is no way to multiply $f$ or $g$ by a scalar to get a quartic term.  It is true that 
$j = 3t\cdot f + 0g,$ but $3t$ is not a scalar.  

}


% \beginedxproblem{Linear Combo 5}{\dpa1}


% Let $v_1 = \left[\begin{array}{c} 0 \\ 37  \\ 2i \end{array} \right]$ and  
% $v_2 = \left[\begin{array}{c} 7i \\ -19  \\ 0 \end{array} \right]$ be vectors in the vector space $\C^3$ (over the scalar
% field $\C$).  

% Can you write the vector $x  = \left[\begin{array}{c} -7i \\ 19 + 37i  \\ -2 \end{array} \right]$
% as a linear combination $x = a_1 v_1 + a_2 v_2$?  If so, enter possible values for $a_1$ and
% $a_2$.  If not, enter No in both boxes.  (Use i for $i$ and * for multiplication.)  

% \edXabox{expect="Placeholder" options="Placeholder"}


% \edXsolution{ 
% }

\endedxvertical


\beginedxvertical{Is it a linear combination?}



\doedxvideo{Linear Combinations in R^n}{v8XhmvbzIEU}


\beginedxproblem{Linear Combo 6}{\dpa1}


Let $v_1 = \left[\begin{array}{c} 1 \\ 3  \\ 1 \end{array} \right]$ and  
$v_2 = \left[\begin{array}{c} 3 \\ 10  \\ 2 \end{array} \right]$  be vectors in $\R^3$.

Use row reduction to determine the answer to the following: 
Can you write $w = \left[\begin{array}{c} 1 \\ 5 \\ -1 \end{array} \right]$
as a linear combination $w = a_1 v_1 + a_2 v_2$?  If so, enter possible values for $a_1$ and
$a_2$.  If not, enter No in both boxes.  


\edXinline{$a_1= $ }\edXabox{type="formula" expect="-5" samples="No@1:5#5" feqin="1" tolerance=".01" inline="1"}\\
\edXinline{$a_2= $ }\edXabox{type="formula" expect="2" samples="No@1:5#5" feqin="1" tolerance=".01" inline="1"}\\


\edXsolution{ 

% We are solving the system given by the augmented 
% \\ \\ 
% $\left[\begin{array}{cc} v_1 & v_2 \end{array} \right] \cdot \left[\begin{array}{c} a_1 \\ a_2 \end{array} \right] = 
% \left[\begin{array}{c} w \end{array} \right]$
% \\ \\ 
With the vectors $v_1$, $v_2$, and $w$, form an augmented matrix:
\\ \\ 
$\left[\begin{array}{cc:c} 1 & 3 & 1 \\ 3 & 10 & 5 \\ 1 & 2 & -1  \end{array} \right]$
\\ \\ 
Multiply the first row by -3, and add this to the second row:
\\ \\ 
$\left[\begin{array}{cc:c}  1 & 3 & 1 \\ 0 & 1 & 2 \\ 1 & 2 & -1 \end{array} \right]$
\\ \\ 
Multiply the first row by -1, and add this to the third row:
\\ \\ 
$\left[\begin{array}{cc:c} 1 & 3 & 1 \\ 0 & 1 & 2 \\ 0 & -1 & -2 \end{array} \right]$
\\ \\ 
Add the second row to the third row:
\\ \\ 
$\left[\begin{array}{cc:c} 1 & 3 & 1 \\ 0 & 1 & 2 \\ 0 & 0 & 0 \end{array} \right]$
\\ \\ 
Multiply the second row by -3, and add this to the first row:
\\ \\ 
$\left[\begin{array}{cc:c} 1 & 0 & -5 \\ 0 & 1 & 2 \\ 0 & 0 & 0 \end{array} \right]$
\\ \\ 
We've fully row reduced the matrix; we see that the system is consistent and that $a_1 = -5$ and $a_2 = 2$.  
(Note also that we have no free variables; our solution, given by the right side of the augmented matrix, is unique.)
\\ \\ 
You can verify your solution by computing $-5v_1 + 2v_2$; you should get $w$.  

}


\endedxproblem

\endedxvertical


\beginedxvertical{Span as Noun}


\doedxvideo{Span as a Noun}{BhyIFSpOcRs}

\endedxvertical


\beginedxvertical{Span Questions}

\beginedxtext{Span (Noun)}

Given vectors $v_1, v_2, \ldots v_n$ in a vector space $V$ over a field $F$, we define
the {\keyb{\bf span}} of $\{v_1; v_2; \ldots v_n\}$ to be the set of all linear combinations
of  $v_1, v_2, \ldots v_n$.  In other words, 
\[ \mathrm{Span}\{v_1; v_2; \ldots v_n\} = \{a_1 v_1  + a_2 v_2 + \ldots + a_nv_n  : a_i \in F \}.\]

The span of a list with zero vectors is defined to be $\{\veco\}$.  

% In $\R^m$,...

\endedxtext



\beginedxproblem{Span Question}{\dpa1}


Let $v_1 = \left[\begin{array}{c} 1 \\ 1  \\ 0 \end{array} \right]$, 
$v_2 = \left[\begin{array}{c} 0 \\ -1  \\ 2 \end{array} \right]$,
$v_3 = \left[\begin{array}{c} -1 \\ -2  \\ 3 \end{array} \right]$, and
$v_4 = \left[\begin{array}{c} -2 \\ -5  \\ 8 \end{array} \right]$.  

Use row reduction to determine if  $w = \left[\begin{array}{c} 3 \\ 1 \\ 3 \end{array} \right]$
is in the span of $\{v_1; v_2; v_3; v_4 \}$.  Is $w \in \mathrm{Span}\{v_1; v_2; v_3; v_4 \}$?   

\edXabox{expect="Yes" options="Yes","No"}


\edXsolution{ 
The question is equivalent to asking about the consistency of a 
system of linear equations given by the following augmented matrix:
\\ \\ 
$\left[\begin{array}{cccc:c} 1 & 0 & -1 & -2 & 3 \\ 1 & -1 & -2 & -5 & 1 \\ 0 & 2 & 3 & 8 & 3 \end{array} \right]$
\\ \\ 
We will row reduce this matrix. Then we'll see if the row reduced augmented matrix represents a consistent system of equations. If so, then the above augmented matrix also represents a consistent system, and $w$ is in the span of the given set of vectors. \\ \\
Multiply the first row by -1, and add this to the second row:
\\ \\ 
$\left[\begin{array}{cccc:c} 1 & 0 & -1 & -2 & 3 \\ 0 & -1 & -1 & -3 & -2 \\ 0 & 2 & 3 & 8 & 3 \end{array} \right]$
\\ \\ 
Multiply the second row by 2, and add this to the third row:
\\ \\ 
$\left[\begin{array}{cccc:c} 1 & 0 & -1 & -2 & 3 \\ 0 & -1 & -1 & -3 & -2 \\ 0 & 0 & 1 & 2 & -1 \end{array} \right]$
\\ \\
We could stop here. Our matrix now has 3 pivots for its 3 rows. It must represent a consistent system of equations. Here we will continue the process of row reduction, but only for the purpose of elucidating our claim.\\
Add the third row to the second row:
\\ \\
$\left[\begin{array}{cccc:c} 1 & 0 & -1 & -2 & 3 \\ 0 & -1 & 0 & -1 & -3 \\ 0 & 0 & 1 & 2 & -1 \end{array} \right]$
\\ \\
Add the third row to the first row:
\\ \\
$\left[\begin{array}{cccc:c} 1 & 0 & 0 & 0 & 2 \\ 0 & -1 & 0 & -1 & -3 \\ 0 & 0 & 1 & 2 & -1 \end{array} \right]$
\\ \\
For good measure, multiply the second row by -1:
\\ \\
$\left[\begin{array}{cccc:c} 1 & 0 & 0 & 0 & 2 \\ 0 & 1 & 0 & 1 & 3 \\ 0 & 0 & 1 & 2 & -1 \end{array} \right]$
\\ \\
% The first three columns are now the identity matrix. Suppose we multiply the left side of our augmented matrix by our variable vector, in the manner of our original equation:
% \\ \\
% $\left[\begin{array}{cccc} 1 & 0 & 0 & 0 \\ 0 & 1 & 0 & 1 \\ 0 & 0 & 1 & 2 \end{array} \right] \cdot \left[\begin{array}{c} a_1 \\ a_2 \\ a_3 \\ a_4 \end{array} \right] =  \left[\begin{array}{c} 2 \\ 3 \\ -1 \end{array} \right]$
% \\ \\
% $\left[\begin{array}{c} a_1 \\ a_2 + a_4 \\ a_3 + 2a_4 \end{array} \right] =  \left[\begin{array}{c} 2 \\ 3 \\ -1 \end{array} \right]$
% \\ \\
The above is consistent, as there is no row of all zeros followed by a non-zero in the right column. Therefore, $w$ is in the span of the given set of vectors. 
% \\ 
% We knew we were on our way to this result when we had a pivot in every row, because in such a case, we can continue to row reduce until our matrix contains $\mathbb{I}$ as a submatrix. At that point (just as above), our system will be consistent no matter what the right hand side is. We can conclude that the given set of vectors must span all of $\R^3$, and indeed, any matrix with m rows and m pivots must span all of $\R^m$. 
}


\endedxproblem

\beginedxproblem{Span Question 2}{\dpa1}

Now determine if  $u = \left[\begin{array}{c} 301.34 \\ -234.8 \\ \pi \end{array} \right]$
is in the span of $\{v_1; v_2; v_3; v_4 \}$.  Try to avoid doing any additional calculation!

Is $u \in \mathrm{Span}\{v_1; v_2; v_3; v_4 \}$?   

\edXabox{expect="Yes" options="Yes","No"}


\edXsolution{ 
The question is equivalent to asking about the consistency of a 
system of linear equations given by the following augmented matrix:
$\left[\begin{array}{cccc:c} 1 & 0 & -1 & -2 & 301.34 \\ 1 & -1 & -2 & -5 & -234.8 \\ 0 & 2 & 3 & 8 & \pi \end{array} \right]$.

However, we don't need to do any new row-reduction -- from our previous calculation, we know that when we row-reduce the coefficient part, there will be no row of all zeros in the row-reduced coefficient matrix (put another way, there will be a pivot in every row).  
Hence, in the augmented matrix,
 there can be no row of all zeros followed by a non-zero in the right-most column.  The system must be consistent, so $u$ is in $\mathrm{Span}\{v_1; v_2; v_3; v_4 \}$.  This will be true
for any vector in $\R^3$, not just $u$.  
}


\endedxproblem


\endedxvertical


\beginedxvertical{Span as Verb}


\doedxvideo{Span as a Verb}{SNkX6dX7MKo}


\beginedxtext{Span (Verb)}


Given vectors $v_1, v_2, \ldots v_n$ in a vector space $V$ over a field $F$, we say that the list $\{v_1; v_2; \ldots v_n\}$ {\keyb{\bf spans}} $V$ if 
the span of the list of vectors is all of $V$.



\endedxtext


\beginedxtext{Spanning in R^m}

If $\{v_1; v_2; \ldots v_n\}$ is a list of vectors in $\R^m$, the list spans $\R^m$ if and only if 
the matrix 
\[ A = \left[ \begin{array}{cccc} | & | & & | \\ 
v_1 & v_2 & \cdots & v_n \\
 | & | & & | \end{array} \right] \]
 row reduces to have a pivot in every row.  
 

If $n<m$, then it is impossible for the list $\{v_1; v_2; \ldots v_n\}$ to span $\R^m$.    


\endedxtext


\endedxvertical


\beginedxvertical{Questions about spanning}



\beginedxproblem{Spanning Question}{\dpa1}

Consider the following vectors in $\R^2$.  

\begin{center}
\includesvg[300]{c1s4lincombo2}   
\end{center}

Is $u$ an element of $\mathrm{Span}\{v_1; v_2\}$?  

\edXabox{expect="Yes" options="Yes","No"}


Is $w$ an element of $\mathrm{Span}\{v_1; v_2\}$?  

\edXabox{expect="Yes" options="Yes","No"}


Is every vector in $\R^2$ an element of $\mathrm{Span}\{v_1; v_2\}$?  

\edXabox{expect="Yes" options="Yes","No"}

Does the list $\{v_1; v_2\}$ span $\R^2$?  

\edXabox{expect="Yes" options="Yes","No"}

\edXsolution{ 
From the parallelogram grid, we can see that $u = -2v_1 - 2v_2$ and $w = 2v_1 + 4v_2$, so both are in the span of 
$\{v_1;v_2\}$.  Indeed, every vector
in $\R^2$ will be somewhere on this grid, and therefore will be in $\mathrm{Span}\{v_1;v_2\}$.  Therefore this list
spans $\R^2$.  
}


\endedxproblem

\beginedxproblem{Spanning Question 2}{\dpa1}


Does every list of two non-zero vectors in $\R^2$ span $\R^2$?  

\edXabox{expect="No" options="Yes","No"}

\edXsolution{ 
If the two vectors are scalar multiples of each other, then they lie along the same line, and their span cannot
include anything off that line.  For instance, in the following picture, $w$ is not a linear combination of
$v_1$ and $v_2$.  

\begin{center}
\includesvg[300]{c1s4lincombo2}   
\end{center}

}


\endedxproblem




\beginedxproblem{Does the list span?}{\dpa3}


\begin{itemize}
\item
List A: $\left\{ \left[ \begin{array}{c} 1 \\ 7 \end{array} \right] ; 
\left[ \begin{array}{c} 2 \\ 14  \end{array} \right] ; 
\left[ \begin{array}{c} -3 \\ -21  \end{array} \right] \right\} $

\item
List B: $\left\{  \left[ \begin{array}{c} 1 \\ 1 \\ 1 \end{array} \right] ; 
\left[ \begin{array}{c} 1 \\ 1 \\ 0 \end{array} \right] ;
\left[ \begin{array}{c} 1 \\ 0 \\ 1 \end{array} \right]  \right\}
$

\item
List C: $\left\{  \left[ \begin{array}{c} 1 \\ 2 \\ 3 \end{array} \right] ; 
\left[ \begin{array}{c} 0 \\ 0 \\ 0 \end{array} \right] ;
\left[ \begin{array}{c} 2 \\ 5 \\ 7 \end{array} \right] ;
\left[ \begin{array}{c} 2 \\ 1 \\ 0 \end{array} \right] \right\} $

\item
List D: $\left\{ \left[ \begin{array}{c} 1 \\ 2 \\ 3 \\ 4\end{array} \right] ; 
\left[ \begin{array}{c} 2 \\ 3 \\ 5 \\ 7\end{array} \right] ; 
\left[ \begin{array}{c} 12 \\ 23 \\ 35 \\ 58\end{array} \right] \right\} $


\item
List E: $\left\{ \left[ \begin{array}{c} 1 \\ 2 \\ 0 \\ 4\end{array} \right] ; 
\left[ \begin{array}{c} 2 \\ 13 \\ 0 \\ 7\end{array} \right] ; 
\left[ \begin{array}{c} 0 \\ 1 \\ 0 \\ -3\end{array} \right] ; 
\left[ \begin{array}{c} 15 \\ -2 \\ 0 \\ 2\end{array} \right] ; 
\left[ \begin{array}{c} 5 \\ 3 \\ 0 \\ 11\end{array} \right]  \right\}$



\item
List F: $\left\{ \left[ \begin{array}{c} 1 \\ 3 \\ 8 \end{array} \right] ; 
\left[ \begin{array}{c} 2 \\ 6 \\ 16 \end{array} \right] ;
\left[ \begin{array}{c} 1 \\ 2 \\ 8 \end{array} \right] ; 
\left[ \begin{array}{c} 0 \\ 1 \\ 8 \end{array} \right] \right\} $
\end{itemize}

Given above are six lists of vectors.  {\keyb{\bf Without doing any row reduction}}, you should be able to pick out three lists
that do not span their respective $\R^m$ spaces.  Which three?  



\edXabox{type="oldmultichoice" expect="A","D","E" options="A","B","C","D","E","F"}

\edXsolution{
List A has three vectors that are scalar multiples of each other; they all lie on the same line. Thus any
linear combination of them also lies on the line, so they cannot span all of $\R^2$.\\
List D contains three vectors.  Since these vectors are in $\R^4$, there aren't enough of them to span the whole space.\\
List E has five vectors in $\R^4$. This may seem promising, but notice that all five of them have zero as the third entry. Any vector in $\R^4$ with a non-zero third entry cannot be expressed as a linear combination of these vectors, so they don't span all of $\R^4$.

}

\endedxproblem



\endedxvertical
