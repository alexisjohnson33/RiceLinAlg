

\beginedxvertical{Page One}

\beginedxtext{Preliminaries}


For time budgeting purposes, this sequence has 4 videos totaling 14 minutes, 
plus some questions.  

% Remember, when you're doing the online learning sequences, you may seek help if you 
% do not understand a video, but you should think about all of the questions 
% entirely individually.  You have pledged to do so under the Honor Code!  


\endedxtext

\endedxvertical

\beginedxvertical{Introduction}



\doedxvideo{Functions}{???}


\beginedxproblem{System Consistency?}{\dpa1}

Define the function $T$ by $T(\vecx) = A\vecx$, where $A = \left[ \begin{array}{ccc}
6 & 5 & 3 \\
2 & 4 &  23 
\end{array} \right].$

What is the domain of $T$?

\edXabox{type="multichoice" expect="R^3" options="R","R^2","R^3"}

What is the codomain of $T$?

\edXabox{type="multichoice" expect="R^2" options="R","R^2","R^3"}

\edXsolution{ 

}

\endedxproblem



\endedxvertical





\beginedxvertical{Defining Linear Transformations}

\doedxvideo{Linear Transformations}{???}
%example: rotations, non-example: something quadratic


\endedxvertical









\beginedxvertical{Linear Transformation Definition}

\beginedxtext{Definition of Linear Transformation}

Let $\F$ be a field.  A function $T: \F^n \rightarrow \F^m$ is a {\keyb{\bf linear transformation}}
if both of the following properties hold:

\begin{itemize}
\item For any vectors $\vecv, \vecw$ in the domain, $T(\vecv + \vecw) = T(\vecv) + T(\vecw).$ 
\item For any vector $\vecv$ in the domain, and any scalar $a\in \F$, $T(a\vecv) = aT(\vecv)$.    
\end{itemize}

Informally, if $T$ satisfies the first condition we say that ``$T$ respects vector addition", and 
if $T$ satisfies the second condition we say that ``$T$ respects scalar multiplication."  

\endedxtext

\endedxvertical


\beginedxvertical{Some Examples and Non-examples}



\beginedxproblem{Linear Transformation? 1}{\dpa1}

Define $T: \R^3 \rightarrow \R^3$ by $T(\vecv) = 2\vecv$.  

Does $T$ respect vector addition?  That is, is $T(\vecv + \vecw) = T(\vecv) + T(\vecw)$ for all 
vectors $\vecv,\vecw$?
Try a couple of specific numerical examples if you're not sure.  

\edXabox{expect="Yes" options="Yes","No"}

Does $T$ respect scalar multiplication?  
That is, is $T(a\vecv) = aT(\vecv)$ for all vectors $\vecv$ and scalars $a$?
Try a couple of specific numerical examples if you're not sure.  


\edXabox{expect="Yes" options="Yes","No"}

Is $T$ a linear transformation?

\edXabox{expect="Yes" options="Yes","No"}

\edXsolution{ 
}

\endedxproblem


\beginedxproblem{Linear Transformation? 2}{\dpa1}

Define $T: \R^2 \rightarrow \R^2$ by $T(\vecv) = \veco$.  

Does $T$ respect vector addition?  That is, is $T(\vecv + \vecw) = T(\vecv) + T(\vecw)$ for all 
vectors $\vecv,\vecw$?
Try a couple of specific numerical examples if you're not sure.  

\edXabox{expect="Yes" options="Yes","No"}

Does $T$ respect scalar multiplication?  
That is, is $T(a\vecv) = aT(\vecv)$ for all vectors $\vecv$ and scalars $a$?
Try a couple of specific numerical examples if you're not sure.  


\edXabox{expect="Yes" options="Yes","No"}

Is $T$ a linear transformation?

\edXabox{expect="Yes" options="Yes","No"}

\edXsolution{ 
}

\endedxproblem

\beginedxproblem{Linear Transformation? 3}{\dpa1}

Define $T: \R^2 \rightarrow \R^2$ by $T(\vecv) = \left[\begin{array}{c}
4 \\
0 
\end{array} \right]$.  

Does $T$ respect vector addition?  That is, is $T(\vecv + \vecw) = T(\vecv) + T(\vecw)$ for all 
vectors $\vecv,\vecw$?
Try a couple of specific numerical examples if you're not sure.  

\edXabox{expect="No" options="Yes","No"}

Does $T$ respect scalar multiplication?  
That is, is $T(a\vecv) = aT(\vecv)$ for all vectors $\vecv$ and scalars $a$?
Try a couple of specific numerical examples if you're not sure.  


\edXabox{expect="No" options="Yes","No"}

Is $T$ a linear transformation?

\edXabox{expect="No" options="Yes","No"}

\edXsolution{ 
}

\endedxproblem


\endedxvertical


\beginedxvertical{More Linear Transformations}

\doedxvideo{More Examples of Linear Transformations}{???}

%matrix multiplication; T(0) = 0

\endedxvertical


