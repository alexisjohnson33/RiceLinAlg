
\beginedxproblem{Dropdown Problem}{\dpa1}

This is a dropdown menu multiple choice.  Test test


\edXabox{expect="Yes" options="Yes","Not necessarily","Whatever dude"}

\edXsolution{ 
}

\endedxproblem




\beginedxproblem{Multiple Choice}{\dpa1}

This is a multiple choice problem with radio buttons, one right answer.  If the answers are 
long or math-heavy this is better than a dropdown menu.  

\edXabox{type="multichoice" expect="$4\times 3$" options="$3\times 4$","$4\times 3$"}

\edXsolution{ 
}

\endedxproblem


\beginedxproblem{Multiple Answer Multiple Choice}{\dpa1}

This is a multiple choice question with checkboxes 
where students can select more than one answer.  You should tell them
that they should select all of the choices that fit the criteria.  

\edXabox{type="oldmultichoice" expect="1","3" options="1","2","3","4"}

\edXsolution{ 
}

\endedxproblem


\beginedxproblem{Numerical question}{\dpa1}

This is a numerical question.  

What is the maximum number of pivots in a $4\times 6$ coefficient  matrix, when row-reduced?  

% Tolerance is how far off the student can be and still get it correct.  You can delete it 
% (ie just have \edXabox{type="numerical" expect="4"}  )
% if the answer is obviously going to be an integer, but there's
% also not really any harm in keeping it in.  You can also have things like expect="4*pi", in which case some
% students will enter as a decimal and we definitely want to have a fudge factor, typically 0.01.  

\edXabox{type="numerical" expect="4" tolerance="0.01"}


\edXsolution{ 
}

\endedxproblem



\beginedxproblem{Formula Question}{\dpa5}

This is a formula question.  The samples bit tells EdX what variables to expect (in this case $m$ and $d$),
what ranges to take sample values from (m from 1 to 4, d from 1 to 2) and how many samples to take (10).
Tolerance is how far off the student formula's output is allowed to get.  I'm not entirely sure
what the feqin does, but keep it in.  

\edXabox{type="formula" expect="m^3/d" samples="m,d@1,1:4,2#10" tolerance=".01" feqin="1"}

Note, btw, that you can have two boxes in a single problem!

\edXabox{type="numerical" expect="-9000*pi" tolerance="1" feqin="1"}

\edXsolution{ 

}

\endedxproblem


\beginedxproblem{Matrix entry}{\dpa1}

This is a matrix entry problem.


Always have the following line somewhere:

To enter the matrix $\left[ \begin{array}{cc:c}
0&1&2 \\
1&2&3 \end{array} \right],$ type [[0,1,2],[1,2,3]]   

Use decimals only.  

\begin{edXscript}
def MatrixEntry(expect, ans):
  	import ast
	import numpy as np 
	ret= {'ok':False}
  	atol = 0.01
  	try:
		list_ans = ast.literal_eval(ans)
		list_expect = ast.literal_eval(expect)
  		matrix_ans = np.matrix(list_ans)
  		matrix_expect = np.matrix(list_expect) 
  		if matrix_ans.shape != matrix_expect.shape:
  			ret['msg'] = 'Wrong shape of matrix'
  		elif np.allclose(matrix_ans, matrix_expect,0.01,1e-08):
  			ret['ok'] = True
  		else:
  			ret['msg'] = 'Something is wrong'
	except SyntaxError:
		ret['msg'] = 'Wrong input format'
  	return ret
\end{edXscript}



Until Wei gives us the grader, just have the following:

\edXabox{expect="Placeholder" options="Placeholder"}

%After we get the grader, you can have a line like the following:

%\edXabox{type="custom" cfn="MatrixEntry" expect="[[1,2,3],[4,5,6]]"}


\edXsolution{ 
}


\endedxproblem


\beginedxproblem{Vector entry}{\dpa1}

This is a vector entry problem.


Always have the following line somewhere:

To enter a vector such as $\left[\begin{array}{c} 1 \\ 2  \\ 3 \end{array} \right]$, you can either:
\begin{itemize}
\item
enter it as 
you would a $3\times 1$ matrix: [[1],[2],[3]]  
\item
or, you can enter it as <1,2,3>
\end{itemize}

Use decimals only.  

\begin{edXscript}
def VectorEntry(expect, ans):
	import ast
	import numpy as np 
  	atol = 0.01
	list_expect = ast.literal_eval(expect)
	vec_expect = np.matrix(list_expect)
  	ret = {"ok":False}
	try:
  		# input format [[1],[2],[3]]
		list_ans = ast.literal_eval(ans)
		vec_ans = np.matrix(list_ans)
  		if vec_ans.shape != vec_expect.shape:
  			ret['msg'] = 'Wrong shape of vector!'
  		elif np.allclose(vec_ans, vec_expect,atol,1e-08):
  			ret['ok'] = True
  		else:
  		# More error message. Will improve this part
  			ret['msg'] = 'something is wrong'   			
	#except SyntaxError:
		#ret['msg'] = 'Wrong input format'
	except SyntaxError:
  		# input format &lt;1,2,3&gt;
		list_ans = ans.replace('&lt;', '[').replace('&gt;', ']')
		list_ans = ast.literal_eval(list_ans)
		vec_ans = np.transpose(np.matrix(list_ans))
  		if vec_ans.shape != vec_expect.shape:
  			ret['msg'] = 'Wrong shape of vector!'
  		elif np.allclose(vec_ans, vec_expect,0.01,1e-08):
  			ret['ok'] = True
  		else:
    		# More error message. Will improve this part
  			ret['msg'] = 'something is wrong' 
  	except:
  		ret['msg'] = 'Wrong input format'
  	return ret 
\end{edXscript}



Until Wei gives us the grader, just have the following:

\edXabox{expect="Placeholder" options="Placeholder"}

%After we get the grader, you can have a line like the following:

%\edXabox{type="custom" cfn="VectorEntry" expect="[[1],[2],[3]]"}


\edXsolution{ 
}


\endedxproblem
