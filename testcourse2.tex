\documentclass[12pt]{article}

\usepackage{edXpsl}	% edX
\usepackage{amsmath, amsthm, amsfonts, amssymb, color, mathrsfs, comment}
\usepackage{cancel}

%------------------------------------------
\parindent=0pt
\parskip=1ex

\begin{document}
%-----------------------------------------------%

%\documentclass[12pt]{article}
%
%\usepackage{edXpsl}	% edX
%\usepackage{amsmath, amsthm, amsfonts, amssymb, color, mathrsfs, comment}
%\usepackage{cancel}
%
%%------------------------------------------
%\parindent=0pt
%\parskip=1ex
%
%\begin{document}
%%-----------------------------------------------%

\newcounter{edXtext}
\newcounter{edXproblem}
\newcounter{edXvideo}
\newcounter{showhide}
\newcounter{psetproblem}
%\newcounter{edXvertical}


%If seems \ifplastex needs to go after the \begin{document}
% \newif\ifplastex 
% \plastexfalse
% \ifplastex
  %% Many of these could be better handled in python. When I get the chance to understand plastex better I'll do that. JMO
   % \def\USING{USING PLASTEX FIX MACROS}
   \def\cancel#1{#1}
%   \def\footnote#1{} %\par {\small{\color{red} FOOTNOTE:} #1}}
   \def\mylabel#1{\label{#1}\tag{\value{equation}}}

   \def\displaynametag{display_name}

   %\def\includesvg{\includegraphics}
   %\def\includegraphics{\edXincludegraphics}

   \newcounter{mynumbereditem}
   \newcounter{temp}
   \def\mysecnum{\arabic{section}}
   \def\mynumbereditemnum{\arabic{mynumbereditem}}
   \def\mynumbereditemlabel{\mysecnum.\mynumbereditemnum}

   \def\mysection#1{\section{#1}}

   \newenvironment{theorem}{\addtocounter{mynumbereditem}{1}%
\textbf{Theorem \mynumbereditemlabel.} }{\par\vspace{.5in}}
   \newenvironment{proposition}{\addtocounter{mynumbereditem}{1}%
\textbf{Proposition \mynumbereditemlabel.} }{\par\vspace{.5in}}
   \newenvironment{lemma}{\addtocounter{mynumbereditem}{1}%
\textbf{Lemma \mynumbereditemlabel.} }{\par\vspace{.5in}}
   \newenvironment{definition}{\addtocounter{mynumbereditem}{1}%
\textbf{Definition \mynumbereditemlabel.} }{\par\vspace{.5in}}
   \newenvironment{remarks}{\addtocounter{mynumbereditem}{1}%
\textbf{Remarks \mynumbereditemlabel.} }{\par\vspace{.5in}}
   \newenvironment{corollary}{\addtocounter{mynumbereditem}{1}%
\textbf{Corollary \mynumbereditemlabel.} }{\par\vspace{.5in}}
   \newenvironment{remark}{\addtocounter{mynumbereditem}{1}%
\textbf{Remark \mynumbereditemlabel.} }{\par\vspace{.5in}}
   \newenvironment{examples}{\addtocounter{mynumbereditem}{1}%
\textbf{Examples \mynumbereditemlabel.} }{\par\vspace{.5in}}
   \newenvironment{example}{\addtocounter{mynumbereditem}{1}%
\textbf{Example \mynumbereditemlabel.} }{\par\vspace{.5in}}
   \newenvironment{exercise}{\addtocounter{mynumbereditem}{1}%
\textbf{Exercise \mynumbereditemlabel.} }{\par\vspace{.5in}}

   \newenvironment{stheorem}{\textbf{Theorem. }}{\par\vspace{.5in}}
   \newenvironment{sproposition}{\textbf{Proposition. }}{\par\vspace{.5in}}
   \newenvironment{slemma}{\textbf{Lemma. }}{\par\vspace{.5in}}
   \newenvironment{sdefinition}{\textbf{Definition. }}{\par\vspace{.5in}}
   \newenvironment{sremarks}{\textbf{Remarks. }}{\par\vspace{.5in}}
   \newenvironment{scorollary}{\textbf{Corollary. }}{\par\vspace{.5in}}
   \newenvironment{sremark}{\textbf{Remark. }}{\par\vspace{.5in}}
   \newenvironment{sexamples}{\textbf{Examples. }}{\par\vspace{.5in}}
   \newenvironment{sexample}{\textbf{Example. }}{\par\vspace{.5in}}
   \newenvironment{sexercise}{\textbf{Exercise. }}{\par\vspace{.5in}}


% \def\dpa{weight="1" showanswer="attempted" attempts="3"}

\def\beginedxsequential#1#2{\begin{edXsection}{#1}[#2 url_name="\edxbaseoutputname-sequential"]}

\def\endedxsequential{\end{edXsection} \setcounter{psetproblem}{0}}


\def\beginedxtext#1{\refstepcounter{edXtext}\begin{edXtext}{#1}[url_name="\edxbaseoutputname-tab\theedXvertical-text\theedXtext"]}
\def\endedxtext{\end{edXtext}}
\def\beginedxproblem#1#2{\refstepcounter{edXproblem}\begin{edXproblem}{#1}{url_name="\edxbaseoutputname-tab\theedXvertical-problem\theedXproblem" #2}}
\def\endedxproblem{\end{edXproblem}}

%generate pset problem names
\def\beginedxpset#1#2{\refstepcounter{edXproblem}\refstepcounter{psetproblem}\begin{edXproblem}{#1 (\thepsetproblem)}{url_name="\edxbaseoutputname-tab\theedXvertical-problem\theedXproblem" #2}}
\def\endedxpset{\end{edXproblem}}

\def\beginedxvertical#1{\begin{edXvertical}{#1}[url_name="\edxbaseoutputname-vertical\theedXvertical"]}
\def\endedxvertical{\end{edXvertical} \setcounter{edXtext}{0} \setcounter{edXproblem}{0} \setcounter{edXvideo}{0}}


%New-allow for source command
\providecommand{\doedxvideo}[3][]{\refstepcounter{edXvideo}\edXvideo{#2}{#3}[url_name="\edxbaseoutputname-tab\theedXvertical-video\theedXvideo" #1]}

%New wrapper.
\newenvironment{shh}{}{}
\def\beginedxshowhide#1{\begin{shh}\begin{edXshowhide}{#1}}  
\def\endedxshowhide{\end{edXshowhide}\end{shh}}


% EVH added
% \def\edXmathlet#1{\edXxml{<iframe src="https://s3.amazonaws.com/1801-static-assets/build/#1.html" width="820 px" height="630 px" style="border:0px"/>}}

%JEF added
% \providecommand{\includesvg}[2][400]{\edXxml{<img src="images/#2.svg" width="#1 px" style="margin: 10px 25px 25px 25px; border:0px"/>}}
\providecommand{\includesvg}[2][400]{\edXxml{<img src="/static/images/#2.svg" width="#1 px" style="margin: 10px 25px 25px 25px; border:0px"/>}}


%-------------------------------------------------%

\def\myheader#1{\noindent \textbf{#1}\par}

\def\ds{\displaystyle}
\def\Re{\mathrm{Re\,}}
\def\Im{\mathrm{Im\,}}

\newcommand{\R}{\mathbb R}
\newcommand{\Z}{\mathbb Z}
\newcommand{\C}{\mathbb C}
\newcommand{\Q}{\mathbb Q}
\newcommand{\inv}{^{-1}}
\newcommand{\eps}{\epsilon}
\newcommand{\veca}{\mathbf{a}}
\newcommand{\vecb}{\mathbf{b}}
\newcommand{\vecc}{\mathbf{c}}
\newcommand{\vecw}{\mathbf{w}}
\newcommand{\vecv}{\mathbf{v}}
\newcommand{\vecu}{\mathbf{u}}
\newcommand{\vecd}{\mathbf{d}}
\newcommand{\vece}{\mathbf{e}}
\newcommand{\vecx}{\mathbf{x}}
\newcommand{\veco}{\mathbf{0}}
\newcommand{\vecy}{\mathbf{y}}
\newcommand{\F}{{\bf {F}}}

%commands for Mattuck/Jerison Problems
% \def\qw{\qquad}
% \def\q{\quad}
% \def\f{\frac}
% \def\disp{\displaystyle}
% \def\To{\implies}
% \def\e{\epsilon}
% \def\t{\theta}
% \def\D{\Delta}

%These two are needed for content from David Jerison's notes
% \def\Implies{\ \implies \ }
% \def\Iff{\ \iff \ }

\def\imgdir{images}
%_________________________________



%\def\defaultproblemattributes{attempts="5" showanswer="attempted" rerandomization="per_student"}


% \def\qw{\qquad}
% \def\q{\quad}
% \def\fig{\includegraphics}

% \def\inv{^{-1}}

% \def\dpaA{showanswer="finished" attempts="1" rerandomize="per_student"}
% \def\dpaB{showanswer="finished" attempts="3" rerandomize="per_student"}
% \def\dpaC{showanswer="finished" attempts="5" rerandomize="per_student"}
% \def\dpaD{showanswer="finished" attempts="7" rerandomize="per_student"}
% \def\dpa#1{showanswer="finished" attempts="#1" rerandomize="per_student"}
\def\dpa#1{showanswer="finished" attempts="#1" rerandomize="per_student"}
% \def\dpwa[1]{showanswer="finished" weight="#1" attempts="1" rerandomize="per_student"}
% \def\dpwb[1]{showanswer="finished" weight="#1" attempts="2" rerandomize="per_student"}
% \def\dpwc[1]{showanswer="finished" weight="#1" attempts="3" rerandomize="per_student"}
% \def\dpwd[1]{showanswer="finished" weight="#1" attempts="4"  rerandomize="per_student"}
% \def\dpwe[1]{showanswer="finished" weight="#1" attempts="5" rerandomize="per_student"}
\def\dpaZ{attempts="100" rerandomize="onreset" showanswer="attempted"}
\def\dpadnd{showanswer="attempted" attempts="3" rerandomize="onreset"}
\def\resetdpa#1{showanswer="attempted" attempts="#1" rerandomize="onreset"}


%color for 18.01x
%added by Jen
\definecolor{blue}{cmyk}{1,1,0,0}
\definecolor{orange}{cmyk}{0,0.5,1,0}

\definecolor{bordeaux}{cmyk}{0,.84,.71,.40}
\newcommand{\keya}{\color{bordeaux}}

\definecolor{royalblue}{cmyk}{.72,.54, 0, .45}
% \newcommand{\keyb}{\color{royalblue}}
\newcommand{\keyb}{\color{bordeaux}}


\protected\def\blue#1{%
  \ifmmode
  	{\color{blue}{#1}}
  \else
  	\textbf{{\color{blue}{#1}}}
   \fi
}
\protected\def\red#1{%
  \ifmmode
  	{\color{orange}{#1}}
  \else
  	\textbf{{\color{orange}{#1}}}
   \fi
}



 %custom macros

\def\defaultproblemattributes{attempts="1" showanswer="attempted" rerandomize="per_student"}

% edXcourse: {course_number}{course display_name}[optional arguments like semester]
\begin{edXcourse}{Test}{Test LinAlg}[semester="2018_Summer" info_sidebar_name="Other Documents" start="2018-01-11T12:00" end="2018-12-18T18:00" course_image="rice-logo.jpg" display_coursenumber="TestLinAlg" course_organization="TestRice" graceperiod="1800 seconds" invitation_only="true" allow_anonymous="false" mobile_available="true"  org="Rice"]
 
% \begin{edXchapter}{Getting started}[url_name="cyca" start="2015-08-23T18:00"]
%  

% \def\edxbaseoutputname{edxtutorial}

% \input{overview/edxTutorial.tex}
 

% \endedxsequential
 
% \end{edXchapter} 





\begin{edXchapter}{Vectors and Matrices}[url_name="block1" start="2018-01-11T16:00"]



\def\edxbaseoutputname{b2matrixops}



\beginedxsequential{More Matrix Operations}{due="2018-12-13T14:15" graded="true" format="Exercises"}






\beginedxvertical{Page One}

\beginedxtext{Preliminaries}





At the end of this sequence, and after some practice, you should be able to:

\begin{itemize}
\item Understand the relationship between elementary matrices and row operations.  
\item Find the transpose of a matrix.  
\item Recognize more conditions under which a matrix is invertible.   
\end{itemize}


For time budgeting purposes, this sequence has X videos totaling X minutes, 
plus some questions.  




\endedxtext

\endedxvertical



\beginedxvertical{Inverses and Products}


\doedxvideo{Inverses and Products}{KdbmhHwABYo}

\beginedxtext{Inverses and Products}

{\keya{\bf{Proposition.}}}  
If $A$ and $B$ are $n\times n$ matrices which are both invertible, then $AB$ is invertible, and 
$(AB)\inv = B\inv A\inv$. 


{\keya{\bf{Proposition.}}}  If $A$ and $B$ are $n\times n$ matrices, and their product $AB$ is invertible, then
both $A$ and $B$ are invertible.  

\endedxtext




\endedxvertical


\beginedxvertical{Inverse Questions}




\beginedxproblem{Invertible Product}{\dpa1}

True or false: It is possible for two non-invertible matrices to have an invertible product.  

\edXabox{expect="True" options="True","False"}

\edXsolution{ The proposition proven on the previous pages states that if the product of two {\keyb{\bf{square}}}
matrices is invertible, then each individual matrix is.  However, this problem does not indicate that the matrices
are square.  

For instance, given 
\[A = \left[ \begin{array}{ccc} 1 & 2 & 0  \\
0 & 1 & 1
 \end{array} \right], \  \mathrm{and} \ B = \left[ \begin{array}{cc} 1 & 0  \\
0 & 0 \\
0 & 1
 \end{array} \right], \]
we find that $AB$ is the $2\times 2$ identity matrix, which is invertible, yet neither $A$ nor $B$ are invertible.

 }

\endedxproblem

\beginedxproblem{Solve for A}{\dpa1}

Suppose that $A,B,C,D$ are all invertible $n\times n$ matrices such that
$DA\inv B\inv = C\inv$.  Solve for $A$.  


\edXabox{type="multichoice" expect="$A=B\inv CD$" options="$A=DCB\inv$","$A=DB\inv C$","$A=B\inv CD$","$A=BCD\inv$","$A=D\inv CB$","$A=CB\inv D$","None of the above"}

\edXsolution{ 
 }
 
\endedxproblem


\endedxvertical

\beginedxvertical{Introducing Elementary Matrices}


\beginedxproblem{Find a matrix}{\dpa3}

Find a matrix $E$ such that, for any matrix
$A =   \left[ \begin{array}{cc} a_{11} & a_{12}  \\
a_{21} & a_{22}  \\ 
a_{31} & a_{32}  
 \end{array} \right],$ we get

\[EA = \left[ \begin{array}{cc} a_{11} & a_{12}  \\
a_{21} & a_{22}  \\ 
4a_{31} & 4a_{32}  
 \end{array} \right].\]


To enter the matrix $\left[ \begin{array}{cc:c}
0&1&2 \\
1&2&3 \end{array} \right],$ type [[0,1,2],[1,2,3]]   

Use decimals only.  

\begin{edXscript}
def MatrixEntry(expect, ans):
  	import ast
	import numpy as np 
	ret= {'ok':False}
  	atol = 0.01
  	try:
		list_ans = ast.literal_eval(ans)
		list_expect = ast.literal_eval(expect)
  		matrix_ans = np.matrix(list_ans)
  		matrix_expect = np.matrix(list_expect) 
  		if matrix_ans.shape != matrix_expect.shape:
  			ret['msg'] = 'Wrong shape of matrix'
  		elif np.allclose(matrix_ans, matrix_expect,0.01,1e-08):
  			ret['ok'] = True
  		else:
  			ret['msg'] = 'Something is wrong'
	except SyntaxError:
		ret['msg'] = 'Wrong input format'
  	return ret
\end{edXscript}


\edXabox{type="custom" cfn="MatrixEntry" expect="[[1,0,0],[0,1,0],[0,0,4]]"}

\edXsolution{

}

\endedxproblem


\doedxvideo{Introducing Elementary Matrices}{20gbxsXOooY}


\beginedxproblem{Find another matrix}{\dpa3}

Find a matrix $E$ such that, for any matrix
$A =   \left[ \begin{array}{cc} a_{11} & a_{12}  \\
a_{21} & a_{22}  \\ 
a_{31} & a_{32}  
 \end{array} \right],$ we get

\[EA = \left[ \begin{array}{cc} a_{21} & a_{22}  \\
a_{11} & a_{12}  \\ 
a_{31} & a_{32}  
 \end{array} \right].\]


\begin{edXscript}
def MatrixEntry(expect, ans):
  	import ast
	import numpy as np 
	ret= {'ok':False}
  	atol = 0.01
  	try:
		list_ans = ast.literal_eval(ans)
		list_expect = ast.literal_eval(expect)
  		matrix_ans = np.matrix(list_ans)
  		matrix_expect = np.matrix(list_expect) 
  		if matrix_ans.shape != matrix_expect.shape:
  			ret['msg'] = 'Wrong shape of matrix'
  		elif np.allclose(matrix_ans, matrix_expect,0.01,1e-08):
  			ret['ok'] = True
  		else:
  			ret['msg'] = 'Something is wrong'
	except SyntaxError:
		ret['msg'] = 'Wrong input format'
  	return ret
\end{edXscript}


\edXabox{type="custom" cfn="MatrixEntry" expect="[[0,1,0],[1,0,0],[0,0,1]]"}

\edXsolution{

}

\endedxproblem

\endedxvertical

\beginedxvertical{Elementary Matrices}



\beginedxtext{Elementary Matrices}

{\keya{\bf{Definition.}}}  An {\keyb{\bf{elementary matrix}}} is a square matrix of one of three types.  


\begin{edXshowhide}{Elementary Matrix of the First Type}
An elementary matrix of the first type is one obtained from the identity matrix by changing the diagonal entry in the 
$i$th row and column to a number 
$c$, where $c \ne 0$:  
\[ E_1 = \left[ 
\begin{array}{ccccccc}
1 &  &  &  &  &  &  \\
 & \ddots &  &  &  &  &  \\
 &   & 1& &  &  &  \\
 &   &  &  c &  &  &  \\
 &   &  &  & 1 &  &  \\
 &   &  &  &  & \ddots  &  \\
 &   &  &  &  &   & 1 
\end{array}
\right].  \]
Multiplying  a matrix $A$ by $E_1$ (where $E_1$ is on the left) has the same effect as scaling row $i$ of $A$ by the scalar $c$.  
\end{edXshowhide}

\begin{edXshowhide}{Elementary Matrix of the Second Type}
An elementary matrix of the second type is one obtained from the identity matrix by changing the non-diagonal entry in the $i$th row and $j$th column ($i\ne j$) to a number $c \ne 0$: 
\[ E_2 = \left[ 
\begin{array}{ccccccc}
1 &  &  &  &  &  &  \\
 & \ddots &  &  &  &  &  \\
 &   & 1& &  &  &  \\
 &   &  &  \ddots &  &  &  \\
 &   &  c &  & 1 &  &  \\
 &   &  &  &  & \ddots  &  \\
 &   &  &  &  &   & 1 
\end{array}
\right].  \]
Multiplying  a matrix $A$ by $E_2$ (where $E_2$ is on the left) has the same effect as adding $c$ times the $j$th row of $A$ to row $i$.  
\end{edXshowhide}

\begin{edXshowhide}{Elementary Matrix of the Third Type}
An elementary matrix of the third type is one obtained from the identity matrix by changing the $i$th and $j$th diagonal entries to 0, and changing the entries in the $i$th row and $j$th column, and the $j$th row and $i$th column, to 1: 
\[ E_3 = \left[ 
\begin{array}{ccccccccccc}
1 &  &  &  &  &  &  & & & &\\
 & \ddots &  &  &  &  &  & & & &\\
 &   & 1& &  &  &  & & & & \\
 &   &  &  0 &  &  &  & 1 & & & \\
 &   &   &  & 1 &  &  & & & &\\
 &   &  &  &  & \ddots  & & & & & \\
 &   &  &  &  &   & 1 & & & & \\
&   &  &  1 &  &   &  & 0 & & &  \\ 
&   &  &  &  &   &  &  & 1 & & \\
&   &  &  &  &   &  &  &  & \ddots & \\ 
&   &  &  &  &   &  &  &  & & 1
 \end{array}
\right].  \]
Multiplying  a matrix $A$ by $E_3$ (where $E_3$ is on the left) has the same effect as swapping the $i$th and $j$th rows
of $A$.  
\end{edXshowhide}




\endedxtext

\endedxvertical

\beginedxvertical{Elementary Matrices and Inverses}


\doedxvideo{Elementary Matrices and Inverses}{EtoZ2FyFKS4}


\endedxvertical

\beginedxvertical{Transposes}

\beginedxtext{Transposes}

The last matrix operation we will introduce in this section is the transpose.  The transpose of a matrix $A$
is essentially the matrix whose columns are the rows of $A$ (and whose rows are the columns of $A$).  Essentially,
$A$ gets flipped across its diagonal.  More formally:

{\keya{\bf{Definition.}}}  Given an $m\times n$ matrix $A$, the {\keyb{\bf{transpose}}} of $A$ is the
$n\times m$ matrix denoted $A^t$ whose entry in the $i$th row and $j$th column is the same as the
entry in the $j$th row and $i$th column of $A$.  

For instance, if \[A = \left[ \begin{array}{ccc} 1 & 2 & 0  \\
3 & 4 & -1
 \end{array} \right],\] then \[A^t  = \left[ \begin{array}{cc} 1 & 3  \\
2 & 4 \\ 0 & -1
 \end{array} \right].\]
 
\endedxtext


\beginedxproblem{Find a Transpose}{\dpa3}

Suppose $B = \left[ \begin{array}{cc} 2 & 0  \\
1 & 3 \\ -2 & -1 \\ 5 & 0
 \end{array} \right].$  
 
What is $B^t$?  

\begin{edXscript}
def MatrixEntry(expect, ans):
  	import ast
	import numpy as np 
	ret= {'ok':False}
  	atol = 0.01
  	try:
		list_ans = ast.literal_eval(ans)
		list_expect = ast.literal_eval(expect)
  		matrix_ans = np.matrix(list_ans)
  		matrix_expect = np.matrix(list_expect) 
  		if matrix_ans.shape != matrix_expect.shape:
  			ret['msg'] = 'Wrong shape of matrix'
  		elif np.allclose(matrix_ans, matrix_expect,0.01,1e-08):
  			ret['ok'] = True
  		else:
  			ret['msg'] = 'Something is wrong'
	except SyntaxError:
		ret['msg'] = 'Wrong input format'
  	return ret
\end{edXscript}


\edXabox{type="custom" cfn="MatrixEntry" expect="[[2, 1, -2, 5],[0,3,-1,0]]"}

\edXsolution{The first column of $B$ becomes the first row of $B^t$.  The second column of $B$
becomes the second row of $B^t$.  
}

\endedxproblem



\beginedxproblem{Double Transpose}{\dpa1}

True or false: Any matrix $A$ is equal to the transpose of its transpose.  In other words, $A = (A^t)^t$.  

\edXabox{expect="True" options="True","False"}

\edXsolution{ Switching the rows and columns of the transpose gives the original matrix.  
 }

\endedxproblem


\endedxvertical

\beginedxvertical{Transposes and Products}

\beginedxproblem{Possible Product}{\dpa1}

Suppose $A$ is a $4\times 3$ matrix, and $B$ is an $3 \times 5$ matrix, so that $AB$ is a $4\times 5$ matrix.  

Which of the following matrix products is defined?  

\edXabox{type="multichoice" expect="$B^tA^t$" options="$A^tB^t$","$B^tA^t$","Both are defined","Neither are defined"}

\edXsolution{ 
 }

\endedxproblem



\doedxvideo{Transposes and Products}{AT6y-Q3TI_w}

\endedxvertical

\beginedxvertical{Transposes and Inverses}

\beginedxproblem{Symmetric Matrices}{\dpa1}

{\keya{\bf{Definition.}}} A matrix $A$ is {\keyb{\bf{symmetric}}} if $A = A^t$.  

Of the following matrices, which ones are symmetric?  Click all that apply.  

\edXabox{type="oldmultichoice" expect="$\left[ \begin{array}{cc} \pi & 1   \\ 1 & 4 \end{array} \right]$","$I_n$" options="$\left[ \begin{array}{ccc} 1 & 0 & 0  \\ 0 & 2 & 0 \end{array} \right]$","$\left[ \begin{array}{cc} 1 & 1   \\ 2 & 2 \end{array} \right]$","$\left[ \begin{array}{cc} \pi & 1   \\ 1 & 4 \end{array} \right]$","$I_n$"}

\edXsolution{ 
 }

\endedxproblem

\doedxvideo{Transposes and Inverses}{bg1nsJJS-ZY}

\endedxvertical

\beginedxvertical{Summary of Results}



\beginedxtext{Transpose Results Summary}

{\keya{\bf{Proposition.}}}  If $A$ and $B$ are matrices such that the product $AB$ is defined, then 
$(AB)^t = B^t A^t$.  

{\keya{\bf{Proposition.}}}  If $A$ is invertible, then $A^t$ is invertible, and $(A^t)\inv = (A\inv)^t$.  

{\keya{\bf{Invertible Matrix Theorem (extended).}}}  For a square ($n\times n$) matrix $A$, the following statements are all
equivalent to one another, and to the conditions previously stated in the Invertible Matrix Theorem:
\begin{itemize}
\item $A$ is invertible.
\item $A^t$ is invertible.
\item The rows of $A$ (viewed as vectors in $\R^n$) span $\R^n$.  
\item The rows of $A$ are linearly independent.  
\end{itemize}

 
\endedxtext


\beginedxproblem{Solve for A, again}{\dpa1}

Suppose that $A,B,C,D$ are all invertible $n\times n$ matrices such that
$B^t (A\inv)^t = D^t C\inv$.  Solve for $A$.  


\edXabox{type="multichoice" expect="$A=BD\inv C^t$" options="$A=D\inv C^t B$","$A=B\inv D\inv C^t$","$A=BD\inv C^t$","$A=D\inv (C\inv )^t B$","$A=DC^t B\inv$","$A=C^t D\inv B\inv$","$A=BC^t D\inv $","$A=C^t D\inv B$","None of the above"}


\edXsolution{ 
 }
 
\endedxproblem


\beginedxproblem{More IMT}{\dpa3}

Let $T: \R^4 \rightarrow \R^4$ be a linear transformation with standard matrix $A$.  Suppose the second row of $A$ is equal
to three times the last row of $A$ minus the first row of $A$.  Which of the following must be true?  Click all that apply.  

\edXabox{type="oldmultichoice" expect="$T$ is not onto","$A$ has a free variable","The kernel of $T$ contains infinitely many vectors","The columns of $A$ do not span $\R^4$" options="$T$ is not onto","$A$ is invertible","$A$ has a free variable","The kernel of $T$ contains infinitely many vectors","The equation $Ax = \left[\begin{array}{c} 1 \\ 3 \\ 5 \\7 \end{array} \right]$ has more than one solution","The columns of $A$ do not span $\R^4$"}


\edXsolution{
}

\endedxproblem



\endedxvertical


% 

\beginedxvertical{Wait for it}

\beginedxtext{Coming Soon}

Coming soon!

\endedxtext

\endedxvertical




\endedxsequential


\end{edXchapter}




\end{edXcourse}
\end{document}
