

\beginedxvertical{Page One}

\beginedxtext{Preliminaries}





At the end of this sequence, and after some practice, you should be able to:

\begin{itemize}
\item 


\end{itemize}


For time budgeting purposes, this sequence has X videos totaling X minutes, 
plus some questions.  




\endedxtext

\endedxvertical



\beginedxvertical{Subspaces}


\doedxvideo{Subspaces}{Q0kn88woMvU}


\beginedxproblem{Subset? Subspace?}{\dpa1}

Is $\R^4$ a subset of $\R^5$?  


\edXabox{expect="No" options="Yes","No"}

Is $\R^4$ a subspace of $\R^5$?  


\edXabox{expect="No" options="Yes","No"}

\edXsolution{
}

\endedxproblem

\endedxvertical



\beginedxvertical{Criteria for being a Subspace}

\doedxvideo{Subspace Criteria}{pQWBhzg_6Bc}

\beginedxtext{Subspaces}

{\keya{\bf{Definition.}}}  
Suppose $V$ is a vector space over a field $F$.  A subset $W \subset V$ is a {\keyb{\bf{subspace}}}  
of $V$ if $W$ is a vector space over $F$ in its own right, using the same vector operations as in $V$.  

If 


\endedxtext




\endedxvertical



\beginedxvertical{Into, Onto, and Invertibility}






\beginedxproblem{Into and Onto}{\dpa1}

We've noted that a linear transformation has an inverse if and only if it is both into and onto.  
Let's recall our results about when a linear transformation from $\R^n$ to $\R^m$ can or must
be into or onto.  

Suppose $T: \R^n \rightarrow \R^m$ is a linear transformation.


If $n<m$, what can we conclude about $T$?  

\edXabox{expect="T might be into, or might not" options="T must be into","T might be into, or might not","T cannot be into"}

\edXabox{expect="T cannot be onto" options="T must be onto","T might be onto, or might not","T cannot be onto"}


If $n>m$, what can we conclude about $T$?  

\edXabox{expect="T cannot be into" options="T must be into","T might be into, or might not","T cannot be into"}

\edXabox{expect="T might be onto, or might not" options="T must be onto","T might be onto, or might not","T cannot be onto"}



% If $n=m$, what can we conclude about $T$?  (Check all that apply.)  

% \edXabox{expect="T might be into, or might not" options="T must be into","T might be into, or might not","T cannot be into}

% \edXabox{expect="T might be onto, or might not" options="T must be onto","T might be onto, or might not","T cannot be onto}

\edXsolution{
}

\endedxproblem
\doedxvideo{Criteria}{pQWBhzg_6Bc}


\doedxvideo{Span as Subspace}{hjeYq1Ly0BE}


\beginedxproblem{When is invertible possible?}{\dpa1}

Again, let $T: \R^n \rightarrow \R^m$ be a linear transformation.  Given the above, 
if $n<m$, is it possible for $T$ to be invertible?

\edXabox{expect="No" options="Yes","No"}

If $n>m$, is it possible for $T$ to be invertible?

\edXabox{expect="No" options="Yes","No"}


\edXsolution{
}

\endedxproblem


\endedxvertical



\beginedxvertical{Conditions for Invertibility}




\doedxvideo{Conditions for Invertibility}{pYqavDw0wt4}


\beginedxproblem{Square Matrices}{\dpa1}

True or False: If a matrix $A$ is not square, it cannot be invertible.  

\edXabox{expect="True" options="True","False"}

True or False: If a matrix $A$ is square, it must be invertible.  

\edXabox{expect="False" options="True","False"}


\edXsolution{ Watch the last video again!  Non-square matrices cannot be invertible, but not 
every square matrix is invertible.  For instance, a square matrix of all zeros is not invertible.  

}
\endedxproblem


\endedxvertical



\beginedxvertical{Invertible Matrix Theorem}


\beginedxtext{Inverse Definitions}


\endedxtext



\beginedxproblem{Using the Invertible Matrix Theorem}{\dpa1}

Suppose $A$ is an $m\times n$ matrix which has at least one free variable.  Can we conclude that the columns of 
$A$ do not span $\R^m$?  

\edXabox{expect="No" options="Yes","No"}

Suppose $A$ is an $n\times n$ matrix which has at least one free variable. Can we conclude that the columns of $A$ do not span $\R^n$?  

\edXabox{expect="Yes" options="Yes","No"}

\edXsolution{ For the first question, we cannot make that conclusion, since we are not told the matrix is square.  For instance, $\left[ \begin{array}{ccc} 1 & 0 & 0 \\ 0 & 1 & 0 \end{array} \right]$ has a free variable,
but its columns span $\R^2$.  

For the second question, we know the matrix is square, so the Invertible Matrix Theorem applies.
Since $A$ has a free variable, we can conclude that none of the conditions in the Invertible Matrix Theorem
are true.  In particular, its columns do not span $\R^n$.  
}
\endedxproblem



\endedxvertical



\beginedxvertical{Calculating the Inverse}


\beginedxproblem{Is it Invertible?}{\dpa1}

Let $A = \left[ \begin{array}{cc} 1 & 2 \\ 2 & 5 \end{array} \right]$.  By row-reducing $A$, answer
the question, is $A$ invertible?

\edXabox{expect="Yes" options="Yes","No"}


\edXsolution{ $A$ row reduces to the identity matrix, so it is invertible.  
}

\endedxproblem




\endedxvertical


\beginedxvertical{Inverse Practice}




\beginedxproblem{Invert a Matrix}{\dpa3}

The following $3\times 3$ matrix is invertible:

\[A = \left[ \begin{array}{ccc} 1 & 1 & 2  \\
3 & 0  & -2  \\
-1 & 1 & 0
 \end{array} \right] \]

Find $A\inv$.  


To enter the matrix $\left[ \begin{array}{cc:c}
0&1&2 \\
1&2&3 \end{array} \right],$ type [[0,1,2],[1,2,3]]   

Use decimals only.  

\begin{edXscript}
def MatrixEntry(expect, ans):
  	import ast
	import numpy as np 
	ret= {'ok':False}
  	atol = 0.01
  	try:
		list_ans = ast.literal_eval(ans)
		list_expect = ast.literal_eval(expect)
  		matrix_ans = np.matrix(list_ans)
  		matrix_expect = np.matrix(list_expect) 
  		if matrix_ans.shape != matrix_expect.shape:
  			ret['msg'] = 'Wrong shape of matrix'
  		elif np.allclose(matrix_ans, matrix_expect,0.01,1e-08):
  			ret['ok'] = True
  		else:
  			ret['msg'] = 'Something is wrong'
	except SyntaxError:
		ret['msg'] = 'Wrong input format'
  	return ret
\end{edXscript}


\edXabox{type="custom" cfn="MatrixEntry" expect="[[0.2,0.2,-0.2],[0.2,0.2,0.8],[0.3,-0.2,-0.3]]"}

\edXsolution{

}

\endedxproblem


\beginedxproblem{Solve Using Inverse}{\dpa1}


Let  $A$ be the matrix from above.  Without doing any further row-reduction, find the solution
to $Ax = \left[ \begin{array}{c} 100 \\ 0 \\ 10 \end{array} \right].$

To enter a vector such as $\left[\begin{array}{c} 1 \\ 2  \\ 3 \end{array} \right]$, you can either:
\begin{itemize}
\item
enter it as 
you would a $3\times 1$ matrix: [[1],[2],[3]]  
\item
or, you can enter it as <1,2,3>
\end{itemize}

Use decimals only.  

\begin{edXscript}
def VectorEntry(expect, ans):
	import ast
	import numpy as np 
  	atol = 0.01
	list_expect = ast.literal_eval(expect)
	vec_expect = np.matrix(list_expect)
  	ret = {"ok":False}
	try:
  		# input format [[1],[2],[3]]
		list_ans = ast.literal_eval(ans)
		vec_ans = np.matrix(list_ans)
  		if vec_ans.shape != vec_expect.shape:
  			ret['msg'] = 'Wrong shape of vector!'
  		elif np.allclose(vec_ans, vec_expect,atol,1e-08):
  			ret['ok'] = True
  		else:
  		# More error message. Will improve this part
  			ret['msg'] = 'something is wrong'   			
	#except SyntaxError:
		#ret['msg'] = 'Wrong input format'
	except SyntaxError:
  		# input format &lt;1,2,3&gt;
		list_ans = ans.replace('&lt;', '[').replace('&gt;', ']')
		list_ans = ast.literal_eval(list_ans)
		vec_ans = np.transpose(np.matrix(list_ans))
  		if vec_ans.shape != vec_expect.shape:
  			ret['msg'] = 'Wrong shape of vector!'
  		elif np.allclose(vec_ans, vec_expect,0.01,1e-08):
  			ret['ok'] = True
  		else:
    		# More error message. Will improve this part
  			ret['msg'] = 'something is wrong' 
  	except:
  		ret['msg'] = 'Wrong input format'
  	return ret 
\end{edXscript}



\edXabox{type="custom" cfn="VectorEntry" expect="[[18],[28],[27]]"}

\edXsolution{ 

	
}


\endedxproblem


\endedxvertical



\beginedxvertical{One-Sided Inverses?}


\beginedxproblem{Non-square inverse? 1}{\dpa1}

Is it possible for a $2\times 3$ matrix to be invertible?  

\edXabox{expect="No" options="Yes","No"}


\edXsolution{ We have shown that non-square matrices cannot be invertible.  
}

\endedxproblem

\beginedxproblem{An Interesting Product}{\dpa3}

Given 
\[A = \left[ \begin{array}{ccc} 1 & 2 & 0  \\
0 & 1 & 1
 \end{array} \right], \  \mathrm{and} \ B = \left[ \begin{array}{cc} 1 & 0  \\
0 & 0 \\
0 & 1
 \end{array} \right], \]
calculate $AB$.  


\begin{edXscript}
def MatrixEntry(expect, ans):
  	import ast
	import numpy as np 
	ret= {'ok':False}
  	atol = 0.01
  	try:
		list_ans = ast.literal_eval(ans)
		list_expect = ast.literal_eval(expect)
  		matrix_ans = np.matrix(list_ans)
  		matrix_expect = np.matrix(list_expect) 
  		if matrix_ans.shape != matrix_expect.shape:
  			ret['msg'] = 'Wrong shape of matrix'
  		elif np.allclose(matrix_ans, matrix_expect,0.01,1e-08):
  			ret['ok'] = True
  		else:
  			ret['msg'] = 'Something is wrong'
	except SyntaxError:
		ret['msg'] = 'Wrong input format'
  	return ret
\end{edXscript}


\edXabox{type="custom" cfn="MatrixEntry" expect="[[1,0],[0,1]]"}

\edXsolution{

}

\endedxproblem

\beginedxproblem{Non-square inverse? 2}{\dpa1}

Look at the result of the previous problem.  
Does this mean that $A$ and $B$ are inverses of one another?

\edXabox{expect="No" options="Yes","No"}


\edXsolution{ No.  For $A$ and $B$ to be inverses, {\keyb{\bf{both}}}  $AB$ and $BA$ must be the identity.  
}

\endedxproblem


\endedxvertical


\beginedxvertical{One Side Only}

\doedxvideo{One-Sided Inverses}{X5sMleqtvAk}

\beginedxtext{One-Sided Inverses}

{\keya{\bf{Proposition.}}}  If  $A$ and $B$ are square matrices, and 
$AB = I$, then $BA = I$, and the two matrices are inverses of each other.  


This allows us to add two lines to the Invertible Matrix Theorem: 

{\keya{\bf{Invertible Matrix Theorem (updated).}}}  
Suppose that $T: \R^n \rightarrow \R^n$ is a linear transformation
with (square) standard matrix $A$.  Then the following conditions are all equivalent (i.e., either all of
them are true, or all of them are false):

\begin{enumerate}
\item $T$ is into.
\item $\mathrm{Ker}(T) = \{ \veco \}$.  
\item The columns of $A$ are linearly independent. 
\item $Ax = \veco$ has only the trivial solution.
\item $A$ row reduces to have a pivot in every column (has no free variables).  
\item $A$ row reduces to the identity matrix.  
\item $A$ row reduces to have a pivot in every row.  
\item For every $v\in \R^n$, the equation $Ax = v$ is consistent.
\item The columns of $A$ span $\R^n$.  
\item $T$ is onto.  
\item $T$ is invertible.
\item $A$ is invertible.
\item There is a matrix $B$ such that $AB = I_n$.  ($A$ has a right inverse.)
\item There is a matrix $B$ such that $BA = I_n$.  ($A$ has a left inverse.)  

\end{enumerate}



\endedxtext


\endedxvertical


\beginedxvertical{Some IMT Practice}

\beginedxproblem{Using the IMT}{\dpa1}

Let $T: \R^3 \rightarrow \R^3$ be a linear transformation with standard matrix $A$.  Suppose the columns of $A$ are linearly
independent.  Which of the following must be true?  Click all that apply.  

\edXabox{type="oldmultichoice" expect="$T$ is onto","$A$ is invertible","The equation $Ax = \left[\begin{array}{c} 1 \\ 3 \\ 5 \end{array} \right]$ has a unique solution","$T$ is into","$A$ row reduces to the identity matrix" options="$T$ is onto","$A$ is invertible","$A$ has a free variable","The equation $Ax = \left[\begin{array}{c} 1 \\ 3 \\ 5 \end{array} \right]$ has exactly one solution","$T$ is into","$A$ row reduces to the identity matrix"}


\edXsolution{
}

\endedxproblem


\endedxvertical