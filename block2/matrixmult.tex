

\beginedxvertical{Page One}

\beginedxtext{Preliminaries}





At the end of this sequence, and after some practice, you should be able to:

\begin{itemize}
\item Recognize when two matrices can be multiplied.  
\item Calculate the product of two matrices. 
\item Draw parallels between matrix multiplication and composition of linear transformations.
\end{itemize}


For time budgeting purposes, this sequence has X videos totaling X minutes, 
plus some questions.  




\endedxtext

\endedxvertical


\beginedxvertical{Composition of Linear Transformation}



\doedxvideo{Composition of Linear Transformations}{r-gtj-GO9WI}

\beginedxtext{Fundamental Fact of Matrix Multiplication}

{\keya{\bf{Proposition.}}}  
If $S: \R^p \rightarrow \R^n$ and $T: \R^n \rightarrow \R^m$ are linear transformations, then $T\circ S: \R^p \rightarrow \R^m$ is a linear transformation.  

{\keya{\bf{Fundamental Fact of Matrix Multiplication.}}}  
If the $m\times n$ matrix $A$ is the standard matrix for $T: \R^n \rightarrow \R^m$, and the $n\times p$ matrix $B$ is the standard matrix for $S: \R^p \rightarrow \R^n$, then $AB$ will be an $m\times p$ matrix which is the standard matrix for
$T\circ S: \R^p \rightarrow \R^m$.  

(If the number of columns of $A$ is not equal to the number of rows of $B$, then $AB$ is not defined, just as the composition $T\circ S$ would not  be defined.)  

\endedxtext




\endedxvertical



\beginedxvertical{Questions about Composition}






\beginedxproblem{Composing}{\dpa1}

Suppose $T_1: \R^7 \rightarrow \R^4$ and $T_2: \R^3 \rightarrow \R^7$ are both linear transformations.

Is $T_1\circ T_2$ a linear transformation?  

\edXabox{expect="Yes" options="Yes","No"}

Is $T_2\circ T_1$ a linear transformation?  

\edXabox{expect="No" options="Yes","No"}

\edXsolution{
}

\endedxproblem


\beginedxproblem{Matrix Sizes}{\dpa2}

Let $C_1$ be the standard matrix for $T_1$ from the previous question, and let $C_2$ be the
standard matrix for $T_2$.  

What size is $C_1$?  

\edXabox{expect="4 x 7" options="4 x 7","7 x 4"}

What size is $C_2$?  

\edXabox{expect="7 x 3" options="3 x 7","7 x 3"}


\edXsolution{ 

}
\endedxproblem


\beginedxproblem{Product Sizes}{\dpa2}

What size is $C_1C_2$?  


\edXabox{type="multichoice" expect="4 x 3" options="4 x 7","4 x 3","3 x 4","7 x 3","7 x 7","Not defined"}

What size is $C_2C_1$?  

\edXabox{type="multichoice" expect="Not defined" options="4 x 7","4 x 3","3 x 4","7 x 3","7 x 7","Not defined"}

\edXsolution{ 

}
\endedxproblem


\endedxvertical


\beginedxvertical{Calculating the Product}


\doedxvideo{Calculating the Product}{QbFZTxWh_9w}

\beginedxtext{Definition of Matrix Product}

{\keya{\bf{Definition.}}}  
If $A$ is an $m\times n$ matrix, and $B$ is an $n\times p$ matrix, then $AB$ is the $m\times p$ matrix
whose $i$th column is $A$ times the $i$th column of $B$.  

\endedxtext



\beginedxproblem{Which product?}{\dpa1}

Consider the matrices
\[ C = \left[\begin{array}{cc} 1 & -1 \\ 0 & 2  \\ -1 & 1  \end{array} \right]. \]
and 
\[ D = \left[\begin{array}{ccc} 1 & 2 & 3 \\ 0 & 1 & 1 \\ -1 & 4 & 3  \end{array} \right] \]

Of $CD$ and $DC$, which matrix product is defined?  

\edXabox{expect="DC" options="CD","DC","Both are defined"}


\edXsolution{ 

}

\endedxproblem



\beginedxproblem{Compute the Product}{\dpa1}

From the problem above, enter the product which is defined.  (If both are, you may enter either one.)  

 
To enter the matrix $\left[ \begin{array}{cc:c}
0&1&2 \\
1&2&3 \end{array} \right],$ type [[0,1,2],[1,2,3]]   

Use decimals only.  

\begin{edXscript}
def MatrixEntry(expect, ans):
  	import ast
	import numpy as np 
	ret= {'ok':False}
  	atol = 0.01
  	try:
		list_ans = ast.literal_eval(ans)
		list_expect = ast.literal_eval(expect)
  		matrix_ans = np.matrix(list_ans)
  		matrix_expect = np.matrix(list_expect) 
  		if matrix_ans.shape != matrix_expect.shape:
  			ret['msg'] = 'Wrong shape of matrix'
  		elif np.allclose(matrix_ans, matrix_expect,0.01,1e-08):
  			ret['ok'] = True
  		else:
  			ret['msg'] = 'Something is wrong'
	except SyntaxError:
		ret['msg'] = 'Wrong input format'
  	return ret
\end{edXscript}


\edXabox{type="custom" cfn="MatrixEntry" expect="[[-2,6],[-1,3],[-4,12]]"}

\edXsolution{

}

\endedxproblem




\beginedxproblem{Compute the Products}{\dpa1}

Let $A =  \left[\begin{array}{cc} 1 & -1 \\ 2 & 0  \end{array} \right] $
and $B = \left[\begin{array}{cc} 0 & 3 \\ 1 & 2  \end{array} \right] $.
 
To enter the matrix $\left[ \begin{array}{cc:c}
0&1&2 \\
1&2&3 \end{array} \right],$ type [[0,1,2],[1,2,3]]   

Use decimals only.  

\begin{edXscript}
def MatrixEntry(expect, ans):
  	import ast
	import numpy as np 
	ret= {'ok':False}
  	atol = 0.01
  	try:
		list_ans = ast.literal_eval(ans)
		list_expect = ast.literal_eval(expect)
  		matrix_ans = np.matrix(list_ans)
  		matrix_expect = np.matrix(list_expect) 
  		if matrix_ans.shape != matrix_expect.shape:
  			ret['msg'] = 'Wrong shape of matrix'
  		elif np.allclose(matrix_ans, matrix_expect,0.01,1e-08):
  			ret['ok'] = True
  		else:
  			ret['msg'] = 'Something is wrong'
	except SyntaxError:
		ret['msg'] = 'Wrong input format'
  	return ret
\end{edXscript}


What is $AB$?  

\edXabox{type="custom" cfn="MatrixEntry" expect="[[-1,1],[0,6]"}

What is $BA$?  

\edXabox{type="custom" cfn="MatrixEntry" expect="[[6,0],[5,-1]"}


Is $AB$ the same as $BA$?

\edXabox{expect="No" options="Yes","No"}


\edXsolution{

}

\endedxproblem





\endedxvertical

\beginedxvertical{Properties of Matrix Multiplication}



\doedxvideo{Commutative?  Associative?}{D80IuF4bedo}


\beginedxtext{Associativity of Multiplication}

{\keya{\bf{Proposition.}}}  
If $A$ is an $m\times n$ matrix, $B$ is an $n\times p$ matrix, and $C$ is a $p\times q$ matrix,
then $A(BC) = (AB)C$.  In other words, matrix multiplication is associative.  

\endedxtext


\endedxvertical



\beginedxvertical{Applying the Fundamental Fact of Matrix Multiplication}


\doedxvideo{A Proposition}{HuitPIiVBZ4}



\endedxvertical