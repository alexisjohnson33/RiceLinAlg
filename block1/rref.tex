

\beginedxvertical{Page One}

\beginedxtext{Preliminaries}





At the end of this sequence, and after some practice, you should be able to:

\begin{itemize}
\item Use elementary {\keyb{\bf row operations}} to put matrices into {\keyb{\bf row-reduced echelon form (RREF)}}.
\item Interpret RREF matrices to determine {\keyb{\bf existence and uniqueness}} of solutions.  
\end{itemize}


For time budgeting purposes, this sequence has X videos totaling X minutes, 
plus some questions.  




\endedxtext

\endedxvertical


\beginedxvertical{Row-Reduced Echelon Form}

\doedxvideo{RREF}{???}



\endedxvertical


\beginedxvertical{RREF Defined}

\beginedxtext{Row-Reduced Echelon Form}


A matrix is in Row-Reduced Echelon Form (RREF) if it satisfies the following conditions:

\begin{itemize}
\item All rows of all zeros are at the bottom.
\item In any non-zero row, the first non-zero entry is 1.  (This is called a {\keyb{\bf pivot}}.) 
\item Each pivot is the only non-zero entry in its column.
\item A pivot in a lower row is to the right of any pivot in a higher row.  
\end{itemize}



\endedxtext


\beginedxproblem{RREF or not?}{\dpa2}

Which of the following matrices are in RREF?  Click all that apply.  



\edXabox{type="oldmultichoice" expect="mx1" options="mx1","mx2"}

\edXsolution{  }

\endedxproblem



\beginedxproblem{Make it RREF}{\dpa2}

The following matrix is not in RREF.

However, you can apply a single row operation to put it into RREF.  What matrix is the
result?  



\edXabox{expect="Placeholder" options="Placeholder"}

\edXsolution{  }

\endedxproblem



\beginedxproblem{Pivots vs. Rows}{\dpa2}

True or False: If a matrix that is in RREF has 6 rows, then it must have 6 pivots.  


\edXabox{expect="False" options="True","False"}

\edXsolution{  }

\endedxproblem

\endedxvertical



\beginedxvertical{Interpretations}

\doedxvideo{The Row Reduction Algorithm}{XXX}



\beginedxtext{The Row-Reduction Algorithm}

The following algorithm will always work to reduce a matrix into Row-Reduced Echelon Form.  

\begin{itemize}
\item Swap two rows to get the left-most remaining non-zero entry to the topmost remaining row.  
\item Scale that row to make its leading entry a 1.    
\item Clear all non-zero entries in the column of that pivot (both above and below the pivot) by adding/subtracting multiples of that row to/from other rows.  
\item With that pivot now set, repeat the procedure with the remaining rows.  
\end{itemize}

\endedxtext

This is sometimes called {\keyb{\bf  Gaussian elimination}}.  

\endedxvertical


\beginedxvertical{Row-Reducing Practice}


\beginedxproblem{Row-reduction Practice}{\dpa1}

Let 
\[
A = \left[ \begin{array}{ccc} 0 & 1 & 2 \\ 3 & 0 & -3 \\ 2 & 2 & 5 \\ 0 & -2 & 1\end{array} \right]. \]

What matrix do you get if you row-reduce $A$? 


\edXabox{expect="Placeholder" options="Placeholder"}

\edXsolution{ 
}


\endedxproblem


\beginedxproblem{System}{\dpa1}

Consider the augmented matrix
\[
A = \left[ \begin{array}{ccc:c} 0 & 1 & 2 & -1 \\ 3 & 0 & -3 & 3 \\ 
2 & 2 & 5 & 3\\ 0 & -2 & 1 & 7\end{array} \right]. \]


Note that the coefficient matrix is the same as the matrix $A$ from the previous question.

Applying the row-reduction algorithm to this augmented matrix, what can we conclude about
the system?  

\edXabox{type="multichoice" expect="The system is consistent and has a unique solution" options="The system is consistent and has a unique solution","The system is consistent but has more than one solution","The system is inconsistent"}

\edXsolution{ 
}


\endedxproblem



\endedxvertical



\beginedxvertical{Interpretations}

\doedxvideo{Interpreting a RREF matrix}{XXX}





\beginedxproblem{Matrix 1}{\dpa1}

Suppose you do some row operations to an augmented matrix, and the result is the following augmented
matrix:
\[
\left[ \begin{array}{cc:c} 1 & -3 & -2 \\ 0 & 0 & 1 \\ 0 & 0 & 0 \end{array} \right] \]

How many free variables are there?
\edXabox{type="numerical" expect="1"}

What can you conclude about the original system of equations?  

\edXabox{type="multichoice" expect="The system is inconsistent" options="The system is consistent and has a unique solution","The system is consistent but has more than one solution","The system is inconsistent"}

\edXsolution{ 
}

\endedxproblem

\beginedxproblem{Matrix 2}{\dpa1}

Suppose you do some row operations to an augmented matrix, and the result is the following augmented
matrix:
\[  
\left[ \begin{array}{cc:c} 1 & 0 & 0 \\ 0 & 1 & 1 \\ 0 & 0 & 0 \end{array} \right] \]


How many free variables are there?
\edXabox{type="numerical" expect="0"}

What can you conclude about the original system of equations?  

\edXabox{type="multichoice" expect="The system is consistent and has a unique solution" options="The system is consistent and has a unique solution","The system is consistent but has more than one solution","The system is inconsistent"}

\edXsolution{ 
}

\endedxproblem


\beginedxproblem{Matrix 3}{\dpa1}

Suppose you do some row operations to an augmented matrix, and the result is the following augmented
matrix:
\[ 
\left[ \begin{array}{cccc:c} 1 & 0 & 0 & 1 & -2 \\ 0 & 0 & 1 & 3 & 3 \\ 0 & 0 & 0 & 0 & 0 \end{array} \right] \]


How many free variables are there?
\edXabox{type="numerical" expect="2"}



What can you conclude about the original system of equations?  

\edXabox{type="multichoice" expect="The system is consistent but has more than one solution" options="The system is consistent and has a unique solution","The system is consistent but has more than one solution","The system is inconsistent"}

\edXsolution{ 
}

\endedxproblem


\beginedxproblem{Free Variables}{\dpa1}

If a system of linear equations has free variables when row-reduced, is it possible for the
system to be inconsistent?  

\edXabox{expect="Yes" options="Yes","No"}

\edXsolution{ 
}

\endedxproblem

\endedxvertical


\beginedxvertical{Finding Solutions}




\beginedxproblem{Solutions to a System}{\dpa1}

Recall the augmented matrix from the last problem:

\[ 
\left[ \begin{array}{cccc:c} 1 & 0 & 0 & 1 & -2 \\ 0 & 0 & 1 & 3 & 3 \\ 0 & 0 & 0 & 0 & 0 \end{array} \right] \]

Working from right to left as in the last video, what are all possible solutions to this system?
Type the word `free' (lower-case) if the variable is a free variable.  For others, remember to use * for times and type 
xn for the variable $x_n$.  For instance, to enter $-3x_2 + 2x_4$, type -3*x2 + 2*x4.  

\edXinline{$x_4 = $}\edXabox{type="formula" expect="free" samples="free,x3,x2,x1@1,1,1,1:,4,4,4,4#10" tolerance=".001" feqin="1" inline="1"}

\edXinline{$x_3 = $}\edXabox{type="formula" expect="3-3*x4" samples="free,x4,x2,x1@1,1,1,1:4,4,4,4#10" tolerance=".001" feqin="1" inline="1"}

\edXinline{$x_2 = $}\edXabox{type="formula" expect="free" samples="free,x4,x3,x1@1,1,1,1:4,4,4,4#10" tolerance=".001" feqin="1" inline="1"}

\edXinline{$x_1 = $}\edXabox{type="formula" expect="-2-x4" samples="free,x4,x3,x2@1,1,1,1:4,4,4,4#10" tolerance=".001" feqin="1" inline="1"}


\edXsolution{ 
}

\endedxproblem


\endedxvertical




\beginedxvertical{Going Further}



\beginedxtext{The Size of a Matrix}

If a matrix $M$ has $m$ rows and $n$ columns, we say that 
$M$ is an $m\times n$ matrix.  For instance, 

\[ 
\left[ \begin{array}{cccc:c} 1 & 0 & 0 & 1 & -2 \\ 0 & 0 & 1 & 3 & 3 \\ 0 & 0 & 0 & 0 & 0 \end{array} \right] \]

is a $3\times 5$ augmented matrix, since it has 3 rows and 5 columns.  The coefficient matrix of the above system would be a $3\times 4$ matrix.  


\endedxtext




\beginedxproblem{Matrix Size}{\dpa1}

Let 
\[
A = \left[ \begin{array}{ccc} 1 & -3 & -2 \\ 0 & 0 & 1 \\ 0 & 0 & 0 \\ 0 & 0 & 0\end{array} \right]. \]

What is the size of $A$? 
\edXabox{type="multichoice" expect="$4\times 3$" options="$3\times 4$","$4\times 3$"}

\edXsolution{ 
}

\endedxproblem


\beginedxproblem{Maximum pivots}{\dpa1}

What is the maximum number of pivots in a $6\times 3$ coefficient matrix, when row-reduced?  

\edXabox{type="numerical" expect="3"}


\edXsolution{ 
}

\endedxproblem

\beginedxproblem{Maximum pivots 2}{\dpa1}

What is the maximum number of pivots in a $4\times 6$ coefficient  matrix, when row-reduced?  

\edXabox{type="numerical" expect="4"}


\edXsolution{ 
}

\endedxproblem

\beginedxproblem{Number of Free Variables}{\dpa1}

What is the strongest statement you can make about the number of free variables in a $4\times 6$ coefficient  matrix, when row-reduced?  

\edXinline{The number of free variables must be  }\edXabox{expect="at least" options="exactly","at least","at most" inline="1"} \edXabox{type="numerical" expect="2" inline="1"}

\edXsolution{ 
}

\endedxproblem

\beginedxproblem{4 Equations, 6 Variables}{\dpa1}

Given the answer to the previous question, which of the following results are possible for a system of 4 linear equations in 6 variables?

\edXabox{type="oldmultichoice" expect="The system is consistent but has more than one solution","The system is inconsistent" options="The system is consistent and has a unique solution","The system is consistent but has more than one solution","The system is inconsistent"}

\edXsolution{ 
}

\endedxproblem



\endedxvertical


\beginedxvertical{Fields}



\beginedxproblem{Real Numbers}{\dpa1}

If a matrix has real number entries, will the entries stay real numbers when you 
row-reduce it?  

\edXabox{expect="Yes" options="Yes","Not necessarily"}

\edXsolution{ 
}

\endedxproblem

\beginedxproblem{Rational Numbers}{\dpa1}

If a matrix has rational number entries, will the entries stay rational numbers when you 
row-reduce it?  (Recall that rational numbers are those that can be written as fractions
$\frac{p}{q}$,
where $p,q$ are integers.)  

\edXabox{expect="Yes" options="Yes","Not necessarily"}

\edXsolution{ 
}

\endedxproblem

\beginedxproblem{Integers}{\dpa1}

If a matrix has integer entries, will the entries stay integers when you 
row-reduce it?  

\edXabox{expect="Not necessarily" options="Yes","Not necessarily"}

\edXsolution{ 
}

\endedxproblem


\doedxvideo{Fields}{XXX}


\endedxvertical


\beginedxvertical{Fields Defined}


\beginedxtext{Fields}

A {\keyb{\bf field}} is, essentially, a set in which one can add, subtract, multiply, and
divide any two elements (other than dividing by zero).  The formal definition is below.  

The field that we will
most commonly use in this course is the set of real numbers $\R$.  Other fields include
the set of complex numbers $\C$, the set of rational numbers $\Q$, and the integers modulo
a prime $p$.  

\begin{edXshowhide}{Formal Definition of a Field}
A field is a set $F$ together with binary operations $+, \cdot$ on $F$ which satisfy the
following conditions:

\begin{itemize}
\item Addition is associative and commutative; that is, for all $a,b,c \in F$, we have
$a+(b+c) = (a+b)+c$ and $a+b  = b+a$.  
\item There is an element $0 \in F$ such that $a+0 = a$ for all $a \in F$.
\item For every $a\in F$, there is an element $-a \in F$ such that $-a + a = 0$.  
\item Multiplication is associative and commutative; that is, for all $a,b,c \in F$, we have
$a(bc) = (ab)c$ and $ab  = ba$.  
\item There is a non-zero element $1 \in F$ such that $1a = a$ for all $a \in F$.  
\item For every non-zero $a\in F$, there is an element $a\inv$ such that $a\inv a = 1$.  
\item Multiplication distributes over addition; that is, for all $a,b,c \in F$, $a(b+c) = ab + ac$.  
\end{itemize}

\end{edXshowhide}




\endedxtext

\endedxvertical

