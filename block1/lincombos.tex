

\beginedxvertical{Page One}

\beginedxtext{Preliminaries}





At the end of this sequence, and after some practice, you should be able to:

\begin{itemize}
\item Describe the span of some lists of vectors.
\item Determine whether a list of vectors in $\R^m$ spans $\R^m$.  
\end{itemize}


For time budgeting purposes, this sequence has X videos totaling X minutes, 
plus some questions.  




\endedxtext

\endedxvertical


\beginedxvertical{Introduction to Linear Combinations}

\doedxvideo{Linear Combinations}{???}

\beginedxtext{Definition of Linear Combination}

Let $V$ be a vector space over a field $F$.  A {\keyb{\bf linear combination}} of a list of vectors $\{v_1; v_2; 
\ldots v_n\}$
in $V$ is a vector of the form $a_1v_1 + a_2v_2 + \ldots + a_n v_n$, where the $a_i$ are scalars in $F$.  


\endedxtext


\endedxvertical


\beginedxvertical{Linear Combinations}


\beginedxproblem{Linear Combo 1}{\dpa1}


Let $v_1 = \left[\begin{array}{c} 2 \\ 3  \\ 4 \end{array} \right]$ and  
$v_2 = \left[\begin{array}{c} 1 \\ -1  \\ 0 \end{array} \right]$ be vectors in $\R^3$.

Can you write the vector $x  = \left[\begin{array}{c} -4 \\ -6  \\ -8 \end{array} \right]$
as a linear combination $x = a_1 v_1 + a_2 v_2$?  If so, enter possible values for $a_1$ and
$a_2$.  If not, enter No in both boxes.  

\edXabox{expect="Placeholder" options="Placeholder"}


\edXsolution{ 
}


\endedxproblem

\beginedxproblem{Linear Combo 2}{\dpa1}


Let $v_1 = \left[\begin{array}{c} 2 \\ 3  \\ 4 \end{array} \right]$ and  
$v_2 = \left[\begin{array}{c} 1 \\ -1  \\ 0 \end{array} \right] $  be vectors in $\R^3$.

Can you write the zero vector in $\R^3$
as a linear combination $\veco = a_1 v_1 + a_2 v_2$?  If so, enter possible values for $a_1$ and
$a_2$.  If not, enter No in both boxes.  

\edXabox{expect="Placeholder" options="Placeholder"}


\edXsolution{ 
}


\endedxproblem


\beginedxproblem{Linear Combo 3}{\dpa1}


Let $v_1 = \left[\begin{array}{c} 2 \\ 3  \\ 4 \end{array} \right]$ and  
$v_2 = \left[\begin{array}{c} 1 \\ -1  \\ 0 \end{array} \right]$  be vectors in $\R^3$.

Can you write $v_2$
as a linear combination of $v_1$?   

\edXabox{expect="No" options="Yes","No"}


\edXsolution{ 
}


\endedxproblem


\beginedxproblem{Linear Combo 4}{\dpa1}

\begin{center}
\includesvg[300]{placeholder}   
\end{center}

Given the picture above in $\R^2$, 
can you write $w$ as a linear combination $w = a_1 v_1 + a_2 v_2$?  If so, enter possible values for $a_1$ and
$a_2$.  If not, enter No in both boxes.  

\edXabox{expect="Placeholder" options="Placeholder"}


\edXsolution{ 
}

\endedxproblem

\beginedxproblem{Linear Combo 5}{\dpa1}


Let $f$ denote the polynomial $x^3  + 1$, and let $g$ denote the
polynomial $x^2 - x$.  Recall that the set of polynomials is a vector space over $\R$, so $f$ and $g$
are vectors in that vector space.  

Let $h$ denote $17x^3 + 2x^2 - 2x + 17$.
Can you write $h$  as a linear combination $h = a_1 f + a_2 g$?  If so, enter possible values for $a_1$ and
$a_2$.  If not, enter No in both boxes.  

\edXabox{expect="Placeholder" options="Placeholder"}

Let $h$ denote $3x^4 + 3x$.
Can you write $h$  as a linear combination $h = a_1 f + a_2 g$?  If so, enter possible values for $a_1$ and
$a_2$.  If not, enter No in both boxes.  

\edXabox{expect="Placeholder" options="Placeholder"}

\edXsolution{ 
}


% \beginedxproblem{Linear Combo 5}{\dpa1}


% Let $v_1 = \left[\begin{array}{c} 0 \\ 37  \\ 2i \end{array} \right]$ and  
% $v_2 = \left[\begin{array}{c} 7i \\ -19  \\ 0 \end{array} \right]$ be vectors in the vector space $\C^3$ (over the scalar
% field $\C$).  

% Can you write the vector $x  = \left[\begin{array}{c} -7i \\ 19 + 37i  \\ -2 \end{array} \right]$
% as a linear combination $x = a_1 v_1 + a_2 v_2$?  If so, enter possible values for $a_1$ and
% $a_2$.  If not, enter No in both boxes.  (Use i for $i$ and * for multiplication.)  

% \edXabox{expect="Placeholder" options="Placeholder"}


% \edXsolution{ 
% }

\endedxvertical


\beginedxvertical{Is it a linear combination?}



\doedxvideo{Linear Combinations in R^n}{???}


\beginedxproblem{Linear Combo 6}{\dpa1}


Let $v_1 = \left[\begin{array}{c} 1 \\ 3  \\ 1 \end{array} \right]$ and  
$v_2 = \left[\begin{array}{c} 3 \\ 10  \\ 2 \end{array} \right]$  be vectors in $\R^3$.

Use row reduction to determine the answer to the following: 
Can you write $w = \left[\begin{array}{c} 1 \\ 5 \\ -1 \end{array} \right]$
as a linear combination $w = a_1 v_1 + a_2 v_2$?  If so, enter possible values for $a_1$ and
$a_2$.  If not, enter No in both boxes.  

\edXabox{expect="Placeholder" options="Placeholder"}


\edXsolution{ 
}


\endedxproblem

\endedxvertical


\beginedxvertical{Span as Noun}


\doedxvideo{Span as a Noun}{???}

\endedxvertical


\beginedxvertical{Span Questions}

\beginedxtext{Span (Noun)}

Given vectors $v_1, v_2, \ldots v_n$ in a vector space $V$ over a field $F$, we define
the {\keyb{\bf span}} of $\{v_1; v_2; \ldots v_n\}$ to be the set of all linear combinations
of  $v_1, v_2, \ldots v_n$.  In other words, 
\[ \mathrm{Span}\{v_1; v_2; \ldots v_n\} = \{a_1 v_1  + a_2 v_2 + \ldots + a_nv_n  : a_i \in F \}.\]

The span of a list with zero vectors is defined to be $\{\veco\}$.  

% In $\R^m$,...

\endedxtext



\beginedxproblem{Span Question}{\dpa1}


Let $v_1 = \left[\begin{array}{c} 1 \\ 1  \\ 0 \end{array} \right]$, 
$v_2 = \left[\begin{array}{c} 0 \\ -1  \\ 2 \end{array} \right]$,
$v_3 = \left[\begin{array}{c} -1 \\ -2  \\ 3 \end{array} \right]$, and
$v_4 = \left[\begin{array}{c} -2 \\ -5  \\ 8 \end{array} \right]$.  

Use row reduction to determine if  $w = \left[\begin{array}{c} 3 \\ 1 \\ 3 \end{array} \right]$
is in the span of $\{v_1; v_2; v_3; v_4 \}$.  Is $w \in \mathrm{Span}\{v_1; v_2; v_3; v_4 \}$?   

\edXabox{expect="Yes" options="Yes","No"}


\edXsolution{ 
}


\endedxproblem

\beginedxproblem{Span Question 2}{\dpa1}

Now determine if  $u = \left[\begin{array}{c} 301.34 \\ -234.8 \\ \pi \end{array} \right]$
is in the span of $\{v_1; v_2; v_3; v_4 \}$.  Try to avoid doing any additional calculation!

Is $u \in \mathrm{Span}\{v_1; v_2; v_3; v_4 \}$?   

\edXabox{expect="Yes" options="Yes","No"}


\edXsolution{ 
}


\endedxproblem


\endedxvertical


\beginedxvertical{Span as Verb}


\doedxvideo{Span as a Verb}{???}


\beginedxtext{Span (Verb)}


In $\R^m$,...


\endedxtext

\beginedxtext{Theorem on Spanning in R^m}


In $\R^n$,...


\endedxtext

\endedxvertical
